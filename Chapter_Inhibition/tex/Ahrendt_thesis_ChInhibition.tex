%%%%%%%%%%%%%%%%%%%%%%%%%%%%%%%%%%%%%%%%%%%%%%%%%%%%%%%%%%%%%%%%%%%%%%%%%%%%%%%%%
%% Document: Thesis for PhD at UC Riverside                                    %%
%% Title: Investigating the evolution of environmental and biotic interactions %%
%%          in basal fungal lineages through comparative genomics              %%
%% Author: Steven Ahrendt                                                      %%
%%%%%%%%%%%%%%%%%%%%%%%%%%%%%%%%%%%%%%%%%%%%%%%%%%%%%%%%%%%%%%%%%%%%%%%%%%%%%%%%%
% INHIBITION CHAPTER %
%%%%%%%%%%%%%%%%%%%%%%
\chapter{Growth suppression properties of the Chytridiomycete \textit{Homolaphlyctis polyrhiza} JEL142}
\label{chap:HpInhibition}
\section{Introduction}
Interactions between microorganisms are facilitated by biological signals. These include proteins, small molecules, and various chemical compounds, either bound to the cell surface or secreted into the environment. Many of these compounds can be classified as secondary metabolites: chemicals not required for growth or development of the organism. \\
\begin{itemize}
  \item Resource competition likely plays a role in the evolution of natural antifungal production \cite{Vicente2003}. Secretion by an organism in a resource-limited environment of secondary metabolites which also happen to negatively impact neighboring organisms would confer a selective advantage upon the producer.\\
  \item There are many examples of fungal pathogens of economically important crops (eg. \textit{Ashbya gossypii}) and humans (eg. \textit{Aspergillus fumigatus}). Research into new natural antifungal treatment is increasingly important as certain pathogenic fungi become progressively resistant to current, conventionally synthesized antifungal drugs (see reviews by Bossche et al. in 1998\nocite{Bossche1998}, Kontoyiannis and Lewis in 2002\nocite{Kontoyiannis2002}, and Pfaller in 2012\nocite{Pfaller2012}. \\
  \item There are numerous examples of fungi as potent producers of these types of compounds, including toxins \cite{Kokkonen2010}, antimicrobials \cite{Wiemann2014}, and plant hormone mimics \cite{Howlett2006}.\\ 
\end{itemize}
\indent The bulk of research into fungal-derived secondary metabolites has thus far focused primarily on Ascomycete and Basidiomycete fungi due to their diverse product types \cite{Berdy2012}. However, chytrids are a relatively understudied group, both in general and in this regard specifically, and as such represent an excellent source for further lines of inquiry. Comparative genomic analyses have identified a variety of degradation enzymes in their genomes, but the full extent of competition-based secondary metabolite production has yet to be explored. It is likely that these organisms have evolved mechanisms for mediating chemical interactions with other organisms, and therefore could present a potentially valuable source of novel antifungal compounds.\\
\indent \textit{Homolaphlyctis polyrhiza} (\textit{Hp}) is a non-pathogenic member of the Chytridiomycota and is most closely related to the amphibian pathogen \textit{B. dendrobatidis} (\textit{Bd}). The specific isolate, \textit{Hp} JEL 142, has been used previously in phylogenetic studies of chytrids \cite{James2000,James2006,Letcher2008} and was provided a formal name in 2011 \cite{Longcore2011}. A draft 454 genome assembly was produced in that same year in order to gain comparative insights into pathogenicity of \textit{Bd} \cite{Joneson2011}.\\
\indent This isolate of \textit{Hp} was collected in Maine, USA from a 1.7 ha oligotrophic and fishless lake with a pH of 4.6 \cite{Davis1994,Rhodes1995}. It was cultured using onionskin bait and isolated into pure culture on mPmTG nutrient agar. Its overall morphology was deemed typical of members of the Rhizophydiales, but phylogenetic reanalysis prompted the description of a novel genus and species \cite{Longcore2011}.\\
\indent While working with \textit{Hp} in the Stajich lab for the phototaxis and protein expression experiments described in Chapter~\ref{chap:RhodAux}, unintentional contamination of \textit{Hp} plates by \textit{N. crassa} led to the observation of growth inhibition of the latter by the former. Specifically, I first noticed the distinctive zone created by \textit{N. crassa} hyphae growing near \textit{Hp} sporangia. I also had previously observed \textit{N. crassa} contamination on plates of related Chytridiomycetes \textit{Batrachochytrium dendrobatidis} (\textit{Bd}) and \textit{Spizellomyces punctatus} (\textit{Sp}), yet these species did not demonstrate any inhibitory effect against \textit{N. crassa}.\\
\indent This property of \textit{Hp} is reliable and reproducible. This chapter, therefore, describes the experimental and computational work towards identification and characterization of a compound responsible for hyphal growth suppression of \textit{N. crassa} by \textit{Hp}. This represents the first study of \textit{Hp} in this context and presents novel findings about microbial interactions within early branching fungal lineages. Significant help, specifically concerning media prep and general strain maintenance, was provided by undergraduate students in the Stajich lab Na Jeong and Sapphire Ear, and visiting student in the Research Experience for Undergraduates (REU) program Spencer Swansen.\\
\section{Methods}
\subsection*{Fungal strains and maintenance}
Cultures of \textit{Homolaphlyctis polyrhiza} JEL142, \textit{Operculomyces laminatus} JEL223, \textit{Rhizoclosmatium hyalinus} JEL800, and \textit{Obelidium mucronatum} JEL802 were individually maintained on mPmTG [peptonized milk (0.4 g/L), tryptone (0.4 g/L), dextrose (2 g/L)], \textit{Batrachochytrium dendrobatidis} JEL423 cultures were maintained on 1\% Tryptone [tryptone (10 g/L), dextrose (3.2 g/L)], and \textit{Spizellomyces punctatus} SW-1 cultures were maintained on PmTG [peptonized milk (0.5 g/L), tryptone (1 g/L), dextrose (5 g/L)]. All chytrid cultures were maintained at room temperature (approx. 23$^{\circ}$C. Unless otherwise specified, all experiments using chytrids were carried out using the specific media described above for each species. Motile chytrid zoospores, when required, were obtained from actively growing (ie 2-4 day old) plates by flooding and subsequently (after 30-45 minutes at room temperature) collecting 2-4 mls of sterile di H$_{2}$O. Sporangia samples for inoculation were obtained by removing a block (approx. 1 cm$^{2}$) of agar containing actively-growing chytrid sporangia.\\
\indent Vogel's minimal medium (VM) \cite{Davis1970} was used for vegetative growth of \textit{Neurospora crassa} FGSC 2489, \textit{Neurospora discreta} FGSC 8579 and \textit{Neurospora tetrasperma} FGSC 2508. \textit{Neurospora crassa} kinase/phosphotase mutants (Table~\ref{tab:kinase}; \cite{Park2011}) were maintained on VM + Hygromycin and grown similarly to \textit{N. crassa} WT strains. \textit{Trichoderma reesei} FGSC 10290 , \textit{Phycomyces blakesleeanus}, and \textit{Ashbya gossypii} cultures were maintained on PDA media [potato dextrose agar (39 g/L)] at 28$^{\circ}$C, 20$^{\circ}$C, and 30$^{\circ}$C, respectively. \textit{Aspergillus nidulans} FGSC A4 cultures were maintained on minimal media [dextrose (10 g/L), nitrate salts (50 mL/L), Trace elements (1 ml/L)], at 28$^{\circ}$C. \textit{Saccharomyces cerevisiae} strains MAU99 and AH109 were maintained on YPDA [yeast extract (10 g/L), peptone (20 g/L), dextrose (20 g/L), adenine sulfate (0.003\%)] media at 35$^{\circ}$C. \textit{Corpinopsis cinerea} FGSC 9003 cultures were maintained on YPD media at 37$^{\circ}$C.  Plates and slants of 1\% tryptone, PmTG, mPmTG, VM, YPD, and PDA media included 1\% agar. Plates and slants of MM included 1.8\% agar. Plates and slants of YPDA included 2\% agar.\\
\subsection*{Bioactivity on solid media and liquid media}
To assess the breadth of fungal species susceptible to \textit{Hp} sporangia, \textit{Hp} was individually co-cultured on mPmTG media with the variety of fungi described above. A block of agar containing active \textit{Hp} sporangia was added to mPmTG plate, flushed with sterile diH$_{2}$O, and incubated for 48h, at which point they were inoculated at a single point with 1x10$^{\circ}$ \textit{Neurospora} conidia in diH$_{2}$O. For \textit{Neurospora} experiments, plates were left to grow for an additional 24 - 48 hrs.\\
\indent Similarly, for other fungi (which are much slower growing than \textit{Neurospora}), \textit{Hp} and the target fungus were inoculated at the same time on an mPmTG solid media plate.\\
\indent To obtain aliquots of filter-sterilized\textit{Hp}-conditioned media ("filtrate"), initial preparations (in triplicate) were constructed by adding eight blocks of agar containing actively-growing \textit{Hp} sporangia, each approximately 1cm$^{2}$ to 10 ml of liquid mPmTG. To initially help determine if the compound was being constitutively produced or a response mechanism, a second preparation was made identically to the one just described, to which an additional 0.5 $\mu$l of a 100 $\mu$l  \textit{N. crassa} conidial suspension was added. To help control for potential nutrient depletion by \textit{N. crassa} in this second preparation, a third preparation contained eight mPmTG agar blocks with no \textit{Hp} sporangial growth in 10 ml mPmTG, and 0.5 $\mu$l of a 100 $\mu$l \textit{N. crassa} conidial suspension. These preparations were left to incubate at room temperature for 72hrs, at which point the total filtrate was obtained by passing each individual replicate through a 0.22 $\mu$m syringe filter into a 50 ml conical tube. All replicates for a given treatment were pooled after filtration. \\
\indent Large scale preparations were constructed in a similar manner by adding twenty blocks of agar containing \textit{Hp} sporangia to 50 ml of mPmTG media, and filtered using a 0.22 $\mu$m syringe filter.\\
\indent For temperature assays, the filtrate from each small scale preparation was subsequently divided into eight 2 ml aliquots (in duplicate) to test for bioactivity over three separate experiments. The control sample was left undisturbed and uninoculated throughout the experiment. Another sample was inoculated with 2x10$^{6}$ conidia but otherwise untreated. The filtrate was subjected to six different temperature treatments to test its thermostability: -80$^{\circ}$C (30 min), -20$^{\circ}$C (1 hr), 4$^{\circ}$C (1 hr), 28$^{\circ}$C (1 hr), 65$^{\circ}$C (1 hr), and 90$^{\circ}$C (30 min). Each experimental treatment was inoculated with 2x10$^{\circ}$6 conidia. These temperature treatments were allowed to return to room temperature prior to inoculation.\\
\indent To test the sensitivity of \textit{Bd} to \textit{Hp}, 3 ml of the filtrate from the large scale preparation was used as an incubation medium for one blcok of 1\% Tryptone agar containing actively growing \textit{Bd} sporangia. Similarly, 3 ml of mPmTG media was used as a positive control. These preparations were incubated for 96h at room temperature to allow for potential zoospore production and release. After 96h, the suspension was mixed briefly and 1 ml was added to mPmTG plates, while another 1 ml was added to 1\% Tryptone plates. The plates were incubated at room temperature for a maximum of 14d and photographed periodically.\\
\subsection*{Reassembly and annotation}
\textit{Hp} sporangia material was collected by scraping plates with sterile spatulas. Material was ground using bead-beating methods, and DNA obtained using extraction methods. A DNAseq Illumina library was prepared using the NEBNext Ultra DNA library kit and submitted to the University of California, Riverside Genomics Institute for Integrateve Genomic Biology (IIGB) core facility for MiSeq Illumina HT sequencing.\\
\subsection*{Secretome and small-metabolite screen}
A comparative analysis of putative small metabolite and secretome composition in chytrids was performed using antiSMASH \cite{Blin2013}, and a predictive workflow optimized for fungi described by \cite{Min2010}.\\
\subsection*{Temperature and pH growth assays}
To compare to the assays described in \cite{Piotrowski2004}, \textit{Hp} was grown at a range of temperatures: 18$^{\circ}$C, 23$^{\circ}$C, 25$^{\circ}$C, 28$^{\circ}$C, and 30$^{\circ}$C. Briefly, a 15 ml mPmTG inoculum was incubated at 23$^{\circ}$C for 1 week. 1 ml of culture was used to inoculate 30 ml of mPmTG in 50 ml conical tubes. \\
\indent Similarly, \textit{Hp} and \textit{Bd} were grown at a range of pHs by preparing mPmTG and 1\% Tryptone media at different pH levels: 4.0, 5.0, 6.0, 6.8, 8.0, and 9.0.\\
\section{Results}
\subsection*{The physiology of \textit{Hp} suggests that it is more tolerant of environmental stresses than \textit{Bd}}
To assess whether or not the observed growth suppression effect was the result of acidification of the local environment, \textit{Hp} was cultured on plates augmented with bromophenol blue or phenol red. After 96h growth, neither plates changed color, indicating that \textit{Hp} sporangia do not decrease the pH of mPmTG media below 6.8. However, phenol red plates which were additionally inoculated with \textit{N. crassa} did change from red to yellow in the areas around \textit{N. crassa} hyphae, suggesting that \textit{N. crassa} acidifies mPmTG media to below pH 6.8. This effect is not observed in bromophenol blue plates, suggesting that the final pH lies between 6.8 and 4.6.\\
\indent To get a sense of the environmental tolerances of \textit{Hp} compared to \textit{Bd}, these two species were individually cultured in liquid media (mPmTG and 1\% Tryptone, respectively) prepared at different pHs, ranging from 4.0 to 9.0. The average OD$_{450}$, a measure of cell density, of \textit{Hp} at low pH (4 and 5) was 0.037 and 0.068 respectively, which was higher than that for \textit{Bd} (0.001 and 0.004 respectively). This suggests that \textit{Hp} is more tolerant of acidic environments than \textit{Bd} (Figure~\ref{fig:ChInhib_HpBdpHTemp}A). Similarly, \textit{Hp} and \textit{Bd} were grown at temperatures ranging from 18$^{\circ}$C to 30$^{\circ}$C, and the OD$_{450}$ assessed during a 21-day period (Figure~\ref{fig:ChInhib_HpBdpHTemp}B). Taken together, these results suggest that \textit{Hp} has a higher tolerance for environmental changes than \textit{Bd}. This finding isn’t terribly surprising given that this isolate was obtained from an acidic lake \cite{Longcore2011}.\\
\subsection*{\textit{Hp} inhibition is unique among the chytrids and broadly active against other fungal species}
My initial observation was repeatable, and under controlled circumstances, I was able to demonstrate that \textit{Hp} sporangia are reliably and repeatedly capable of inhibition of \textit{N. crassa} vegetative growth on solid mPmTG media (Figure~\ref{fig:ChInhib_NcraChytrid}A). The related species, \textit{Bd} and \textit{Sp}, do not demonstrate any inhibitory activity (Figure~\ref{fig:ChInhib_NcraChytrid}B and C). Additional Chytridiomycota species (\textit{Operculomyces laminatus} JEL 223, \textit{Rhizoclosmatium hyalinus} JEL 800, and \textit{Obelidium mucronatum} JEL 802) do not inhibit hyphal growth of \textit{N. crassa} (Figure~\ref{fig:ChInhib_NcraChytrid}D-F), providing additional support that this behavior is unique to \textit{Hp}. \\
\indent This phenomenon is not the result of any thigmotropism-related response or object avoidance type behavior in \textit{N. crassa}, as hyphae of \textit{N. crassa} are not inhibited by an agar block lacking \textit{Hp} sporangia (Figure~\ref{fig:ChInhib_NcAvoidance}).\\
\indent To determine if the sensitivity to \textit{Hp} sporangia was unique to \textit{N. crassa}, we screened a panel of fungi from among the Ascomycota, Basidiomycota, and Zygomycota. Figure~\ref{fig:ChInhib_HpOtherFungi}A recapitulates our previous observations about \textit{N. crassa}, while Figure~\ref{fig:ChInhib_HpOtherFungi}B and C demonstrate, respectively, that vegetative hyphal growth of \textit{N. discreta} and \textit{N. tetrasperma} can also be suppressed by the presence of \textit{Hp} sporangia.
\indent Within the Ascomycota, but outside of the genus Neurospora, \textit{Trichoderma reesei} (Sordariomycetes; Hypocreales; Hypocreaceae), \textit{Aspergillus nidulans} (Eurotiomycetes; Eurotiales; Trichocomaceae), and \textit{Ashbya gossypii} (Saccharomycetes; Saccharomycetales; Saccharomycetaceae) are all completely sensitive to \textit{Hp} (Figure~\ref{fig:ChInhib_HpOtherFungi}D-F, respectively). \\
\indent Within the Basidiomycota, \textit{Coprinopsis cinerea} (Agaricomycetes; Agaricales; Psathyrellaceae) is completely sensitive to \textit{Hp} sporangia (Figure~\ref{fig:ChInhib_HpOtherFungi}G). Growth of members of the order Mucorales had mixed sensitivities. \textit{Phycomyces blakesleeanus} (Mucorales; Phycomycetaceae) was completely sensitive to \textit{Hp} sporangia, yet \textit{Rhizopus oryzae} (Mucorales; Mucoraceae) appeared to be insensitive (Figure~\ref{fig:ChInhib_HpOtherFungi}H \& I, respectively). Despite a possibly limited panel of ten fungi, the broadly observed pattern of sensitivity suggests that the responsible compound has a generalized mechanism of action.\\
\indent Additionally, the growth in \textit{Hp} conditioned media filtrate of two strains of yeast (\textit{Saccharomyces cerevisiae} MAU99 and AH109) and \textit{E. coli} DH5$\alpha$ suggested that this filtrate inhibited liquid growth of these organisms, and also provides evidence that the antimicrobial properties of \textit{Hp} extend to bacteria as well (Figure~\ref{fig:ChInhib_HpYeastEcoli}).\\
\subsection*{Liquid filtrate screening suggests a non-protein compound is responsible for bioactivity}
We were interested in obtaining a sample of bioactive liquid, on which we could perform chemical profiling to isolate a responsible compound. To test for bioactivity, \textit{Hp} cultured in liquid mPmTG media for a period of 72 hrs and subsequently filter-sterilized, producing “conditioned media” to which \textit{N. crassa} conidia was reintroduced. \\
\indent This conditioned media, obtained from \textit{Hp} and absent of all sporangia, retains the same inhibitory properties against \textit{N. crassa} observed with \textit{Hp} sporangia on solid agar plates (Figure~\ref{ChInhib_FilterTempAssay}A). Additionally, conditioned media derived from "Hp alone" and "Hp+Nc" preparations (Prep 1 and 2, Figure S3) were indistinguishable from one another in their effects on \textit{N. crassa} growth. \\
\indent The stability of the conditioned media was tested at seven different temperatures, ranging from -80$^{\circ}$C to 90$^{\circ}$C. The conditioned media retained bioactivity between -20$^{\circ}$C and 60$^{\circ}$C (Figure~\ref{fig:ChInhib_FilterTempAssay}D-E), but was ineffective after treatment of -80$^{\circ}$C and 90$^{\circ}$C (Figure~\ref{fig:ChInhib_FilterTempAssay}F-G). \textit{N. crassa} has no problem growing in fresh mPmTG media (Figure~\ref{fig:ChInhib_FilterTempAssay}H). Additionally, the conditioned media was ineffective if left sterile at room temperature for a period of 96h, and subsequently inoculated with \textit{N. crassa} conidia.\\
\indent After treatment of 100 mg/ml of Proteinase K, the conditioned media was no less effective at inhibition of N. crassa hyphal growth than untreated media (not shown or pull up figure / possibly take a week to repeat this).\\
\subsection*{Secretome + antiSMASH analyses, coupled with improved genome assembly and annotation predicts a few unique secondary metabolite gene clusters with potential relevance}
Secretome prediction made using workflow from \cite{Min2010} (Figure 8). \\
\indent Secondary metabolite clusters were predicted using antiSMASH \cite{Blin2013} (Figure 9).\\
\indent Several unique proteins were found to be related to terpene synthases, and these were unique to Hp when compared to other Chytridiomycota and Blastocladiomycota species.\\
\indent Make a tree from these enzymes.\\
\subsubsection*{Kinase mutant screen}
In order to address potential mechanisms of action for this behavior, we randomly assayed 2 N. crassa phosphotase and 17 kinase mutants from \cite{Park2011}. All were found to be inhibited by growing sporangia of H. polyrhiza (Table~\ref{tab:ChInhib_Kinase}).\\
\section{Discussion}
This study was an attempt to characterize a startling and unexpected observation: actively growing \textit{H. polyrhiza} sporangia inhibit the vegetative, hyphal growth of \textit{N. crassa}. The findings demonstrate that i) \textit{H. polyrhiza} is unique among surveyed chytrids in its ability to inhibit filamentous growth of \textit{N. crassa}, ii) the interaction is not limited to N. crassa but is broadly active against a number of Ascomycete, Basidiomycete, and Zygomycete species, and iii) in silico secondary metabolite analysis suggests a terpene synthase-related enzyme as a possible candidate based on observed unique expansion of genes.\\
\indent H. polyrhiza, an aquatic, non-parasitic member of the Chytridiomycota isolated from a lake in the northeastern United States, and N. crassa, a filamentous, multicellular, non-parasitic fungus primarily observed on decaying plant matter, are unlikely to come in specific contact with one another in the environment, so this response was presumed to be a non-specific interaction. The results from the broad fungal species screen support this presumption, and it appears that H. polyrhiza is active against a multitude of fungal species.\\
\indent Compared with other chytrid isolates we assayed, \textit{Hp} stands alone in its antifungal properties. Although the described basal lineages represent only 2\% of all described fungi, and the estimated diversity of these lineages is presumed high, this is the first time an interaction of this type has been described in any of these lineages. While it is possible that this behavior is not unique to this specific \textit{Hp} isolate, a more exhaustive assay of chytrid isolates is necessary to elaborate on this hypothesis.\\
\indent The observation that conditioned media derived from "Hp alone" and "Hp+Nc" preparations were indistinguishable from one another in their effects on \textit{N. crassa} growth suggests that the compound is being constitutively produced from H. polyrhiza and is not a response to a fungal competitor. Analysis of the transcriptome would be beneficial and would in theory easily identify any constitutively active genes in the sporangia life stage, from which this compound is expected to originate.\\
\indent The liquid compound profiling suggests that the compound is not protein-based. Filter-sterilized conditioned media was subjected to temperatures of -80$^{\circ}$C, -20$^{\circ}$C, 4$^{\circ}$C, 23$^{\circ}$C, 28$^{\circ}$C, 65$^{\circ}$C, and 90$^{\circ}$C, which are temperatures which would be expected during routine storage and/or shipping processes. With the exception of -80$^{\circ}$C and 90$^{\circ}$C, treatment at these temperatures did not negatively impact the growth suppression properties of the filtered conditioned media.\\
\indent We performed both in silico secretome and secondary metabolite screens using the previously published 454-based H. polyrhiza genome. The secondary metabolite screen in particular provided a rough idea of candidate enzymes which were expanded in \textit{H. polyrhiza}.\\
\indent Taken together, these data suggest that the compound is a constitutively-produced secondary metabolite compound with broadly specific activity. Its mechanism of action is unknown, as are its chemical structure and related biosynthetic pathway(s). Near term future work will necessarily focus on chemical profiling of bioactive spent media filtrate to generate a working hypothesis for the chemical nature of the product. A starting point for this research is provided in the form of in silico genomic and transcriptomic research. Finally, it is worth noting that the relative ease with which this discovery was made speaks to the necessity for further research into these basal lineages, of which intimate genomic and biochemical knowledge is lacking.
