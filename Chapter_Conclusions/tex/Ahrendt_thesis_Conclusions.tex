%%%%%%%%%%%%%%%%%%%%%%%%%%%%%%%%%%%%%%%%%%%%%%%%%%%%%%%%%%%%%%%%%%%%%%%
%% Document: Thesis for PhD at UC Riverside                          %%
%% Description: A comparative analysis of environment sensing in EDF %%
%% Author: Steven Ahrendt                                            %%
%%%%%%%%%%%%%%%%%%%%%%%%%%%%%%%%%%%%%%%%%%%%%%%%%%%%%%%%%%%%%%%%%%%%%%%
% CONCLUSIONS %
%%%%%%%%%%%%%%%
\chapter{Conclusions}
\section{Background}
The objective of this research was to enhance the current knowledge about early-diverging fungal groups. More and more reasearchers are turning towards these organisms as afocal points for study as they occupy a unique position between the fungal-animal common ancestor, and the more popular and accessible fungal research systems (ie Ascomycota and Basidiomycota model organisms). Additionally, genomic resources and methodologies are becoming increasingly more accessible, and so the ability to incorporate this data into existing studies is becomign more widespread.\\
\section{Aims}
The aims for this dissertation research were:\\
\begin{enumerate}
  \item Characterize the identification of an opsin-like protein
  \item Describe and identify a compopund responsible for antifungal behavior
  \item Produce and interpret a transcriptome for mosquito pathogen; lay groundwork for enhanced genomic resources
\end{enumerate}
