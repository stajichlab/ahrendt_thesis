%%%%%%%%%%%%%%%%%%%%%%%%%%%%%%%%%%%%%%%%%%%%%%%%%%%%%%%%%%%%%%%%%%%%%%%%%%%%%%%%%
%% Document: Thesis for PhD at UC Riverside                                    %%
%% Title: Investigating the evolution of environmental and biotic interactions %%
%%          in basal fungal lineages through comparative genomics              %%
%% Author: Steven Ahrendt                                                      %%
%%%%%%%%%%%%%%%%%%%%%%%%%%%%%%%%%%%%%%%%%%%%%%%%%%%%%%%%%%%%%%%%%%%%%%%%%%%%%%%%%
% CONCLUSIONS %
%%%%%%%%%%%%%%%
\chapter{Conclusions}
The objective of this dissertation research was to enhance the current knowledge about basal fungal groups. Genomic resources and methodologies are becoming increasingly more accessible, and so the ability to incorporate this data into existing studies is becoming more widespread. This objective was addressed with a particular focus on the mechanisms by which basal fungi interact with other organisms and with their environment. \\
%%%%%%%%%%%%%%%%%%%%%%%%%%%%
%% Inhibition conclusions %%
%%%%%%%%%%%%%%%%%%%%%%%%%%%%
\indent In Chapter~\ref{chap:HpInhibition} I explored the inhibitory properties of the chytridiomycte \textit{Homolaphlyctis polyrhiza}. My surprising and unexpected observation in the Stajich lab of the non-pathogenic Chytridiomycte isolate \textit{Homolaphlyctis polyrhiza} JEL142 inhibiting vegetative hyphal growth of a \textit{Neurospora crassa} contaminant prompted an investigation into its biological nature. My work for this investigation pursued three major questions: "Is this a unique property of \textit{Hp}?", "Is this a specific interaction with \textit{N. crassa}?", and "What is the chemical nature of the responsible compound?"\\
\indent By examining five other chytrid species in culture, I have evidence supporting the idea that this is a specific behavior for \textit{Hp}. No other chytrid surveyed displayed the appropriate inhibitory phenotype, including the most closely related chytrid to \textit{Hp}, the amphibian pathogen \textit{Batrachochytrium dendrobatidis}.\\
\indent Next, I expanded the potential targets of \textit{Hp} to include members of the Ascomycota, Basidiomycota, and Zygomycota. All of these targets were susceptible to \textit{Hp}, with the exception of \textit{Rhizopus oryzae}.\\
\indent Finally, I looked for the chemical nature for the mechanism of action by liquid assay and computational screening. I determined that the active compound is soluble and stable for at least 96h absent any \textit{Hp} sporangia. Additionally, liquid obtained from preparations featuring both \textit{Hp}+\textit{N. crassa} and \textit{Hp} alone were indistinguishably effective against reintroduction of \textit{N. crassa} conidia.\\
\indent Taken together, these data suggest that the compound is a constitutively-produced secondary metabolite compound with broadly specific activity. Its mechanism of action is unknown, as are its chemical structure and related biosynthetic pathway(s). Near term future work will necessarily focus on chemical profiling of bioactive spent media filtrate to generate a working hypothesis for the chemical nature of the product. A starting point for this research is provided in the form of \textit{in silico} genomic and transcriptomic research. Finally, it is worth noting that the relative ease with which this discovery was made speaks to the necessity for further research into these basal lineages, of which intimate genomic and biochemical knowledge is lacking.\\
%%%%%%%%%%%%%%%%%%%%%%%%%%%
%% Rhodopsin conclusions %%
%%%%%%%%%%%%%%%%%%%%%%%%%%%
\indent Chapter~\ref{chap:RhodStruct} and Chapter~\ref{chap:RhodAux} were investigating the structural mechanics and components of basal fungal rhodopsin photosensory pathways. Sunlight is one of the most obvious environmental sources of information, and the most easily studied. It should come as no surprise that photoreception exists in some form in all three domains of life, however varied in its implementation. Found in the metazoan lineages, perhaps the most well known photoreceptor proteins is the Type 2 rhodopsin, a photoreceptive GPCR proteins which accomplishes its signaling via trandsucin and cGMP phosphodiesterase signaling cascade. The research in Chapters~\ref{chap:RhodStruct} and \ref{chap:RhodAux} address four key questions: "How similar, structurally, are the opsin-like proteins from chytrids to each other and to other experimentally verified photoreceptors?", "How does this structural similarity (or dissimilarity) impact functional photoreception?", "What is the complement of downstream associated compenents (eg heterotrimeric G-protein subunits)?", and "Do the auxilary protein presence/absence correlate with known aspects of the evolution of photobiology in fungi?" \\
\indent Sequence homology predicts a number of seven-transmembrane domain proteins in various chytrid species, with particular similarity to the members of the Type 2 GPCR rhodopsin family. I used sequences from \textit{B. dendrobatidis}, \textit{S. punctatus}, and \textit{A. macrogynus} in homology modeling against known rhodopsin crystal structures to produce 3D structural models of these "chytriopsins". Quality metrics indicate that these computational models are reasonably sound predictions of the protein structures as they exist in the cell.\\
\indent Therefore they are good candidates for analysis in ligand docking and molecular dynamics simulations to infer biological function. The structural analysis suggests that the rhodopsin-like protein identified in \textit{S. punctatus} is most likely to be functional given the conservation of important structural features, most importantly the conserved lysine residue to which the chromophore ligand is bound. Automated non-covalent docking screens of retinal-like ligands against other chytropsin proteins from \textit{B. dendrobatidis} and \textit{A. macrogynus} suggest that even though these proteins lack the conserved lysine in a proper position, the binding pockets are spatially and chemically able to accomodate the 11-\textit{cis}-retinal chromophore.\\
\indent Covalent docking prediction of the \textit{Sp} protein using 11-\textit{cis}-retinal suggests that it is the most likely to be functionally active. This prediction was further refined with MD simulations using AMBER. The results were compared to the experimentally verified interaction between the \textit{Todares pacificus} rhodopsin and 11-\textit{cis}-retinal (PDB ID: 2Z73). \\
\indent With respect to the proteins associated with rhodopsin signaling in animals, namely the heterotrimeric G proteins and effectors like phosphodiesterase, sequence similarity searcher uncovered homologs in the basal fungal lineages. \\
%%%%%%%%%%%%%%%%%%%%%%
%% Clat conclusions %%
%%%%%%%%%%%%%%%%%%%%%%
\indent Chapter~\ref{chap:ClatTranscriptome} dealt with the transcriptome analysis of the entomopathogenic Blastocladiomycete \textit{Coelomomyces lativittatus}. Species in the basal fungal lineages occupy a diverse collection of environmental niches, including symbionts, pathogens, and saprotrophs. However, species in the genus \textit{Coelomomyces}, itself a member of the Blastocladiomycota, are the only known basal fungi which are pathogenic in arthropods. Specifically, their development requires oscillation between two hosts: mosquito larvae and microcrustaceans. This lifecycle has made them both attractive targets for research into non-pesticide-based mosquito control, yet also difficult systems in which to pursue this research. The transcriptome analysis presented in Chapter~\ref{ClatTranscriptome} was addressing two major points about entompathogenic chytrids, using \textit{Coelomomyces lativittatus} as a model: "How do aspects of the transcriptome regarding already known aspects of \textit{C. lativittatus} biology, specifically regarding $\beta$-carotene biosynthesis, environment sensing, and insect-association?", and "How does the protein complement of \textit{C. lativittatus} compare related Chytridiomycete and Blastocladiomycete species which are not insect-associated?"\\
\indent PFAM prediction suggests a number of candidate enzymes related to insect virulence, the most prominent being members of the C1 peptidase family. These have documented antihelmintic effects and cuticle degrading activity in nematodes. Furthermore, this family seems expanded in \textit{C. lativittatus} relative to other basal fungi, which are not insect-associated organisms.\\
\indent The transcriptome analysis demonstrated that \textit{C. lativittatus} likely has a complete complement of $\beta$-carotene processing enzymes, despite the apparent lack of one enzyme in the pathway: phytoene desaturase, the first enzyme in the three step cascade and responsible for converting phytoene to lycopene. Because the other two enzymes are accounted for, and because \textit{Coelomomyces} are experimentally verified producers of $\beta$-carotene, it is unlikely that there is a novel biosynthetic pathway at work and instead the phyotene desaturase mRNA was not expressed at a high enough level to be recovered in either RNA extraction or during the HT sequencing.\\ 
\indent The descriptive results from this investigation serve as starting points for future work involving genomic, transcriptomic, and proteomic analyses from multiple developmental stages and timepoints.\\
\indent The research presented in this dissertation will inform future work on chytrids, and expand the knowledgebase for basal fungal lineages.\\
