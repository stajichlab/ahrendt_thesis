%%%%%%%%%%%%%%%%%%%%%%%%%%%%%%%%%%%%%%%%%%%%%%%%%%%%%%%%%%%%%%%%%%%%%%%%%%%%%%%%%
%% Document: Thesis for PhD at UC Riverside                                    %%
%% Title: Investigating the evolution of environmental and biotic interactions %%
%%          in basal fungal lineages through comparative genomics              %%
%% Author: Steven Ahrendt                                                      %%
%%%%%%%%%%%%%%%%%%%%%%%%%%%%%%%%%%%%%%%%%%%%%%%%%%%%%%%%%%%%%%%%%%%%%%%%%%%%%%%%%
% CONCLUSIONS %
%%%%%%%%%%%%%%%
\chapter{Conclusions}
The objective of this dissertation research was to enhance the current knowledge about basal fungal groups. Genomic resources and methodologies are becoming increasingly more accessible, and so the ability to incorporate this data into existing studies is becoming more widespread.
\indent This objective was addressed through the following specific aims:\\
\begin{enumerate}
  \item Describe and identify a compopund responsible for antifungal behavior (Chapter~\ref{chap:HpInhibition})
  \item Characterize the identification of an opsin-like protein, through both structural analyses (Chapter~\ref{chap:RhodStruct}) as well as accessory and downstream proteins (Chapter~\ref{chap:RhodAux})
  \item Produce and interpret a transcriptome for mosquito pathogen, thereby laying groundwork for enhanced genomic resources for this organism (Chapter~\ref{chap:ClatTranscriptome})
\end{enumerate}
\section{Inhibition}
Attributed to Louis Pasteur in 1854: "In the fields of observation, chance favors only the prepared mind." The observation in the Stajich lab of the non-pathogenic Chytridiomycte isolate \textit{Homolaphlyctis polyrhiza} JEL142 inhibiting vegetative hyphal growth of a \textit{Neurospora crassa} contaminant prompted an investigation into its biological nature. My work for this investigation pursued three major questions:\\
\begin{enumerate}
  \item Is this a unique property of \textit{Hp}?
  \item Is this a specific interaction with \textit{N. crassa}?
  \item What is the chemical nature of the responsible compound?
\end{enumerate}
\indent By examining five other chytrid species in culture, I have evidence supporting the idea that this is a specific behavior for \textit{Hp}. No other chytrid surveyed displayed the appropriate inhibitory phenotype, including the most closely related chytrid to \textit{Hp} the amphibian pathogen \textit{Batrachochytrium dendrobatidis}.\\
\indent Taken together, these data suggest that the compound is a constitutively-produced secondary metabolite compound with broadly specific activity. Its mechanism of action is unknown, as are its chemical structure and related biosynthetic pathway(s). Near term future work will necessarily focus on chemical profiling of bioactive spent media filtrate to generate a working hypothesis for the chemical nature of the product. A starting point for this research is provided in the form of \textit{in silico} genomic and transcriptomic research. Finally, it is worth noting that the relative ease with which this discovery was made speaks to the necessity for further research into these basal lineages, of which intimate genomic and biochemical knowledge is lacking.\\
\section{Rhodopsin}
Sunlight is one of the most obvious environmental sources of information, and the most easily studied. It should come as no surprise that photoreception exists in some form in all three domains of life, however varied in its implementation. The research in Chapters~\ref{chap:RhodStruct} and \ref{chap:RhodAux} address four key questions:\\
\begin{enumerate}
  \item How similar, structurally, are the opsin-like proteins from chytrids to each other and to other experimentally verified photoreceptors?
  \item How does this structural similarity (or dissimilarity) impact functional photoreception?
  \item What is the complement of downstream associated compenents (eg heterotrimeric G-protein subunits)?
  \item Do the auxilary protein presence/absence correlate with known aspects of the evolution of photobiology in fungi?
\end{enumerate}
\section{\textit{Coelomomyces}}
Species in the basal fungal lineages occupy a diverse collection of environmental niches, including symbionts, pathogens, and saprotrophs. However, species in the genus \textit{Coelomomyces}, itself a member of the Blastocladiomycota, are the only known basal fungi which are pathogenic in arthropods. Specifically, their development requires oscillation between two hosts: mosquito larvae and microcrustaceans. This lifecycle has made them both attractive targets for research into non-pesticide-based mosquito control, yet also difficult systems in which to pursue this research. The transcriptome analysis presented in Chapter~\ref{ClatTranscriptome} was addressing two major points about entompathogenic chytrids, using \textit{Coelomomyces lativittatus} as a model:\\
\begin{enumerate}
  \item What is possible to infer from the transcriptome regarding already known aspects of \textit{C. lativittatus} biology, specifically regarding $\beta$-carotene biosynthesis, environment sensing, and insect-association?
  \item How does the protein complement of \textit{C. lativittatus} compare related Chytridiomycete and Blastocladiomycete species which are not insect-associated?
\end{enumerate}
\indent The descriptive results from this investigation serve as starting points for future work involving genomic, transcriptomic, and proteomic analyses from multiple developmental stages and timepoints.\\
