%%%%%%%%%%%%%%%%%%%%%%%%%%%%%%%%%%%%%%%%%%%%%%%%%%%%%%%%%%%%%%%%%%%%%%%
%% Document: Thesis for PhD at UC Riverside                          %%
%% Description: A comparative analysis of environment sensing in EDF %%
%% Author: Steven Ahrendt                                            %%
%%%%%%%%%%%%%%%%%%%%%%%%%%%%%%%%%%%%%%%%%%%%%%%%%%%%%%%%%%%%%%%%%%%%%%%
% ABSTRACT %
%%%%%%%%%%%%
\begin{abstract}
Species belonging to the early diverging fungal lineages (Blastocladiomycota, Chytridiomycota, Cryptomycota, and Neocallimastigomycota) reproduce via motile uniflagellated zoospores. These organisms are traditionally understudied, despite being active decomposers, parasites, and symbionts with other organisms in the ecosystem. This dissertation research uses a comparative approach to answer questions about how these organisms interact with their environment, regarding putative photoreception (Chapters~\ref{chap:RhodStruct} and \ref{chap:RhodAux}), molecular aspects of the evolutionary transition from aquatic motile single cells to terrestrial multicellullar organisms (Appendix~\ref{app:Flagella}), potential anti-fungal and competitive behavior and its mechanisms (Chapter~\ref{chap:Hp_inhibition}).
\end{abstract}
