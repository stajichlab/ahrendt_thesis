%%%%%%%%%%%%%%%%%%%%%%%%%%%%%%%%%%%%%%%%%%%%%%%%%%%%%%%%%%%%%%%%%%%%%%%%%%%%%%%%%
%% Document: Thesis for PhD at UC Riverside                                    %%
%% Title: Investigating the evolution of environmental and biotic interactions %%
%%          in basal fungal lineages through comparative genomics              %%
%% Author: Steven Ahrendt                                                      %%
%%%%%%%%%%%%%%%%%%%%%%%%%%%%%%%%%%%%%%%%%%%%%%%%%%%%%%%%%%%%%%%%%%%%%%%%%%%%%%%%%
% ABSTRACT %
%%%%%%%%%%%%
\begin{abstract}
Species belonging to the early diverging fungal lineages (Blastocladiomycota, Chytridiomycota, Cryptomycota, and Neocallimastigomycota) reproduce via motile uniflagellated zoospores. These organisms are traditionally understudied, despite being active decomposers, parasites, and symbionts with other organisms in the ecosystem. This dissertation research uses a comparative approach to answer questions about how these organisms interact with their environment, regarding potential antifungal and competitive behavior and its mechanisms (Chapter~\ref{chap:HpInhibition}), putative photoreception (Chapters~\ref{chap:RhodStruct} and \ref{chap:RhodAux}), transcriptome analysis of one member of the only known insect associate genus (Chapter~\ref{chap:ClatTranscriptome}), and molecular aspects of the evolutionary transition from aquatic motile single cells to terrestrial multicellullar organisms (Appendix~\ref{app:Flagella}).
\end{abstract}
