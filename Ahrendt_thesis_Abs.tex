%%%%%%%%%%%%%%%%%%%%%%%%%%%%%%%%%%%%%%%%%%%%%%%%%%%%%%%%%%%%%%%%%%%%%%%%%%%%%%%%%
%% Document: Thesis for PhD at UC Riverside                                    %%
%% Title: Investigating the evolution of environmental and biotic interactions %%
%%          in basal fungal lineages through comparative genomics              %%
%% Author: Steven Ahrendt                                                      %%
%%%%%%%%%%%%%%%%%%%%%%%%%%%%%%%%%%%%%%%%%%%%%%%%%%%%%%%%%%%%%%%%%%%%%%%%%%%%%%%%%
% ABSTRACT %
%%%%%%%%%%%%
\begin{abstract}
Species belonging to the early diverging fungal lineages (Blastocladiomycota, Chytridiomycota, Cryptomycota, and Neocallimastigomycota) reproduce via motile zoospores and are found in both aquatic and terrestrial environments. These organisms are traditionally understudied, despite being active decomposers, parasites, and symbionts with other organisms in the ecosystem. This dissertation research uses a comparative genomics approach to answer questions about these fungi and their interactions with their environment and other fungi. Chapter~\ref{chap:HpInhibition} describes surprising observations regarding competitive and inhibitory behavior in one member of the Chytridiomycota. Chapters~\ref{chap:RhodStruct} and \ref{chap:RhodAux} examine the feasibility of rhodopsin-mediate photoreception in basal lineages using structural mechanics and genome-wide gain-loss analyses. Chapter~\ref{chap:ClatTranscriptome} provides a transcriptome analysis of one member of the genus \textit{Coelomomyces}, the only known entomopathic chytrid genus. Appendix~\ref{app:Flagella} briefly looks at gain-loss analysis of molecular aspects of the evolutionary transition from aquatic motile single cells to terrestrial multicellullar organisms.\\
\end{abstract}
