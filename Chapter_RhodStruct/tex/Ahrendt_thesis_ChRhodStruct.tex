%%%%%%%%%%%%%%%%%%%%%%%%%%%%%%%%%%%%%%%%%%%%%%%%%%%%%%%%%%%%%%%%%%%%%%%%%%%%%%%%%
%% Document: Thesis for PhD at UC Riverside                                    %%
%% Title: Investigating the evolution of environmental and biotic interactions %%
%%          in basal fungal lineages through comparative genomics              %%
%% Author: Steven Ahrendt                                                      %%
%%%%%%%%%%%%%%%%%%%%%%%%%%%%%%%%%%%%%%%%%%%%%%%%%%%%%%%%%%%%%%%%%%%%%%%%%%%%%%%%%
%% RHODOPSIN: STRUCTURAL CHAPTER %%
%%%%%%%%%%%%%%%%%%%%%%%%%%%%%%%%%%%
\chapter{Structural characteristics of opsin-like proteins found in basal fungal lineages}
\label{chap:RhodStruct}
%%%%%%%%%%%%%%%%%%
%% Introduction %%
%%%%%%%%%%%%%%%%%%
\section{Introduction}
During the course of a given day, an organism experiences a multitude of environmental stimuli, including chemicals, gravity, the Earth's magnetic field, pressure, and light. The biochemical ability to appropriately process and respond to these signals is an incredibly complex and involved task, and understanding the underlying mechanisms of these responses is an ongoing scientific challenge. \\
\indent The presence or absence of light is perhaps one of the easiest sources of stimuli to comprehend and observe. The sun and rotation of the planet has had such a profound influence on the development of life that it comes as no surprise to find some form of photoreception in every major lineage on the planet. The widespread occurrence of such an ability, however varied in its implementation, speaks to its importance during the earliest stages of development of life. \\
\indent In Fungi, there are several classes of proteins capable of photoreception, all of which have different mechanisms of action and specializations. These include phytochrome, cryptochrome, white-collar, and opsin \cite{Idnurm2010}. In plants, phytochromes function as day-night sensors to regulate the circadian rhythm and flowering response. This is accomplished through a conformational shift between the red and far-red sensitive forms of the protein structure \cite{Rockwell2006}. While relatively little is known about fungal phytochrome function, research on \textit{A. nidulans} suggests that the phytochrome protein is a member of an elaborate complex with regulatory functions involved with the asexual-sexual transition and secondary metabolite biosynthesis \cite{Idnurm2010}. The white-collar complex (WCC), on the other hand, is very well characterized in Fungi. First studied in the model filamentous Ascomycete fungus \textit{Neurospora crassa}, WCC functions as a heterodimer comprising White-collar 1 and 2 proteins to sense blue and near UV wavelengths, and, when active, directly interacts with DNA to regulate the circadian clock machinery, sporulation, pigmentation, and phototropism \cite{Ballario1997,Purschwitz2006,Corrochano2007}. Cryptochromes are photoreceptors which belong to a large group of flavoproteins. Initial observations of blue-light sensitive photoreception in plants, without concurrent description of the responsible photoreceptor protein, led to the name "cryptochrome": because of their "cryptic" nature \cite{Cashmore1999}. These proteins can be found in plants, animals, and insects, and are involved with circadian rhythm regulation and light activated DNA damage repair \cite{Idnurm2010}.\\
\indent Rhodopsin is a broadly defined term used to describe a large class of seven-transmembrane proteins which use retinylidene compounds for photoreception. This class can be subdivided into two types based on sequence similarity and function, despite similarities in structure (ie seven helical transmembrane domains) and mechanism of activation (ie photoisomerization of a retinaldehyde chromophore) \cite{Spudich2000}.\\
\indent The ion transporter rhodopsins ("Type 1") are activated by the photoisomerization of all-\textit{trans}-retinal to 13-\textit{cis}-retinal. These function as membrane channels and are typically used for light-driven membrane depolarization via proton or chloride ion pumping. Examples of this group can be found in bacteria, archaea, and eukaryotes, and include the bacterial sensory rhodopsins, channelrhodopsins, bacteriorhodopsins, halorhodopsins, and proteorhodopsins.\\
\indent The G-protein coupled receptor (GPCR) rhodopsins ("Type 2") are activated by the photoisomerization of 11-\textit{cis}-retinal to all-\textit{trans}-retinal. These function as visual receptors, and are the largest class of an even larger GPCR superfamily found only in eukaryotes. The general class of photosensitive GPCRs in animals are often referred to as "opsins", with the visual opsins being one of the distinct subfamilies and known as "rhodopsins". In animals, the other subfamiles are the melanopsins, peropsins, neuropsins, and encephalopsins. The exact nature of the evolutionary relationship between the Type 1 and Type 2 rhodopsins has not been clearly established and is currently the subject of discussion \cite{Terakita2005,Shichida2009}.\\
\indent Nonetheless, the rhodopsin pigment of both types is generated when the retinaldehyde chromophore is covalently joined to an opsin apoprotein via a Schiff-base linkage to a conserved lysine residue. While 11-\textit{cis}-retinal is the most common chromophore observed in vertebrates and invertebrates, others are found elsewhere in nature. For example, 3,4-dehydroretinal is observed in fish, amphibia, and reptiles. Switching between the 11-\textit{cis} and 3,4-dehydro- chromophores can be employed as a light adaptation strategy in certain freshwater fish \cite{Shichida2009}. 3-hydroxyretinal is found in insects, while 4-hydroxyretinal is observed in the firefly squid. In addition to the 11-\textit{cis} conformation, retinal can adopt a number of different isomers, including all-\textit{trans}, 13-\textit{cis}, and 9-\textit{cis} \cite{Shichida2009}. Molecular mechanics simulations suggest that the 11-\textit{cis}-retinal isomer has been selected for evolutionarily as the optimal chromophore due to the energetic stability of the resulting chromophore-opsin construct \cite{Sekharan2011}.\\ 
\indent Previous work has demonstrated that certain basal fungal species are phototaxic. For example, the phototactic capabilities of the marine fungus \textit{Rhizophydium littoreum} were quantified in 1987 by Muehlstein et al. \nocite{Muehlstein1987} This fungus demonstrated responses to light at a variety of wavelenths, with the most rapid response occuring at 400 nm. While the evidence strongly suggests a blue-light sensitive photoreceptor, the researchers were unable to specifically characterize the active photoreceptor.\\
\indent A decade later, Saranak and Foster described their work on the phototactic capabilities of the Blastocladiomycete \textit{Allomyces reticulatus} \cite{Saranak1997}. This fungus has a visible, red-pigmented eyespot in which the photosensitive proteins are localized. Careful analysis determined that action spectrum of the phototactic zoospores peaks at 536$\pm$ 4 nm, simlar to that of the human green-sensitive cone. Furthermore, the researchers were able to destroy and subsequently restore the phototactic phenotype by reversibly inhibiting the biosynthesis of $\beta$-carotene, the molecular precursor to retinal. Taken together, these results suggested the presence of a rhodopsin protein of the Type 2 subfamily.\\
\indent The increasing availability of genomes from the traditionally understudied basal fungal lineages, coupled with a fairly well understood and important environmental sensing system, yields an opportunity to expand on known information about the photosensory response in fungi. In this chapter, I describe a computational approach toward understanding the structural mechanisms involved with rhodopsin-specific photoreception in basal fungi.\\

%%%%%%%%%%%%%
%% Methods %%
%%%%%%%%%%%%%
\section{Methods}

%---- Homology modeling and docking ----%
\subsection*{Sequence identification and homology modeling}
Putative rhodopsin sequences in basal fungal lineages were identified based on sequence similarity to the Profile Hidden Markov model from the Pfam database \cite{Finn2014}, accession PF00001 ("7tm\_1"). Identified fungal sequences were aligned with a subset of animal rhodopsin sequences from GenBank using Expresso \cite{Armougom2006}, a modified version of T-Coffee \cite{Notredame2000}, which incorporates protein structural information to guide the sequence alignment. The MAFFT \cite{Katoh2002,Katoh2005} and Muscle \cite{Edgar2004} multiple alignment modules were added to the default Expresso alignment. Based on the multiple sequence alignment, the structure model of the \textit{B. dendrobatidis} protein was built using Modeller (v9.9) \cite{Eswar2007} with explicit loop refinement and refined with OPUSRota (v1.0) \cite{Lu2008}. The \textit{S. punctatus} and \textit{A. macrogynus} models were constructed using iTASSER against the provided GPCR specific library \cite{Zhang2008}. The \textit{Bd} sequence was initially modeled using iTASSER along with the \textit{Sp} and \textit{Am} sequences. However, the Modeller-produced model was selected for further analyses as a conserved structural motif in EL2 of the iTASSER best-scoring model for \textit{Bd} was modeled incorrectly when compared to crystal structures 2Z73 and 1U19, and the \textit{Sp} and \textit{Am} iTASSER models.\\ 
\indent For the structures generated using Modeller, the output consisted of five potential models and corresponding Discrete Optimized Protein Energy ("DOPE") \cite{Shen2006} and MODELLER objective function ("molpdf") scores. "Optimal" models were therefore selected which had the lowest DOPE and molpdf values. For the iTASSER structures, the optimal model was selected using the iTASSER provided "c-score", a confidence value based on the significance of threading template alignments.\\
\indent The quality of these selected models was assessesd using PROCHECK (v3.5) \cite{Laskowski1993,Wiederstein2007} and Verify3D \cite{Luthy1992}. The melatonin model was constructed using the human melatonin sequence (UniProt ID: P48039) and subjected to homology modeling with Modeller (v9.9) \cite{Eswar2007} using the \textit{T. pacificus} rhodopsin crystal structure (2Z73) as a template. The Modeller-generated homology model was of better stereochemical quality than the iTasser generated Melatonin model using the GPCR database. As such, the highest quality Modeller-generated model was selected for further side chain refinements with OPUSRota \cite{Lu2008}, similar to the \textit{Bd} model generation.\\
\subsection*{Docking}
\indent Automated protein-ligand docking was accomplished using Autodock 4 \cite{Morris2009} which implements a Lamarckian genetic algorithm approach for calculating the minimum free energy of binding of small molecules. Small molecule files were obtained from PubChem \cite{Bolton2008} for the following isomers of retinal: 11-\textit{cis} (A1), all-\textit{trans}, 9-\textit{cis}, 13-\textit{cis}, 3,4-dehydro- (A2), 3-hydroxy- (A3), and 4-hydroxy- (A4) used in the covalent docking screen. A covalent linkage was formed by manually specifying the presence of a bond between the terminal carbon atom in retinal and terminal nitrogen atom in the lysine side chain. The specific lysine predicted to be involved in functional photoreception was inferred through multiple sequence alignment.\\
\indent For the non-covalent docking screen, the 11-\textit{cis}-retinal (A1) isomer was used as a search query in ZINC (v 12) \cite{Irwin2005} with a cutoff value of 0.9. 83 compounds were retrieved and used in addition to 11-\textit{cis}-retinal in Autodock 4.\\
\indent Structure figures were created using the PyMOL Molecular Graphics System, Version 1.7.4 \cite{PyMOL}. Membrane topology figure panels were drawn using the {\TeX}topo package \cite{Beitz2000textopo}.\\
\subsection*{RMSD calculation}
The loop regions in all models were removed such that the models contained only the seven transmembrane helix regions. The helix-only structures were then aligned using the STAMP Structural alignment method \cite{Russell1992} of VMD (v1.9.1) \cite{Humphrey1996}, and the RMSD values of the backbones of the aligned helix-only structures were computed using the $rmsd()$ function in the Bio3D R package \cite{Grant2006}.\\

%---- Molecular dynamics simulations ----%
\subsection*{Molecular Dynamics}
Molecular dynamics simulations were performed using the Amber14 suite of programs \cite{AMBER2015}. For MD simulations of the squid structure, PDBID 2Z73 was used along with the structure of 11-\textit{cis}-retinal crystalized with it. For the \textit{S. punctatus} structure, simulations were performed using 9-\textit{cis}-retinal ligand in the lowest energy conformation. 9-\textit{cis}-retinal was chosen based on the covalent docking screen results in Table~\ref{tab:ChRhodS_covDock}. Initial minimization for 1ns, followed by three equilibration steps for 50ps progressing from 200K to 250K to 298K. The final production simulation was run for 10ns at 298K. Due to the computational expense of an explicit solvation model for simulating water molecules, an implicit solvation model \cite{Onufriev2000} (modified from the generalized Born solvation model \cite{Bashford2000}) was implemented in AMBER by the $igb=2$ flag. Backbone atoms were kept rigid while binding pocket residues (as identified in Table~\ref{tab:ChRhodS_residues}) were made flexible. Trajectory visualization was accomplished using VMD (v1.9.1) \cite{Humphrey1996}. RMSD values and potential energy of the system were summarized using $cpptraj$ and $process\_mdout.perl$ script, respectively, provided with the AMBER package.\\

%%%%%%%%%%%%%
%% Results %%
%%%%%%%%%%%%%
\section{Results}

%---- Structure quality ----%
\subsection*{Homology models are of reasonable quality}
Ramachandran plots were generated for all structure models using PROCHECK \cite{Laskowski1993,Wiederstein2007}. These plots graphically display the backbone dihedral angles $\psi$ and $\phi$ of each amino acid residue in a protein and are indicative of model quality, and are summarized in Table~\ref{tab:ChRhodS_Procheck}. For \textit{B. dendrobatidis}, \textit{S. punctatus}, and \textit{A. macrogynus}, the percentage of model residues after refinement which fell within the most favorable regions was 86.1\%, 84.2\%, and 66.4\%, respectively. For comparison, the \textit{T. pacificus} (2Z73) and \textit{B. taurus} (1U19) published crystal structures have scores of 90.9\% and 79.9\%, respectively. A model with a score of \textgreater 90\% in this category is considered to be of good quality.\\
\indent 3D profile scores, computed using Verify3D \cite{Luthy1992}, are provided in Table~\ref{tab:ChRhodS_Procheck}. The \textit{B. dendrobatidis} model has a score of 55.41, and the models for \textit{S. punctatus} and \textit{A. macrogynus} have scores of 73.60 and 131.62, respectively. For comparison, the 3D profile scores for \textit{T. pacificus} (2Z73) and \textit{B. taurus} (1U19) published crystal structures are 87.85 and 109.14, respectively.
\subsection*{Structural conservation reveals \textit{S. punctatus} structure to be most likely functional as photoreceptor}
\indent After generating models for the chytropsin sequences, the best-scoring models were selected, representing the most computationally and chemically ideal configurations. A number of structural features provide support for the relationship between chytriopsins and other members of the opsin family. Based on sequence similarity, the chytropsin sequences are expected to have seven transmembrane architecture. Table~\ref{tab:ChRhodS_RMSD} displays the pairwise backbone RMSD calculations which display high degree of agreement with rhodopsin crystal structures. Coupled with the overall structure alignment presented in Figure~\ref{fig:ChRhodS_SpStructure}A, this expected architecture is confirmed. \\
\indent These values are from a Ramachandran plot \cite{Ramachandran1963} generated using PROCHECK \cite{Laskowski1993}, a method for checking the stereochemical quality (both overall and residue-by-residue geometry) of a protein structure. The results represent the percentage of residues which, based on their $\phi$ and $\psi$ angles, fall within specific stereochemical regions as defined by analysis of experimentally solved structures. A good quality model would be expected to have over 90\% in the "most favored" regions. Since our models have reasonably high percentages in the "most favored" regions, and reasonably low percentages in the "disallowed" regions, this table suggests that our chytriopsin and melatonin homology models are of reasonable quality.\\
\indent Generally speaking, the root-mean-square deviation (RMSD) is a measure of the difference between values predicted by a model and those actually observed. In the context of protein structure prediction, the RMSD value is a measure of the average distance between backbone atoms of superimposed protein structures. The RMSD measurement can be used as a quantitative comparison between two aligned structures, and similar structures will have lower RMSD values. \\
\indent In our case, these values describe the pairwise similarity for our chytriopsin and melatonin homology models and the experimentally-verified animal rhodopsin crystal structures. Since low RMSD values correspond to similar structures, and since the RMSD values for Melatonin against solved structures are much higher than those for our chytriopsin models, this table suggests that our chytriopsin models are more structurally similar to animal rhodopsins than to melatonin receptors.\\
\indent There is a defined sequence of events in the activation mechanism initiated by photoisomerization of 11-\textit{cis}-retinal: protonation of Glu113 (typically by proton transfer mediated by protonated Schiff-base and Lys296), outward rotation of H6 (breaking the ion lock), and protonation of Glu134 (re-stabalization in active state).\\
\indent \underline{Binding pocket, lysine, and counterion}- The photoisomerization process involves light interacting with a retinal chromophore producing a conformation change and proton transfer cascade \cite{Birge1990,Smith2010}. The most critical residues (\textit{B. taurus} numbering) in this cascade are Lys296, responsible for formation of the protonated Schiff-base covalent linkage to 11-\textit{cis}-retinal, Glu113, the counterion responsible for proton transfer during photoisomerization, and the H-bond network required for dark-state stability, centered around His211 and Glu122, including Glu181, Tyr192, Tyr268, Ser186, Glu113, Cys187, and Thr94. \\
\indent The \textit{S. punctatus} structure posesses both the conserved lysine (K320) and a suitable counterion (D94) in positions favorable for proper function. The structures of \textit{B. dendrobatidis} and \textit{A. macrogynus}, on the other hand, lack the conserved lysine and counterion residues in analagous positions. Binding pocket residues, and lysine, counterion, and H-bond network residues are compared in Figure~\ref{fig:ChRhodS_SpStructure}C.\\
\indent \underline{Ion lock}- The (E/D)RY and NPXXY motifs function together as the "ionic lock": a structural motif responsible for stabilizing the protein in the inactive conformation and which is broken upon receptor activation \cite{Smith2010}. The (E/D)RY motif of \textit{B. taurus} consists of the Glu134-Arg135-Tyr136 residues. A salt bridge between Arg135 on H3 and Glu247 on H6 stabilizes the lock in this inactive state. Upon receptor activation, the NPXXY motif, specifically Tyr306 rotates toward Arg135 to break the lock. The ERY motif in the \textit{S. punctatus} structure comprises Glu115-Arg116-Tyr117, and the NPXXY motif is functionally conserved with Asn326-Pro327-Val328-Leu329-Phe330. The \textit{B. dendrobatidis} motifs are slightly less conserved with Asn104-His105-Tyr106 for ERY, and Asn353-Pro354-Ile355-Val356-Phe357 for NPXXY. The \textit{A. macrogynus} motifs are much more conserved: Glu155-Arg156-Tyr157 for ERY, and Asn504-Pro505-Leu506-Leu507-Ser508 for the NPXXY motif. Comparisons are displayed in Figures~\ref{fig:ChRhodS_SpStructure}E. \\
\indent \underline{Salt bridge / disulfide bond}- In Bovine rhodopsin, the extracellular loop region (EL2) between Trp175 on H4 and Thr198 on H5 contains two linkages that are critical for correct rhodopsin folding: the conserved disulfide bond between residues Cys110 and Cys187 and a conserved salt bridge between Arg177 and Asp190 \cite{Smith2010}. The residues that correspond to the disulfide bond are conserved in the three chytrid structures: Cys80-Cys155 in \textit{B. dendrobatidis}, Cys91-Cys166 in \textit{S. punctatus}, and Cys131-Cys220 in \textit{A. macrogynus}. The salt bridge residues are relatively conserved in \textit{B. dendrobatidis}, with Lys145 and Asp158. However they are somewhat less conserved in \textit{S. punctatus} (Ala156 and Asp169) and \textit{A. macrogynus} (Thr203 and Ala223). Comparisons are displayed in Figure~\ref{fig:ChRhodS_SpStructure}D.\\

%---- Docking and MD ----%
\subsection*{\textit{in silico} chemical screen}
Computational protein-ligand docking was accomplished using Autodock 4 with 11-\textit{cis}-retinal, all-\textit{trans}-retinal, 9-\textit{cis}-retinal, 13-\textit{cis}-retinal, 3-dihydroretinal, and 4-dihydroretinal (Table~\ref{tab:ChRhodS_covDock}). When docked against the squid crystal structure (PDB ID 2Z73), 11-\textit{cis}-retinal had the lowest free energy of binding. This was to be expected as 11-\textit{cis}-retinal is the functional chromophore for the squid rhodopsin protein. Additionally, all-\textit{trans}-retinal had the highest free energy of binding.\\
\indent The lowest energy conformation for the \textit{S. punctatus} modeled structure were observed when bound to 9-\textit{cis}-retinal isomer, with the next lowest conformation observed with the 11-\textit{cis}-retinal isomer.\\
\subsection*{Molecular Dynamics simulations}
In order to assess how the stability of the predicted \textit{S. punctatus}+9-\textit{cis}-retinal complex compares to that of the canonical squid+11-\textit{cis}-retinal complex, I performed molecular dynamics simulations using AMBER 14. An overview of the potential energy of two systems during the 10ns simulation is given in Figure~\ref{fig:ChRhodS_MDEnergyRMSD}A. While the potential energy of the \textit{S. punctatus} complex is much lower than that of the squid, both complexes are extremely stable over the long term. The average structure was generated using $cpptraj$ by RMS fitting backbone atom coordinates from 2000 snapshots at 5ps intervals and averaging the coordinates. For both complexes, these results are given in Figure~\ref{fig:ChRhodS_MDEnergyRMSD}B. The squid complex achieves equilibrium starting from 1 ns of the trajectory period, and the deviation from the starting structure is about 3\AA. Similarly, the \textit{Sp} complex achieves equilibrium starting from 3 ns, while the deviation from the starting structure is close to 8\AA. \\
\indent In the unbound state, the lysine residue (Lys296) of the squid complex is much more flexible than that of the \textit{Sp} complex (Lys320). The residues within 5\AA of the 11-\textit{cis}-retinal ligand 

%%%%%%%%%%%%%%%%
%% Discussion %%
%%%%%%%%%%%%%%%%
\section{Discussion}
The opsin class of visual receptors can be divided into two subtypes, based on sequence similarity and function. While both types have similar tertiary structure (eg seven transmembrane helices), the Type 2 rhodopsins, which act as GPCR proteins, have thus far only been identified in metazoan lineages, while the Type 1 rhodopsins, which function as ion channels, are typically found in bacteria and archaea. The rhodopsin-like proteins identified in recently sequenced, early-diverging flagellated fungi are most similar to these Type 2 proteins, and thus are in an excellent position to add to the expanding knowledge base of the evolution of vision.\\
\indent The work described in this chapter seeks to address questions related to the structure and function of these identified chytrid rhodopsins, namely I) these proteins are structurally similar to visual rhodopsins, and II) functional characteristics can be determined through \textit{in silico} chemical ligand and molecular dynamics simulations.\\
\indent In support of this hypothesis are the results of a number of comparative analyses. Across the fungi, there are different types of photosensitive proteins, each with different structures and regulatory mechanisms.\\
\indent The rhodopsin-like proteins identified in the chytrid lineages have notable similarities and differences relative to well described rhodopsins. The expected seven transmembrane structure is conserved in every sequence identified, with proper orientation of N- and C-termini. The $\beta$-sheet motif at the top of the structure is conserved, as is the cysteine bridge and ion lock motifs important for structural stability.\\
\indent The lysine residue involved in retinal binding is conserved in the \textit{S. punctatus} sequence, but is absent in the \textit{B. dendrobatidis} and \textit{A. macrogynus} sequences. This is notable for the potential functional and evolutionary implications, especially in light of its presence in the \textit{S. punctatus} structure. However, experimental evidence suggests that the covalent linkage facilitated by the lysine residue, while highly desired and most evolutionarily favorable \cite{Sekharan2011}, is not necessary for activation of the light-driven cascades in bacteriorhodopsin \cite{Schweiger1994} and rhodopsin \cite{Zhukovsky1992}. In the case of bacteriorhodopsin specifically, a K216A mutant was generated and homologously expressed in \textit{Halobacterium salinarium} L33 and provided with retinylidene-n-alkylamines to achieve a Schiff-base construct withought the covalent linkage. As this mutation mirrors the \textit{B. dendrobatidis} protein (alanine present at the critical position), this suggests that perhaps a non-traditional chromophore is required for rhodopsin function in \textit{B. dendrobatidis}.\\
\indent A broad, non-covalent docking screen using ligands similar to 11-\textit{cis}-retinal was performed to assess the capacity of the \textit{B. dendrobatidis} and \textit{A. macrogynus} pockets to accomodate, structurally, a ligand of this shape. The results of this screen suggest that these pockets are indeed sufficiently large to accomodate a molecule of that size. Furthermore, several 11-cis-retinal analogous molecules present lower binding affinities for these pockets, suggesting functionality is present despite the absence of the conserved lysine residue. Thus, the lack of the lysine in these two structures does not necessarily imply that they are non-functional. Future work in the form of \textit{in vitro} functional assays or \textit{in silico} chemical screens may be necessary to further understand the exact functional nature of this protein.\\
\indent Protein models known to be correct have higher 3D profile scores \cite{Luthy1992} compared to incorrectly modeled structures. As indicated in Table 4, the models for \textit{B. dendrobatidis} and \textit{S. punctatus} using the Type 2 rhodopsin structures 2Z73 and 1U19, respectively, had scores nearly double those of the same sequences modeled against the Type 1 sensory rhodopsin II structure 1H68. As could be expected, the experimentally determined crystal structures used as templates had scores 1.5-2 times larger than the modeled structures (80-100). Longer proteins tend to have higher scores in general. All protein models scored were approximately equal in size (approx. 350 aa), with the exception of AMAG00698 (536 aa). As previous work has shown that Type 1 and Type 2 opsin proteins have similar but not quite identical structures \cite{Spudich2000}, this finding supports the hypothesis that the chytrid sequences are Type 2 and not Type 1 rhodopsins.\\
\indent \textit{In silico} docking screens were performed to assess how the \textit{S. punctatus} homology model binds to known ligands, as it is the only Type 2 rhodopsin identified in chytrids which posesses the conserved lysine and counterion residues. Based on this screen, 9-\textit{cis}-retinal and 11-\textit{cis}-retinal appeared to be the most favorable ligands for use by \textit{S. punctatus}. As such, 9-\textit{cis} isomer was used in subsequent refinement by molecular dynamics. When compared to the squid crystal structure (PDB ID: 2Z73) and its canonical 11-\textit{cis}-retinal ligand, the \textit{S. punctatus}+9-\textit{cis}-retinal complex reaches a plateau after more time and at a greater resolution. However both complexes are highly stable. Thus, the \textit{S. punctatus}+9-\textit{cis}-retinal complex after MD simulations is a good candidate for future work involving refined docking screens, and supports the hypothesis that this GCPR is a functional photoreceptor.\\
