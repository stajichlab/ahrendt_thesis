%%%%%%%%%%%%%%%%%%%%%%%%%%%%%%%%%%%%%%%%%%%%%%%%%%%%%%%%%%%%%%%%%%%%%%%
%% Document: Thesis for PhD at UC Riverside                          %%
%% Description: A comparative analysis of environment sensing in EDF %%
%% Author: Steven Ahrendt                                            %%
%%%%%%%%%%%%%%%%%%%%%%%%%%%%%%%%%%%%%%%%%%%%%%%%%%%%%%%%%%%%%%%%%%%%%%%
% RHODOPSIN TABLES %
%%%%%%%%%%%%%%%%%%%%

% PROCHECK results for homology models
% latex table generated in R 3.1.1 by xtable 1.7-3 package
% Wed May 20 14:53:55 2015
\begin{table}[tbp]
\centering
\begin{tabular}{rllllll}
  \hline
\hline
 & Spun & Bden & Amac & Tpac & Btau & Hsap \\ 
  \hline
Most favored regions & 90.7\% & 86.1\% & 66.4\% & 90.0\% & 79.2\% & 91.1\% \\ 
  Additional allowed regions & 7.5\% & 11.6\% & 24.1\% & 9.1\% & 15.8\% & 7.6\% \\ 
  Generously allowed regions & 1.2\% & 1.4\% & 6.0\% & 0.0\% & 3.8\% & 0.6\% \\ 
  Disallowed regions & 0.6\% & 0.9\% & 3.4\% & 0.0\% & 1.2\% & 0.6\% \\ 
   \hline
Verify3D & 73.60 & 55.41 & 131.62 & 87.85 & 109.14 & 00.0 \\ 
   \hline
\hline
\end{tabular}
\caption[Quality metrics for rhodopsin structures]{PROCHECK Ramachandran plot results for chytriopsin and melatonin homology models, and animal rhodopsin crystal structures.} 
\label{tab:ChRhodS_Procheck}
\end{table}
% RMSD of crystal structures
% latex table generated in R 3.1.1 by xtable 1.7-3 package
% Wed May 20 14:53:55 2015
\begin{table}[tbp]
\centering
\begin{tabular}{rrrrrrr}
  \hline
\hline
 & Spun & Bden & Amac & Tpac & Btau & Hsap \\ 
  \hline
Spun & 0.00 & 2.54 & 3.05 & 2.52 & 3.12 & 4.92 \\ 
  Bden & 2.54 & 0.00 & 3.33 & 2.55 & 2.70 & 5.11 \\ 
  Amac & 3.05 & 3.33 & 0.00 & 3.71 & 2.91 & 5.81 \\ 
  Tpac & 2.52 & 2.55 & 3.71 & 0.00 & 3.09 & 4.96 \\ 
  Btau & 3.12 & 2.70 & 2.91 & 3.09 & 0.00 & 5.50 \\ 
  Hsap & 4.92 & 5.11 & 5.81 & 4.96 & 5.50 & 0.00 \\ 
   \hline
\hline
\end{tabular}
\caption[Backbone RMSD measurements for rhodopsin structures]{Pairwise backbone RMSD measurements for chytriopsin and melatonin homology models, and animal rhodopsin crystal structures, calculated using the Bio3D package.} 
\label{tab:ChRhodS_RMSD}
\end{table}
% Important Residues
% latex table generated in R 3.1.1 by xtable 1.7-3 package
% Wed May 20 14:53:55 2015
\begin{table}[tbp]
\centering
\begin{tabular}{rllllll}
  \hline
\hline
 & Description & Bt1U19 & Tp2Z73 & Bd & Sp & Am \\ 
  \hline
1 & Salt Bridge & R177 & A176 & K145 & A156 & T203 \\ 
  2 & Salt Bridge & D190 & D189 & D158 & D169 & A223 \\ 
   \hline
3 & Binding pocket & T118 & G116 & Q88 & V99 & Q139 \\ 
  4 & Misc & E122 & F120 & A92 & S103 & V143 \\ 
  5 & Binding pocket & W265 & W274 & W317 & W268 & W468 \\ 
  6 & H-bond core/Photoisomerase counterion & E181 & E180 & R149 & Q160 & -- \\ 
  7 & Misc & I189 & F188 & Y157 & G168 & -- \\ 
  8 & Binding pocket/H-bond network w/Y268 & Y191 & Y190 & Y159 & W170 & -- \\ 
  9 & Misc & M207 & M204 & L173 & M183 & L237 \\ 
  10 & Misc & F208 & F205 & I174 & C184 & A238 \\ 
  11 & Misc & F212 & F209 & V178 & V188 & L242 \\ 
  12 & Binding pocket & W265 & W274 & W317 & W268 & W468 \\ 
  13 & Binding pocket/H-bond network w/Y191 & Y268 & Y277 & T320 & Y271 & Y471 \\ 
   \hline
14 & Conserved Lysine & K296 & K305 & A347 & K320 & S498 \\ 
  15 & Counterion & E113 & Y111 & V83 & D94 & N134 \\ 
   \hline
16 & [L]AxAD & L79 & L76 & V48 & L60 & L97 \\ 
  17 & L[A]xAD & A80 & A77 & V49 & S61 & -- \\ 
  18 & LAx[A]D & A82 & S79 & S51 & T63 & -- \\ 
  19 & LAxA[D] & D83 & D80 & D52 & D64 & D101 \\ 
   \hline
20 & DisulfideBond & C110 & C108 & C80 & C91 & C131 \\ 
  21 & DisulfideBond & C187 & C186 & C155 & C166 & C220 \\ 
   \hline
22 & [E]RY & E134 & D132 & N104 & E115 & E155 \\ 
  23 & E[R]Y/IonicLock w/E247 & R135 & R133 & H105 & R116 & R156 \\ 
  24 & ER[Y] & Y136 & Y134 & Y106 & Y117 & Y157 \\ 
   \hline
25 & [N]PxxY & N302 & N311 & N353 & N326 & N504 \\ 
  26 & N[P]xxY & P303 & P312 & P354 & P327 & P505 \\ 
  27 & NP[x]xY & V304 & M313 & I355 & V328 & L506 \\ 
  28 & NPx[x]Y & I305 & I314 & V356 & L329 & L507 \\ 
  29 & NPxx[Y] & Y306 & Y315 & F357 & F330 & S508 \\ 
   \hline
30 & IonicLock w/E247 & V138 & V136 & V108 & A119 & R159 \\ 
  31 & IonicLock w/R135 & E247 & E256 & L299 & E250 & A450 \\ 
   \hline
\hline
\end{tabular}
\caption[Locations of key residues in rhodopsin structural motifs]{Conserved Rhodopsin motifs in X-ray crystal structures and chytropsin homology models. Abbreviations: Bt1U19=\textit{Bos taurus} X-ray crystal structure (PDBID 1U19); Tp2Z73=\textit{Todares pacificus} X-ray crystal structure (PDBID 2Z73);} 
\label{tab:ChRhodS_residues}
\end{table}
% Autodock Covalent docking results
% latex table generated in R 3.1.1 by xtable 1.7-3 package
% Wed May 20 14:53:55 2015
\begin{table}[tbp]
\centering
\begin{tabular}{lrrrrrrr}
  \hline
\hline
Structure & X11.cis & all.trans & A2 & A3 & A4 & X9.cis & X13.cis \\ 
  \hline
Squid & -5.52 & -4.75 & -4.33 & -2.51 & -3.10 & -5.41 & -2.55 \\ 
  Spun & 16.68 & 24.18 & 24.78 & 26.06 & 25.58 & 9.56 & 26.24 \\ 
  SRII & -4.67 & -5.72 & -5.83 & -5.42 & -5.69 & -3.70 & -5.86 \\ 
  Nop-1 & -4.30 & -5.70 & -5.20 & -4.01 & -5.27 & -5.27 & -4.67 \\ 
  Beme-BR & -4.81 & -1.28 & -2.95 & -2.07 & -2.10 & -2.50 & -4.13 \\ 
   \hline
\hline
\end{tabular}
\caption[Interaction energies for covalent docking]{Interaction energies for covalent docking with Autodock. Average of five lowest energy conformations using various retinal isomers covalently bound to rhodopsin models and crystal structures} 
\label{tab:ChRhodS_CovDock}
\end{table}