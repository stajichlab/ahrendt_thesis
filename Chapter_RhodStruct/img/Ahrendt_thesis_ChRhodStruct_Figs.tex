%%%%%%%%%%%%%%%%%%%%%%%%%%%%%%%%%%%%%%%%%%%%%%%%%%%%%%%%%%%%%%%%%%%%%%%%%%%%%%%%%
%% Document: Thesis for PhD at UC Riverside                                    %%
%% Title: Investigating the evolution of environmental and biotic interactions %%
%%          in basal fungal lineages through comparative genomics              %%
%% Author: Steven Ahrendt                                                      %%
%%%%%%%%%%%%%%%%%%%%%%%%%%%%%%%%%%%%%%%%%%%%%%%%%%%%%%%%%%%%%%%%%%%%%%%%%%%%%%%%%
%% RHODOPSIN FIGURES %%
%%%%%%%%%%%%%%%%%%%%%%%
% Photosensory protein distribution in Fungi
\begin{figure}[htbp]
  \includegraphics[width=4in]{./Chapter_RhodStruct/img/SpStructure.png}
  \caption[Structural details of the \textit{S. punctatus} chytriopsin homology model.]{\textit{S. punctatus} residues are colored according to function: orange (binding pocket residues), red (putative counterion), purple (disulfide bond), yellow (salt bridge), dark blue (NPxxY motif), and pink \& black (ion lock). Light purple functional and backbone residues belong to \textit{T. pacificus}, while grey backbone residues belong to \textit{S. punctatus}. The ideal position of the 11-\textit{cis}-retinal ligand, taken from the \textit{T. pacificus} crystal structure, is shown in green. A) C$\alpha$ backbone structural alignment of \textit{S. punctatus} homology model and \textit{T. pacificus} x-ray crystal structure. B) Topography plot of membrane spanning regions of \textit{S. punctatus} homology model. C) Detail of \textit{S. punctatus} binding pocket residues aligned with those of \textit{T. pacificus}. D) Detail of \textit{S. punctatus} disulfide bond (purple) and salt bridge (yellow) regions aligned with those of \textit{T. pacificus}. The view is from the top (extracellular side) of the protein, into the 11-\textit{cis}-retinal (green) binding pocket. E) Detail of \textit{S. punctatus} ERY and NPxxY regions aligned with those of \textit{T. pacificus}.}
  \label{fig:ChRhodS_SpStructure}
\end{figure}

%Non-covalent docking screen: Bden
\begin{figure}[htbp]
  \includegraphics[]{./Chapter_RhodStruct/img/dockPlot.png}
  \caption[Docking screen for \textit{Bd} and \textit{Am} models]{A non-covalent docking screen using Autodock 4 and ligands obtained from ZINC against homology models for \textit{B. dendrobatidis} (red), \textit{A. macrogynus} (gray), and \textit{T. pacificus} (squid; 2Z73) (green). Histograms show distribution of ten lowest binding energies for protein-ligand conformations. Binding energy given on X-axis; ligand given on Y-axis. Ligands were obtained from ZINC database based on similarity to 11-\textit{cis}-retinal.}
  \label{fig:ChRhodS_NonCovDock}
\end{figure}

% MD analysis: Energy
\begin{figure}[htbp]
  \includegraphics[width=4in]{./Chapter_RhodStruct/img/TpSp_mdoutRMS.png}
  \caption[MD summary plots]{Overview plots of MD simulation runs of squid 2Z73 crystal structure with 11-\textit{cis}-retinal (purple) and \textit{S. punctatus} model with 9-\textit{cis}-retinal (gray). A) Potential energy over course of simulation. During the simulation, both structures remain relatively stable. The \textit{S. punctatus} stucture has substantially lower potential energy than the squid structure. B) RMSd fits over course of simulation. The RMSd fit of the \textit{S. punctatus} structural model increases much more rapidly than that of the squid structure, but ultimately reaches a plateau after approximately 6 ns.}
  \label{fig:ChRhodS_MDEnergyRMSD}
\end{figure}
