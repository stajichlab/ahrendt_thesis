%%%%%%%%%%%%%%%%%%%%%%%%%%%%%%%%%%%%%%%%%%%%%%%%%%%%%%%%%%%%%%%%%%%%%%%%%%%%%%%%%
%% Document: Thesis for PhD at UC Riverside                                    %%
%% Title: Investigating the evolution of environmental and biotic interactions %%
%%          in basal fungal lineages through comparative genomics              %%
%% Author: Steven Ahrendt                                                      %%
%%%%%%%%%%%%%%%%%%%%%%%%%%%%%%%%%%%%%%%%%%%%%%%%%%%%%%%%%%%%%%%%%%%%%%%%%%%%%%%%%
% INTRODUCTION %
%%%%%%%%%%%%%%%%
\chapter{Introduction to Early-diverging Fungi}
%%%%%
% Fungal phylogenetics
%%%%%
\section{Overview of fungal phylogenetics and the importance of the basal lineages}
Various studies have attempted to estimate the date of the divergence of fungi from animals \cite{Taylor2006}. These studies place this evolutionary split approximately 1 billion years ago, with a range of around $\pm$500 MYA: 600 MYA \cite{Berbee1993}, 965 MYA \cite{doolittle1996}, and 1.6 BYA \cite{Wang1999}. These loose approximations are based on correlation between evolutionary events in fungi and in other organisms, and are under continued re-evaluation and refinement \cite{Berbee2010}.
\indent A comprehensive review of a collection of 21st century phylogenetic studies \cite{Hibbett2007} proposes that the Fungal Kingdom contains seven phyla: Microsporidia, Chytridiomycota, Blastocladiomycota, Neocallimastigomycota, Glomeromycota, Basidiomycota, and Ascomycota, with the most recent inclusion of the phylum Cryptomycota \cite{Jones2011}.\\
\indent The most recently diverged phyla are the Ascomycota and Basidiomycota, and together make up the subkingdom Dikarya. This subkingdom is so named because its members undergo cell fusion (plasmogamy) without nuclear fusion (karyogamy) during sexual development, resulting in cells with nuclei from individual parents ("dikaryons"). Collectively, it is the most widely studied group and is home to several model and non-model organisms of research interest, including \textit{Neurospora crassa}, \textit{Saccharomyces cerevisiae}, and \textit{Aspergillus} spp.\\
\indent Going further back in time is the phylum Glomeromycota. This group contains mycorrhizal fungi (arbuscular and ento-) which associate with approximately 90\% of all plant species and are thus of greate ecological importance. This phylum also contains four subphyla previously attributed to the phylum Zygomycota \cite{White2006}: Mucormycotina, Entomophthoromycotina, Zoopagomcotina, and Kickxellomycotina. These subphyla contain primarily terrestrial fungi of various medical and industrial research interest.\\ 
\indent Closest to the fungal-animal evolutionary divergence are the basal fungal lineages: the Microsporidia, Cryptomycota, Chytridiomycota, Blastocladiomycota, and Neocallimastigomycota. These groups are sometimes collectively referred to as "chytrids", although this is not to be confused with the formal phylum Chytridiomycota. These organisms exist as either motile, flagellated zoospores, or as sessile, non-flagellated, spore-producing sporangia. The asexual life cycle displayed by these lineages involves oscillation between both stages, wherein the zoospore seeks an appropriate environmental substrate, encysts upon it and retracts its flagellum, develops a cell wall and undergoes several rounds of mitotic cell division, and ultimately produces and releases hundreds of new zoospores [citation]. Certain species within the Blastocladiomycota have demonstrated sexual reproduction with zoospores of different mating types [citation needed].\\
%%%%%
% Why they are important
%%%%%
\indent Basal fungal lineages are presumed to have a nearly cosmopolitan distribution [citation needed]. Members of these lineages fulfill nearly all varieties of ecological niches, from saprotrophic feeders of forest detritus (\textit{Spizellomyces punctatus} (Chytridiomycota)), to pathogenic associations with insects (\textit{Coelomomyces lativittatus} (Blastocladiomycota)), plants (\textit{Olpidium brassicae}), vertebrates (\textit{Batrachochytrium dendrobatidis} (Chytridiomycota)), nematodes (\textit{Catenaria anguillulae} (Blastocladiomycota)) and algae (), and even to intracellular symbioses with other chytrids (\textit{Rozella allomyces} (Cryptomycota)). \\
\indent Due to their accepted phylogenetic placement as sister to the Metazoan lineages, chytrid lineages represent an opportunity to infer characteristics presumed to have been present in the fungal-animal common ancestor. \\
%%%%%
% EDF lineages
%%%%%
\indent The basal fungal lineages are those that have been demonstrated to be within the Fungi (characterized as distinct from other water molds, like Oomycetes) but sister to the other fungal lineages (the Dikarya and paraphyletic Zygomycete lineages). The basal lineages, currently understood as comprising the Cryptomycota, Chytridiomycota, Blastocladiomycota, Monoblepharidomycota, and Neocallimastigomycota, represent less than 2\% of all described fungi \cite{Stajich2009}. 
\indent Attempts at formal description of early-diverging fungi, based primarily on collection and observation, began as early as 1858 and proceeded through the latter half of 19th century with pioneering work of researchers such as Schroter, Fischer, Zopf, Lowenthal, Nowakowski, and Woronin \cite{lwerPhycomyces}. A primarily systematic approach allowed for the establishment of (among others) the order Chytridiales, defined broadly as lacking mycelium and having an unknown sexual cycle, and the order Blastocladiales, defined as having mycelium and a sexual reproductive component.\\
\indent Revisions of fungal phylogenetics continued through the 20th century and included Alexopolous's series on Introductory Mycology in the 1950s.\\
\indent Significant microscopy work was carried out on chytrid species as early as 1953 by Koch and others, which allowed for the discussion of zoosporic ultrastructure characters. In the 1970s, the then recent technology of electron microscopy allowed for finer grain examinations of chytrid ultrastructure.\\
\indent In 1993, Mims and Blackwell for fourth edition (1996) using DNA-based phylogentics (specifically SSU rDNA). Importantly, established Fungi as monophyletic, established chytrids as Fungi, and excluded other heterokont flagellates (eg oomycetes).\\
\indent In 1999, James et al attempted to reconcile the previous ultrastructure-based phylogeny (using characters such as zoospore discharge, thallus development, ultrastructural features) with modern molecular techniques (eg ssu rDNA) used in assessing phylogentic relationships in other fungal groups.\\
\indent By 2006, at the time of the establishment of the Deep Hypha coordination network, the Chytridiomycota was accepted as one of the four major fungal phyla (alongside the Zygomycota, Basidiomycota, and Ascomycota; hereafter, the phylogenetic details of these latter three lineages will be discussed in much less detail, if at all). Additionally, the Chytridiomycota was agreed to be the most basal lineage primarily due to the presence of a flagellated life stage. The typical life cycle for these basal lineages begins with motile, posteriorly uniflagellate zoospores. After locating a suitable substrate, these zoospores encyst and develop into sporangia. Within the sporangia, asexual reproduction results in the production of more zoospores. The cycle is renewed when these zoospores are discharged from the sporangia.\\
\indent Concerning chytrids as a system for study, interest waned in the latter 20th century, driven in part because few species are of substantial economic importance \cite{James1999} as well as moderate difficulty in collection and culturing methods. More widespread interest renewed in the early 2000s due to the acceptance of \textit{Batrachochytrium dendrobatitids} as a global pathogen and the causative agent of worldwide amphibian decline (first recognized as concerning in 1989, though data indicates that widespread decline began in 1970 in US, Puerto Rico, and northeastern Australia \cite{Stuart2004}). \\
\indent In 1998, \textit{Bd}-caused chytridiomycosis emerged through experimental research as leading front-runner for cause of widespread decline \cite{Berger1998}, and in 1999, \textit{B. dendrobatidis} isolated from a blue poision dart frog which had died in Washington, DC, was formally described \cite{Longcore1999}. \\
%%%%%
% Chytrid Bioinformatic resources 
%%%%%
\section{Development of Chytrid bioinformatic resources}
As genomic resources became more widely available, chytrid genomes became more easily developed. The first available resource was an expressed sequence tag (EST) dataset published in 2005 for the Blastocladiomycete \textit{Blastocladiella emersonii} \cite{Ribichich2005}. This collection of 16,984 high-quality ESTs provided a first approach to understanding gene complexity in chytrids. In 2006, a draft assembly for the genome of \textit{B. dendrobatidis} strain JEL423 was made publically available through the Broad Institute Fungal Genome Initiative (http://www.broadinstitute.org/annotation/genome/batrachochytrium\_dendrobatidis/MultiHome.html). The resulting assembly is 23.72 Mb, represents 7.4X coverage of this diploid strain, and was the first whole genome assembly for any chytrid. In 2008, a second draft genome was released for \textit{B. dendrobatidis} strain JAM81 through the Joint Genome Institute (http://genomeportal.jgi-psf.org/Batde5/Batde5.home.html). This assembly is 24.3 Mb and represents 8.74X coverage. \\
\indent As part of a push to understand the Origins of Multicellularity, the genomes for the \textit{Allomyces macrogynus} (Blastocladiomycete) and the exclusively terrestrial \textit{Spizellomyces punctatus} (Chytridiomycete) (24.1 Mb) were sequenced by the Broad Institute in 2009. \\
\indent In 2011, the non-pathogenic \textit{Homolaphlyctis polyrhiza} JEL42, the closest relative to \textit{Bd} was sequenced for comparision to try to identivy aspects of pathogenicity in \textit{Bd} \cite{Joneson2011}. The resulting assembly size was inferred at 26.7 Mb (haploid) and represented 11.2X coverage. In 2014, with the help of a postdoctoral researcher in the Stajich Lab, Dr. Peng Liu, I generated Illumina sequencing libraries for \textit{H. polyrhiza} and helped assess the assembly and annotation with Dr. Stajich; the results of which are described in Chapter~\ref{chap:Hp_inhibition}.\\
\indent \textit{Gonapodya prolifera} (Monoblepharidomycota) is an aquatic fungus with both sexual and asexual reproductive schemes, and encompassing both hyphal and zoosporic growth stages. A genome assembly was published in 2011 through JGI. \textit{Catenaria anguillulae} (Blastocladiomycota) is , \\
\indent Piromyces and Orpinomyces are members of the Neocallimastigomycota \\
\indent Rozella allomycis, the intracellular parasite of Allomyces sp. was sequenced in 2013.\\
\indent Also in 2014 the first transcriptome of the mosquito pathogen \textit{Coelomomyces lativitattus} (\textit{Cl}) was generated by isolating RNA from gametes emerging from copepods. The RNA extraction and Illumina library preparation was performed by Rob Hice, a reasercher in Dr. Brian Federici's lab at UCR. The sequencing was performed at the UCR IIGB Genomics core. The resulting sequence data was assembled and annotated using scripts provided by Dr. Jason Stajich, and my analysis is described in Chapter~\ref{chap:Clat_transcriptome}.\\
\indent Rapid advances in the feasability of genome sequencing have yielded and will continue to yield an increasing number of fungal genomes, especially those in the early-diverging lineages, for comparative analyses. While incorporating comparisons to already well-characterized representative fungal groups, this dissertation work gives specific focus to members of the early-diverging lineages, and in particular to the ones for which genomic resources are available: the amphibian pathogen \textit{Batrachochytrium dendrobatidis} and its closest, non-pathogenic relative, \textit{Homolaphlyctis polyrhiza}; the saprobic Blastocladiomycete \textit{Spizellomyces punctatus}, and related \textit{Catenaria anguillulae}; the Monoblepharidomycete \textit{Gonapodya prolifera}; the Neocallimastigomycetes \textit{Piromyces sp.} and \textit{Orpinomyces sp.}; and the aquatic Blastocladiomycete \textit{Allomyces macrogynus} and its intracellular, Microsporidia parasite, \textit{Rozella allomycis}.

%%%%%
% Research goals
%%%%%
\section{Goals of this body of research}
Due to the continued development and availability of bioinformatic resources for fungi in general, and the basal lineages specifically, this thesis research focuses on comparative genomics of these systems. It is presented in four chapters comprising three aspects of biology focused on sensing and interpretation of biotic and environmental signals.\\
\subsection{Competition-based secondary metabolism}
Interactions between microorganisms are facilitated by biological signals. These include proteins, small molecules, and various chemical compounds, either bound to the cell surface or secreted into the environment. Many of these compounds can be classified as secondary metabolites: chemicals not required for growth or development of the organism.\\
\indent Resource competition likely plays a role in the evolution of natural antifungal production \cite{(Vicente2003}. Secretion by an organism in a resource-limeted environment of secondary metabolites which also happen to negatively impact neighboring organisms would confer a selective advantage upon the producer. \\
\indent Comparative genomic analyses have identified a host of degradation enzymes in basal fungi, suggesting saprotrophic and sometimes pathogenic associations with other organisms. There are few explored examples of secreted or secondary metabolite molecules produced by any of the zoosporic fungi. 
\indent Secondary metabolite production as it applies to basal fungi is discussed in more detail in Chapter~\ref{chap:Hp_inhibition} using the non-pathogenic Chytridiomycete \textit{Homolaphlyctis polyrhiza} JEL142. In this chapter, I describe the initial observation of \textit{Hp} inhibition of the vegetative hyphal growth of \textit{N. crassa} via an unknown secreted compound. Then, I go on to describe the observational assays using the sporangia of related Chytridiomycetes, and probing the breadth of non-Chytridiomycete fungi whose growth is susceptible to \textit{Hp}, encompassing the Ascomycete, Basidiomycete, and Zygomycete species, and including both temperature and proteinase screens. Finally, to better explore \textit{Hp} gene content, I produced an improved genome assembly and annotation, by assisting Dr. Peng Liu and Dr. Jason Stajich in the collection of fungal material and the assembly and annotation of the resulting genome sequence.\\
\subsection{Entomopathogenesis in fungi}
Adhesion: Intricate details of fungal adhesion (eg kinetics of macromolecular attachement, nature of reactions, and strucre/chemicsty of fungal and insect surfaces) has gone almost completely unknown for three decades. \cite{atlasBook}. Various hypotheses have been suggested and examined, including eletrocstatic forces  (see Cherbit+Delmas1979 and Coucias+Latge1986), surface receptors, glycoproteins, and various haemagglutinins (see grula1984 and latge1984b) (also boucias1988), and various carbohydrate compounds (see kerwin+washino1986b). \\
\indent Germination: Appressoirium development an secretion of adhesive (see zaccharuck1981). spore germination, while recognized as critical, is poorly understood process. Disconnect between in vitro and in vivo tests/observations\\
\indent 
\begin{itemize}
  \item What types of fungi?
  \item What means of infection/pathogenesis?
  \item General focus on mosquito control. A subset of fungi are entomopathogenic with respect to mosquitos. Of these, only \textit{Coelomomyces lativittatus} (\textit{Cl}) (Blastocladiomycota) is the only known chytrid. While initially studied as a promising avenue for mosquito control, and an alternative to traditional pesticides, difficulties in culturing \textit{Cl} have lead to a decline in its research. However its specific host range and continued search for pesticide alternatives have allowed it to persist as an interesting avenue of research.\\
\end{itemize}
\subsection{Fungal Light Sensing}
A description of findings dealing with structural work on the identified "chytropsin" and opsin-like proteins is given in Chapter~\ref{chap:Rhodopsin}.\\
\subsubsection{Beta-carotene}
\begin{itemize}
  \item What is the basic/what is it used for?
  \begin{itemize}
    \item Protection from sun
    \item used in retinal biosynthesis and rhodopsin-mediated photosensing
  \end{itemize}
  \item What is the evolutionary history?
  \item what is the pathway?
  \item Why is it important
\end{itemize}
\subsection{Eukaryotic Flagellar motility}
One of the defining characteristics of the early-diverging fungal lineages is the presence of a posterior flagellum, which is used by the zoospores for motility. The chytrid flagellar apparatus is composed of the flagellar stalk (axoneme), the kinetosome (basal body), and the rootlet system \cite{Barr1981}. Additionally the characteristic 9+2 arrangment characteristic in other eukaryotic flagellated organisms persists.\\
\indent Use of flagellar characteristics as a means of grouping chytrid species.
\begin{itemize} 
  \item What is evolutionary history?
  \item Why is it imporant in this contex?
\end{itemize}
\indent The chytrid flagellum is the primay method of zoospore motility. In most cases, the chytrid flagellum exists as a posteriorly oriented appendage, with a few exceptions. The zoospores of the Neocallimastigomycota lineages, species most commonly found in the anaerobic environment of the mammalian rumen, are posteriorly multiflagellated \cite{}. In the Blastocladiomycota, \textit{Coelomomcyes} species are biflagellate during a part of their life cycle. A more complete description of the life cycle is provided in Chapter~\ref{chap:Clat_transcriptome}. Briefly, however, prior to infection of the mosquito, the uniflagellate gametes of opposing mating types fuse to form a biflagellate zygote. \cite{}\\ 
\indent A short treatment of comparative genomics work on the chytrid flagellar apparatus is given in Appendix~\ref{app:Flagella}.\\
