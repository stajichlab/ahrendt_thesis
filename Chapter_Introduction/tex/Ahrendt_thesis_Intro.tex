%%%%%%%%%%%%%%%%%%%%%%%%%%%%%%%%%%%%%%%%%%%%%%%%%%%%%%%%%%%%%%%%%%%%%%%%%%%%%%%%%
%% Document: Thesis for PhD at UC Riverside                                    %%
%% Title: Investigating the evolution of environmental and biotic interactions %%
%%          in basal fungal lineages through comparative genomics              %%
%% Author: Steven Ahrendt                                                      %%
%%%%%%%%%%%%%%%%%%%%%%%%%%%%%%%%%%%%%%%%%%%%%%%%%%%%%%%%%%%%%%%%%%%%%%%%%%%%%%%%%
% INTRODUCTION %
%%%%%%%%%%%%%%%%
\chapter{Introduction to the Basal Fungal lineages}
\section{Overview of fungal phylogenetics and the importance of the basal lineages}
%%%%%
% Fungal phylogenetics
%%%%%
The Fungi are one of the major kingdoms of eukaryotic life on Earth. Various studies have attempted to estimate the date of the emergence of the Kingdom Fungi \cite{Taylor2006}, when it diverged from the metazoan lineages. These studies place this occurance at approximately 1 billion years ago, with a range of around $\pm$500 MYA: 600 MYA \cite{Berbee1993}, 965 MYA \cite{Doolittle1996}, and 1.6 BYA \cite{Wang1999}. These loose approximations are based on correlation between evolutionary events in fungi and in other organisms, and are under continued re-evaluation and refinement \cite{Berbee2010}.\\
\indent A comprehensive review of a collection of 21st century phylogenetic studies \cite{Hibbett2007} proposes that the Fungal Kingdom comprises seven phyla: Microsporidia, Chytridiomycota, Blastocladiomycota, Neocallimastigomycota, Glomeromycota, Basidiomycota, and Ascomycota, with the most recent inclusion of the phylum Cryptomycota \cite{Jones2011}.\\
\indent The most recently diverged fungal phyla are the Ascomycota and Basidiomycota, and together make up the subkingdom Dikarya. This subkingdom is so named because its members undergo cell fusion (plasmogamy) without nuclear fusion (karyogamy) during sexual development, resulting in cells with nuclei from individual parents ("dikaryons"). These organisms are muticellular, with terrestrial habitats, sexual and asexual life cycle components, and filamentous growth structures. Collectively, the Dikarya is the most widely studied group and is home to several model and non-model organisms of research interest, including the model filamentous Ascomycete \textit{Neurospora crassa}, the economically critical \textit{Saccharomyces cerevisiae}, and the numerous medically relevant \textit{Aspergillus} spp.\\
\indent Going further back in time is the phylum Glomeromycota. This group contains mycorrhizal fungi (arbuscular and ento-) which associate with approximately 90\% of all plant species and are thus of greate ecological importance. This phylum also contains four subphyla \textit{incertae sedis}: Mucormycotina, Entomophthoromycotina, Zoopagomcotina, and Kickxellomycotina \cite{White2006}. These subphyla were previously classified in the phylum Zygomycota and contain primarily terrestrial fungi of various medical and industrial research interest. It is important to point out here that the definitive phylogenetic relationship between the Glomeromycota and Zygomycota \textit{incertae sedis} lineages is unresolved and is a current focus of research \cite{Hibbett2007}. Therefore in this text, the nonflagellated members of the Glomeromycota, Mucormycotina, Entomophthoromycotina, Zoopagomycotina, and Kickxellomycotina will, for convenience, be referred to collectively as Zygomycota.\\ 
\indent Closest to the fungal-animal evolutionary divergence are the basal fungal lineages: the Microsporidia, Cryptomycota, Chytridiomycota, Blastocladiomycota, and Neocallimastigomycota. It is these lineages, particularly the Blastocladiomycota and Chytridiomycota, on which this dissertation will primarily be regarding. These groups are sometimes collectively referred to as "chytrids", although this is not to be confused with the formal phylum Chytridiomycota. Broadly speaking, these organisms have asexual life cycles which progress through development as motile, flagellated zoospores, followed by sessile, non-flagellated, spore-producing sporangia. During the motile stage, the zoospore seeks an appropriate environmental substrate, encysts upon it, retracts its flagellum, and develops a cell wall. Many species will stay dormant in this stage as a thick-walled "resting spore", and only develop into a thin-walled zoosporangia after a certain time period \cite{James2006Blasto}. Other chytrid species will instead progress directly to the zoosporangia stage, undergo several rounds of mitotic cell division, and ultimately produces and releases hundreds of new zoospores \cite{James2006Blasto}. \\
\indent While generally being described as asexual, certain species within the Blastocladiomycota, such as \textit{Allomyces reticulatus} and \textit{Coelomomyces punctatus}, have demonstrated sexual reproductive cycles utilizing zoospores of different mating types \cite{Alexopoulos1996}.\\
%%%%%
% Importance of EDF
%%%%%
\indent The basal fungal lineages are characterized as true Fungi and distinct from other water molds, like Oomycetes, and fall sister to both Metazoan lineages as well as the other fungal lineages (Zygomycota, Ascomycota, and Basidiomycota). While these lineages only represent less than 2\% of all described fungi \cite{Stajich2009}, they serve as a unique system in which to infer characteristics presumed to have been present in the fungal-animal common ancestor.\\
\indent Additionally, basal fungal lineages are presumed to have a nearly cosmopolitan distribution \cite{Powell1993}. Members of these lineages fulfill nearly all varieties of ecological niches, primarily decomposition in terrestrial and aquatic environments, but also including pathogenic interactions with a wide variety of hosts: arbuscular mycorrhizal fungi (\textit{Spizellomyces punctatum} (Chytridiomycota) \cite{Paulitz1984}), insects (\textit{Coelomomyces psorophorae} (Blastocladiomycota) \cite{Zebold1979}), plants (\textit{Olpidium brassicae} (Chytridiomycota) \cite{Tewari1983}), vertebrates (\textit{Batrachochytrium dendrobatidis} (Chytridiomycota) \cite{Longcore1999}), nematodes (\textit{Catenaria anguillulae} (Blastocladiomycota) \cite{Deacon1997}) and algae (\textit{Zygorhizidium plantonicum} \cite{Canter1967}), and even to intracellular symbioses with other chytrids (\textit{Rozella allomyces} (Cryptomycota) \cite{Held1973}). This distribution of life styles speaks to the vast biological challenges they must face and therefore suggests a number of novel mechanisms which have yet to be fully studied and explored. \\
\indent Attempts at formal description of early-diverging fungi, based primarily on collection and observation, began as early as 1858 and proceeded through the latter half of 19th century with pioneering work of researchers such as Schroter, Fischer, Zopf, Lowenthal, Nowakowski, and Woronin \cite{Phycomyces}. A primarily systematic approach allowed for the establishment of (among others) the order Chytridiales, defined broadly as lacking mycelium and having an unknown sexual cycle, and the order Blastocladiales, defined as having mycelium and a sexual reproductive component.\\
\indent Revisions of fungal phylogenetics continued through the 20th century and included Alexopolous's series on Introductory Mycology in the 1950s.\\
\indent Significant microscopy work was carried out on chytrid species as early as 1953 by Koch and others, which allowed for the discussion of zoosporic ultrastructure characters. In the 1970s, the then recent technology of electron microscopy allowed for finer grain examinations of chytrid ultrastructure.\\
\indent In 1993, Mims and Blackwell for fourth edition (1996) using DNA-based phylogentics (specifically SSU rDNA). Importantly, established Fungi as monophyletic, established chytrids as Fungi, and excluded other heterokont flagellates (eg oomycetes).\\
\indent In 1999, James et al attempted to reconcile the previous ultrastructure-based phylogeny (using characters such as zoospore discharge, thallus development, ultrastructural features) with modern molecular techniques (eg ssu rDNA) used in assessing phylogentic relationships in other fungal groups.\\
\indent It was around that time that the Chytridiomycete \textit{Batrachochytrium dendrobatidis} emerged as a global pathogen and the accepted causative agent of worldwide amphibian decline \cite{Berger1998}.\\
\indent By 2006, at the time of the establishment of the Deep Hypha coordination network, the Chytridiomycota was accepted as one of the four major fungal phyla (alongside the Zygomycota, Basidiomycota, and Ascomycota).\\
\indent Despite widespread distribution in both geography and ecology, chytrids remain understudied as a whole, driven in part because few species are of substantial economic importance \cite{Powell1993,James2000} as well as moderate difficulty in collection and culturing methods. \\
%%%%%
% Chytrid Bioinformatic resources 
%%%%%
\section{History of bioinformatic resources for basal fungi}
As whole genome sequencing continues to become more widely accesible, chytrid genomes are more easily developed. The first available bioinformatic resource was an expressed sequence tag (EST) dataset published in 2005 for the Blastocladiomycete \textit{Blastocladiella emersonii} \cite{Ribichich2005}. This collection of 16,984 high-quality ESTs provided a first approach to understanding gene complexity in chytrids. In 2006, a draft assembly for the genome of \textit{B. dendrobatidis} strain JEL423 was made publically available through the Broad Institute Fungal Genome Initiative (http://www.broadinstitute.org/). The resulting assembly is 23.72 Mb, represents 7.4X coverage of this diploid strain, and was the first whole genome assembly for any chytrid. In 2008, a second draft genome was released for \textit{B. dendrobatidis} strain JAM81 through the Joint Genome Institute (http://genomeportal.jgi-psf.org/Batde5/Batde5.home.html). This assembly is 24.3 Mb and represents 8.74X coverage. \\
\indent As part of a push to understand the Origins of Multicellularity, the genomes for the \textit{Allomyces macrogynus} (Blastocladiomycete) and the exclusively terrestrial \textit{Spizellomyces punctatus} (Chytridiomycete) (24.1 Mb) were sequenced by the Broad Institute in 2009. \\
\indent In 2011, the non-pathogenic \textit{Homolaphlyctis polyrhiza} JEL142, the closest relative to \textit{Bd} was sequenced for comparision to try to identivy aspects of pathogenicity in \textit{Bd} \cite{Joneson2011}. The resulting assembly size was inferred at 26.7 Mb (haploid) and represented 11.2X coverage. In 2014, with the help of a postdoctoral researcher in the Stajich Lab, Dr. Peng Liu, I generated Illumina sequencing libraries for \textit{H. polyrhiza} and helped assess the assembly and annotation with Dr. Stajich; the results of which are described in Chapter~\ref{chap:HpInhibition}.\\
\indent \textit{Gonapodya prolifera} (Monoblepharidomycota) is an aquatic fungus with both sexual and asexual reproductive schemes, and encompassing both hyphal and zoosporic growth stages. In the environment, \textit{G. prolifera} is an active degrader of plant material \cite{Karling1977}. A draft genome assembly was published in 2011 through JGI, with the goal of identifying potentially novel degredation-related enzymes for biofuels applications.\\
\indent \textit{Catenaria anguillulae} (Blastocladiomycota) is a facultative parasite of nematodes \cite{Deacon1997}. A draft genome was published in 2010 by the JGI and represented the second Blastocladiomycete genome (after \textit{A. macrogynus}). Genomic resources for \textit{C. anguillulae} would allow for research into monitoring and potential remediation strategies as the nematodes upon which it parasitizes are themselves parasites of agriculturally important crops.\\
\indent Members of the Neocallimastigomycota lineage, first isolated and described in 1975, are found in the anaerobic environment of mammalian rumen \cite{Orpin1975}. These fungi are uniquely adapted to degradation of the high fiber content of the typical diets of cattle and sheep. Thus they are important models for potential manipulation to not only improve digestion in these livestock sources \cite{Ho1995} but also potential biofuels applications \cite{Youssef2013}. \textit{Piromyces} and \textit{Orpinomyces} are two members of this group and were sequenced in 2011 and 2013, respectively, in the hopes that understanding the genomic content would provide starting points for these applications. \\
\indent The genome of Cryptomycete \textit{Rozella allomycis}, the intracellular parasite of \textit{Allomyces}, was sequenced in 2013 \cite{James2013} and the analysis used to propose a unification of the Cryptomycota and Microsporidian lineages.\\
\indent In 2014 the first transcriptome of the mosquito pathogen \textit{Coelomomyces lativitattus} (\textit{Cl}) was generated by isolating RNA from gametes emerging from copepods. The RNA extraction and Illumina library preparation was performed by Rob Hice, a reasercher in Dr. Brian Federici's lab at UCR. The sequencing was performed at the UCR IIGB Genomics core. The resulting sequence data was assembled and annotated using scripts provided by Dr. Jason Stajich, and my analysis is described in Chapter~\ref{chap:ClatTranscriptome}.\\
\indent The near future of bioinformatics resources for the basal lineages is promising due to the efforts of the 1000 Fungal Genome Project (http://1000.fungalgenomes.org) with plans to sequence at additional chytrid fungi, including \textit{Operculomyces laminatus} JEL223, \textit{Rhizoclosmatium hyalinus} JEL800, and \textit{Obelidium mucronatum} JEL802. \\
%%%%%
% Research goals
%%%%%
\section{Hypotheses and Objectives}
\indent Rapid advances in the feasability of genome sequencing have yielded and will continue to yield an increasing number of fungal genomes, especially those in the early-diverging lineages, for comparative analyses. While incorporating comparisons to already well-characterized representative fungal groups, this dissertation work gives specific focus to members of the early-diverging lineages, and in particular to the ones for which genomic resources are available and which have obvious economic or ecological importance: the amphibian pathogen \textit{Batrachochytrium dendrobatidis} and its closest, non-pathogenic relative, \textit{Homolaphlyctis polyrhiza}; the saprobic \textit{Spizellomyces punctatus}; and the aquatic Blastocladiomycete \textit{Allomyces macrogynus} and related \textit{Catenaria anguillulae}. This thesis is presented in four chapters comprising three aspects of biology focused on sensing and interpretation of biotic and environmental signals.\\
\subsection*{Competition-based secondary metabolism and anti-fungal properties of \textit{Homolaphlyctis polyrhiza}}
Interactions between microorganisms are facilitated by biological signals. These include proteins, small molecules, and various chemical compounds, either bound to the cell surface or secreted into the environment. Many of these compounds can be classified as secondary metabolites: chemicals not required for growth or development of the organism.\\
\indent Resource competition likely plays a role in the evolution of natural antifungal production \cite{Vicente2003}. Secretion by an organism in a resource-limited environment of secondary metabolites which also happen to negatively impact neighboring organisms would confer a selective advantage upon the producer. \\
\indent Comparative genomic analyses have identified a host of degradation enzymes in basal fungi, suggesting saprotrophic and sometimes pathogenic associations with other organisms. There are few explored examples of secreted or secondary metabolite molecules produced by any of the zoosporic fungi.\\ 
\indent Secondary metabolite production as it applies to basal fungi is discussed in more detail in Chapter~\ref{chap:HpInhibition} using the non-pathogenic Chytridiomycete \textit{Homolaphlyctis polyrhiza} JEL142. In this chapter, I address three major questions regarding an initial observation I made in the Stajich lab of \textit{Hp} inhibition of the vegetative hyphal growth of \textit{N. crassa} via an unknown secreted compound. Namely, "Is Hp unique among the chytrids in this behavior?", "Is this behavior specific to \textit{N. crassa}?", and "What is the underlying biochemical mechanism by which this behavior is accomplished?". These questions are addressed using observational assays with the sporangia of related Chytridiomycetes, and probing the breadth of non-Chytridiomycete fungi whose growth is susceptible to \textit{Hp}, encompassing Ascomycete, Basidiomycete, and Zygomycete species, and including both temperature and proteinase screens. Finally, to better explore \textit{Hp} gene content, I produced an improved genome assembly and annotation, by assisting Dr. Peng Liu and Dr. Jason Stajich in the collection of fungal material and the assembly and annotation of the resulting genome sequence.\\
\subsection*{Mechanics and evolution of Fungal rhodopsin-based photosensing in the basal lineages}
\indent During the course of a given day, an organism experiences a multitude of environmental stimuli, with light being one of the most prominent. The biochemical ability to appropriately process and respond to these signals is an incredibly complex and involved task, and understanding the underlying mechanisms of these responses is an ongoing scientific challenge. \\
\indent Previous work has shown that some of the basal fungi are phototaxic (see \cite{Saranak1997} and \cite{Muehlstein1987}), however the full extend of photosensing in zoosporic fungi has not been fully explored. A recent review of fungal photobiology suggests a sporadic distribution of photosensory proteins among the non-flagellated fungal lineages (ie Zygomycota and Dikarya), with little emphasis placed on the basal lineages \cite{Idnurm2010}. There are many classes of photoreceptor proteins in fungi capable of producing a cellular response from an environmental light signal, all of which have different mechanisms of action and specializations: phytochromes, cryptochromes, the white-collar complex, and opsins \cite{Idnurm2010}. In plants, phytochromes function as day-night sensors to regulate the circadian rhythm and flowering response. This is accomplished through a conformational shift between the red and far-red sensitive forms of the protein structure \cite{Rockwell2006}. While relatively little is known about fungal phytochrome function, research on A. nidulans suggests that the phytochrome protein is a member of an elaborate complex with regulatory functions involved with the asexual-sexual transition and secondary metabolite biosynthesis \cite{Idnurm2010}. Cryptochromes, found predominantly in plants, animals, and insects, are blue-light sensitive proteins involved in circadian rhythm regulation and light activated DNA damage repair \cite{Idnurm et al. 2010}. Additional evidence suggests cryptochrome proteins play a role in mediating the phototactic behavior of sponge larvae \cite{Rivera2012}.\\
\indent First studied in the model filamentous Ascomycete fungus \textit{Neurospora crassa}, the well-characterized white-collar complex assembles as a heterodimer comprising White-collar 1 and 2 proteins. This complex functions to sense blue and near UV wavelengths, and, when active, directly interacts with DNA to regulate the circadian clock machinery, sporulation, pigmentation, and phototropism \cite{Ballario1997}, \cite{Purschwitz2006}, \cite{Corrochano2007}. \\
\indent The largest family of membrane receptors by far is that of the seven-transmembrane (7TM) receptors, comprising upwards of 800 genes \cite{Pierce2002}. This family includes various receptors for a wide range of ligands, including hormones, neurotransmitters, odorants, and photons. While there are three distinct subfamilies (A, B, and C), they share very little sequence similarity. Opsins, examples of which can be found in bacteria, archaea, and eukaryotes, are part of the largest family of 7TM proteins. One subclass of opsin, the Type 2 rhodopsins, are G-protein-coupled receptor (GPCR) proteins which function via photoisomerization of a covalently bound retinyledene chromophore, typically 11-\textit{cis}-retinal \cite{Wald1968}.\\
\indent The retinal chromophore utilized in Type 2 rhodopsin-mediated photoreception is biosynthesized from $\beta$-carotene \cite{VonLintig2000}. Photoisomerization of this chromophore results in a conformational shift to the all-\textit{trans} isomer \cite{Smith2010} and activation of the coupled heterotrimeric G-protein. A comparative analysis of auxillary proteins in basal fungi involved in this downstream signalling cascade is given in Chapter~\ref{chap:RhodAux}, and a description of findings dealing with structural and functional analyses of putative homologs of Type 2 rhodopsin in several species of basal fungi is provided in Chapter~\ref{chap:RhodStruct}.\\
\indent Biosynthesis of $\beta$-carotene begins with phytoene cyclase converting two molecules of geranylgeranyl pyrophosphate to one molecule of phytoene. Phytoene desaturase then works in a five-step pathway to convert phytoene into lycopene \cite{Hausmann2000}. Lycopene cyclase finally acts to convert lycopene to $\beta$-carotene \cite{Cunningham1994}. Subsequently, two different cleavage enzymes can potentially act on $\beta$-carotene. The enzyme $\beta$,$\beta$-carotene 15,15'-monooxygenase 1 (BCMO1) cleaves $\beta$-carotene into two all-\textit{trans}-retinal molecules, and is considered a key enzyme for retinoid metabolism \cite{Lietz2012}. The structurally related enzyme $\beta$,$\beta$-carotene 9',10'-dioxygenase (BCDO2) also acts on $\beta$-carotene to produce $\beta$-apo-10'-carotenal and $\beta$-ionone, however its physiological role is less well-characterized \cite{Lobo2012}. Comparative analysis and discussion of retinal biosynthesis enzymes is provided in Chapter~\ref{chap:ClatTranscriptome}.\\
\subsection*{Towards the development of bioinformatic resources for entomopathogenic Blastocladiomycete \textit{Coelomomyces lativittatus}}
\begin{itemize}
  \item Fungi which invade insects have been observed since antiquity (900 AD in Japan)\\
  \item Exact nature of fungal-insect relationship was not cleanly determined until 1888\\
  \item Once that was established, research switched to application of agriculture pest contro (1895-1925)\\
  \item Interest waned (1925-1960), but taxonomic knowledge greatly increased
  \item renewed interest is ongoing
\end{itemize}
\indent Certain insect-associated fungi are pathogenic on mosquitos. Of these, members of the genus \textit{Coelomomyces} in the Blastocladiomycota are the only known chytrid entomopathogens. While initially studied as a promising avenue for mosquito control, and an alternative to traditional pesticides, difficulties in culturing \textit{Cl} have lead to a decline in its research. However its specific host range and continued search for pesticide alternatives have allowed it to persist as an interesting avenue of research.\\
\indent Chapter~\ref{chap:ClatTranscriptome} presents a preliminary analysis of transcriptome data obtained from \textit{Coelomomyces lativittatus}. This analysis serves two purposes. First, it lays the groundwork for future RNASeq and proteomic studies of this organism. And secondarily, it attempts to assign molecular detail to previously published observational research about certain biochemical mechanisms (eg $\beta$-carotene production and photoreception).\\
\subsection*{Eukaryotic Flagellar motility}
One of the defining characteristics of the early-diverging fungal lineages is the presence of a posterior flagellum, which is used by the zoospores for motility \cite{Koch1958}. The chytrid flagellar apparatus is composed of the flagellar stalk (axoneme), the kinetosome (basal body), and the rootlet system \cite{Barr1981}. Additionally the characteristic 9+2 arrangment characteristic in other eukaryotic flagellated organisms persists.\\
\indent During the course of fungal evolution, there was a transition from flagellated motile aquatic single celled organisms to terrestrial multicellular organisms \cite{Taylor2006}. There is support for anywhere from a single flagellar loss event \cite{Liu2006} to at least four different such events \cite{James2006sixGene} prior to the divergence of the Zygomycota.\\
\indent The chytrid flagellum is the primay method of zoospore motility. In most cases, the chytrid flagellum exists as a posteriorly oriented appendage, with a few exceptions. The zoospores of the Neocallimastigomycota lineages, species most commonly found in the anaerobic environment of the mammalian rumen, are posteriorly multiflagellated \cite{Ho1995}. In the Blastocladiomycota, \textit{Coelomomyces} species are biflagellate during a part of their life cycle after the uniflagellate gametes of opposing mating types fuse to form a biflagellate zygote \cite{Padua1986}.\\ 
\indent A short treatment of comparative genomics work on the chytrid flagellar apparatus is given in Appendix~\ref{app:Flagella}. Included is a collection of genes which serve as a "core chytrid" flagellar geneset, which may prove useful in future assessments of chytrid and other basal fungal genomes.\\
