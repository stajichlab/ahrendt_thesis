%%%%%%%%%%%%%%%%%%%%%%%%%%%%%%%%%%%%%%%%%%%%%%%%%%%%%%%%%%%%%%%%%%%%%%%%%%%%%%%%%
%% Document: Thesis for PhD at UC Riverside                                    %%
%% Title: Investigating the evolution of environmental and biotic interactions %%
%%          in basal fungal lineages through comparative genomics              %%
%% Author: Steven Ahrendt                                                      %%
%%%%%%%%%%%%%%%%%%%%%%%%%%%%%%%%%%%%%%%%%%%%%%%%%%%%%%%%%%%%%%%%%%%%%%%%%%%%%%%%%
% RHODOPSIN AUXILLARY CHAPTER %
%%%%%%%%%%%%%%%%%%%%%%%%%%%%%%%
\chapter{A comparative analysis of auxillary components of the rhodopsin and opsin-like signalling pathways in fungi}
\label{chap:RhodAux}
\section{Introduction}
\indent Opsins are a broad class of photosensitive seven-transmembrane proteins that respond to light through photoisomerization of a retinaldehyde chromophore, typically 11-\textit{cis}-retinal. Despite similar overall structure and mechanism of activation, this group can be classified into two types based on sequence similarity and function \cite{Spudich2000}.\\
\indent Four major G$\alpha$ protein subfamilies have been identified: G$_{s}$, G$_{i}$, G$_{q}$, and G$_{12}$ \cite{Hepler1992}. Additional fungal \cite{Bolker1998} and plant \cite{Ma1994} subfamilies exist as well. The type of G-protein to which the receptor is coupled can help classify the Type 2 rhodopsins. The G$_{s}$ group contains the G$_{s}$ and G$_{olf}$  subunits, the latter being found in the olfactory neuroepithelial cells. These enzymes enhance the rate of cAMP synthesis by stimulating adenylate cyclase. Additionally, the G$_{s}\alpha$ subunit regulates Na$^{+}$ and Ca$^{2+}$ channels \cite{Hepler1992}. The G$_{i}$ group contains four subclasses: G$_{i}$, G$_{o}$, G$_{t}$, G$_{z}$. All G$_{i}$ subclasses possess a consensus sequence for ADP-ribosylation by pertussis toxin and function to inhibit adenylylcyclase (Borkovich MycotaIII). G$_{o}$ proteins are abundant in the brain and implicated in membrane trafficking. G$_{t}$, or transducin, is activated by rhodopsin, activates cGMP-phosphodiesterase (cGMP-PDE) and closes cGMP-gated sodium channels. G$_{z}$ activity is relatively unknown, but evidence suggests that it inhibits adenylylcyclase activity as well (Borkovich MycotaIII). The G$_{q}$ group contains members G$_{q}$, G$_{11}$, G$_{14}$, G$_{15}$, and G$_{16}$. These are widely expressed and involved in signal transduction through activation of phospholipase C-$\beta$1. (Borkovich MycotaIII). Finally, the G$_{12}$ group contains the G$_{12}$ and G$_{13}$ subunits. The function of these subunits is currently unknown, however mutations of a gene with high sequence similarity in drosophila causes a maternal effect (Borkovich MycotaIII).\\
\indent Motifs that describe the fungal group come from seven identified fungal G$\alpha$ proteins with no similarity to G$\alpha$ proteins of the previously identified groups. In \textit{S. cerevisiae}, the GP1 protein functions as a mating factor, while the GP2 protein functions in intracellular cAMP regulation. Members of the plant group may play a role in signal transduction from hormone receptors.\\
\indent Additionally, at least four $\beta$-subunits and six $\gamma$-subunits have been identified \cite{Hepler1992} in Fungi, and it has been demonstrated previously that the G$\beta$/$\gamma$ subunit is responsible for signal transduction in yeast \cite{Bolker1998}.\\
\indent Phosphodiesterases are a class of enzymes which cleave phosphodiester bonds. One specific group is the cyclic nucleotide phosphodiesterases, which act on cAMP and cGMP. Due to this activity, they are importance regulators of signal transduction. In mammalian systems, for example, they associate with activeated G$\alpha$ subunits in order to open CNG channels located in the plasma membrane and regulate the influx of Ca$^{2+}$ and Na$^{+}$.\\
%%-- Chapter methods --%%
\section{Methods}
\subsection{General Photosensory overview}
\subsection{G-protein Analysis}
\subsection{PLC}
\subsection{PDE}
\subsection{RGS}
%%-- Chapter results --%%
\section{Results}
\subsection{Photosensory}
\subsection{G-protein Analysis}
\subsubsection{G$\alpha$}
Multiple G$\alpha$ proteins were identified at an e-val < 1e$^{-20}$ in all surveyed chytrid genomes. Counts of predicted proteins are presented in Table~\ref{tab:ChRhodAux_Gprot}.\\
\indent One of the \textit{S. punctatus} G$\alpha$ proteins, SPPG05404, contains a C-terminal pertussis toxin sensitivity motif (C[GAVLIP]{2}X) and is 76.8\% identical to \textit{N.crassa} GNA-1 (NCU06493). Similarly, \textit{A. macrogynus} contains two predicted G$\alpha$ proteins, AMAG03583 and AMAG04903, both of which have the same motif. These proteins are approximately 70\% identical to \textit{N. crassa} GNA-1 (71.2\% and 69.6\%, respectively). Additionally, all of these proteins contain an N-terminal myristoylation motif (MGXXXS), consistent with members of the G$_{i}$ subfamily. \textit{R. allomycis}, possesses one protein with a pertussis toxin motif and has 69.77\% identity to \textit{N. crassa} GNA-1. \textit{Piromyces sp.} possesses two proteins (18092 and 48456) with pertussis motifs. Only the former, however, also posesses an N-terminal myristoylation motif. It is 73.7\% identical to \textit{N. crassa} GNA-1. The latter appears have a large portion of its N-terminus truncated relative to the former, and is more similar to \textit{N. crassa} GNA-3 than GNA-1 (65.7\% vs 55.2\% identity).\\
\indent Two proteins from \textit{B. dendrobatidis}, BDEG06989 and BDEG06990, have high similarity to GNA-1 at 66.3\% and 73.7\% identity, respectively. However, only BDEG06989 contains an N-terminal myristoylation motif. Neither of them contain the C-terminal pertussis motif.\\
\indent SPPG05884 from \textit{S. punctatus} has 75.6\% identity to \textit{N. crassa} GNA-1, but lacks the C-terminal pertussis motif (however it contains the N-terminal myristoylation motif).\\
\indent \textit{G. prolifera} posesses two predicted proteins with high (approx. 74\%) similarity to \textit{N. crassa} GNA-1. Both contain N-terminal myristoylation motifs, however only one also contains the C-terminal pertussis motif. A third predicted protein contains the myristoylation motif and is 62.6\% identical to \textit{N. crassa} GNA-3.\\
\indent The similarities of all identified chytrid G$\alpha$ proteins to identified \textit{N. crassa} G$\alpha$ proteins are described in Table~\ref{tab:gprot-percent}.\\
\indent A multiple sequence alignment, highlighting the pertussis and myristoylation motifs shared among all fungal G$\alpha$ proteins containing such motifs is presented in Figures~\ref{fig:ga_nt_msa} and~\ref{fig:ga_ct_msa}.\\
\indent A phylogenetic analysis was performed on all identified fungal G$\alpha$ proteins using RaxML. The majority of identified but uncharacterized chytrid G$\alpha$ proteins were most closely related to the Group IV family. This group remains largely uncharacterized, though the \textit{Ustilago maydis} homolog is induced during pathogenic development \cite{Bolker1998}. All other characterized G$\alpha$ groups (I, II, and III) contain chytrid members, unsurprisingly suggesting an ancient origin for these families. Of note, certain duplication events seem to have occurred after species divergence, and the fungal and mammalian G$\alpha$ proteins represent distinct groups.\\
\subsubsection{G$\beta$}
All surveyed chytrids were predicted to possess one or more G$\beta$ proteins. Counts and percent identity with \textit{N. crassa} GNB-1 are presented in Tables~\ref{tab:gprot} and~\ref{tab:gprot-percent}.
\subsubsection{G$\gamma$}
\textit{A. macrogynus}, \textit{B. dendrobatidis}, \textit{H. polyrhiza}, and \textit{Piromyces} were each predicted to possess a single G$\gamma$ protein (e-val < 1e$^{-20}$). Of these identified G$\gamma$ proteins, only that from \textit{B. dendrobatidis} contained a pertussis toxin sensitivity motif, with 51.4\% identity to NCU00041 (\textit{N. crassa} GNG-1). Counts and percent identity with \textit{N. crassa} GNG-1 are presented in Tables~\ref{tab:gprot} and~\ref{tab:gprot-percent}.
\subsection{Phototaxis and Heterologous Protein expression}
In order to determine the phototactic abilities of \textit{B. dendrobatidis} and \textit{S. punctatus}, I followed the procedure outlined in \cite{Muehlstein1987} for \textit{Rhizophydium littoreum}. Our setup consisted of a slide projector for a light source, and a mirror to redirect the light source.\\
%%-- Chapter conclusions --%%
\section{Conclusions}
