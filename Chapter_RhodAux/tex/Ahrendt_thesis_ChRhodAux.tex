%%%%%%%%%%%%%%%%%%%%%%%%%%%%%%%%%%%%%%%%%%%%%%%%%%%%%%%%%%%%%%%%%%%%%%%%%%%%%%%%%
%% Document: Thesis for PhD at UC Riverside                                    %%
%% Title: Investigating the evolution of environmental and biotic interactions %%
%%          in basal fungal lineages through comparative genomics              %%
%% Author: Steven Ahrendt                                                      %%
%%%%%%%%%%%%%%%%%%%%%%%%%%%%%%%%%%%%%%%%%%%%%%%%%%%%%%%%%%%%%%%%%%%%%%%%%%%%%%%%%
% RHODOPSIN AUXILLARY CHAPTER %
%%%%%%%%%%%%%%%%%%%%%%%%%%%%%%%
\chapter{Rhodopsin related signaling pathways in basal fungi}
\label{chap:RhodAux}
\section{Introduction}
\indent G-protein coupled receptors (GPCRs) are a broad class of seven-transmembrane proteins which receive extracellular signals and initiate an intracellular response \cite{Lagerstrom2008}. This process is accomplished via signal transduction pathways which link certain effector proteins with the receptors using heterotrimeric GTP-binding and hydrolysing proteins (G proteins) \cite{Hepler1992}. There are two major pathways which GPCRs are principally associated with: the cAMP signaling pathway and the phosphatidylinositol pathway \cite{Gilman1987}.\\
\indent As discussed in Chapter~\ref{chap:RhodStruct}, rhodopsins are a broad class of photosensitive seven-transmembrane proteins which respond to light through photoisomerization of a retinaldehyde chromophore, typically 11-\textit{cis}-retinal. Type 2 rhodopsins are GPCRs which function in, among other things, metazoan visual pathways.\\
\indent The heterotrimeric G proteins are coupled to the intracellular side of the transmembrane receptor. These GTPases are composed of an $\alpha$, a $\beta$, and a $\gamma$ subunit. The G$\alpha$ subunit is most often responsible for signal transduction. The G$\alpha$ subunit is bound to a molecule of GDP in its inactive state. Upon receptor stimulation, for example photoisomerization of 11-\textit{cis}-retinal in rhodopsin, the bound GDP is exchanged for GTP and the now active G$\alpha$ subunit dissociates from both the receptor and the G$\beta$/$\gamma$ subunit \cite{Neves2002}.\\ 
\indent G$\alpha$ proteins of different types interact with difference downstream effectors. Four major G$\alpha$ protein subfamilies have been identified: G$_{s}$, G$_{i}$, G$_{q}$, and G$_{12}$ \cite{Hepler1992}. Additional fungal \cite{Bolker1998} and plant \cite{Ma1994} subfamilies exist as well. The type of G-protein to which the receptor is coupled can help classify the Type 2 rhodopsins. The G$_{s}$ group contains the G$_{s}$ and G$_{olf}$  subunits, the latter being found in the olfactory neuroepithelial cells. These enzymes enhance the rate of cAMP synthesis by stimulating adenylate cyclase. Additionally, the G$_{s}\alpha$ subunit regulates Na$^{+}$ and Ca$^{2+}$ channels \cite{Hepler1992}. The G$_{i}$ group contains four subclasses: G$_{i}$, G$_{o}$, G$_{t}$, G$_{z}$. All G$_{i}$ subclasses possess a consensus sequence for ADP-ribosylation by pertussis toxin and function to inhibit adenylylcyclase \cite{KABMycotaIII}. G$_{o}$ proteins are abundant in the brain and implicated in membrane trafficking. G$_{t}$, or transducin, is activated by rhodopsin, activates cGMP-phosphodiesterase (cGMP-PDE) and closes cGMP-gated sodium channels. G$_{z}$ activity is relatively unknown, but evidence suggests that it inhibits adenylylcyclase activity as well \cite{KABMycotaIII}. The G$_{q}$ group contains members G$_{q}$, G$_{11}$, G$_{14}$, G$_{15}$, and G$_{16}$. These are widely expressed and involved in signal transduction through activation of phospholipase C-$\beta$1. \cite{KABMycotaIII}. Finally, the G$_{12}$ group contains the G$_{12}$ and G$_{13}$ subunits.\\
\indent Motifs that describe the fungal group come from seven identified fungal G$\alpha$ proteins with no similarity to G$\alpha$ proteins of the previously identified groups. In \textit{S. cerevisiae}, the GP1 protein functions as a mating factor, while the GP2 protein functions in intracellular cAMP regulation. Members of the plant group may play a role in signal transduction from hormone receptors.\\
\indent GTP-bound G$\alpha$ proteins are themselves acted on by a regulator of G-protein signaling (RGS) protein and the $\gamma$-subunit of cGMP phosphodiesterase.\\
\indent Phosphodiesterases (PDEs) are a class of enzymes which cleave phosphodiester bonds. One specific group is the cyclic nucleotide phosphodiesterases, which act on cAMP and cGMP. Due to this activity, they are importance regulators of signal transduction. In mammalian systems, for example, they associate with activeated G$\alpha$ subunits in order to close cGMP-gated cation channels located in the plasma membrane and regulate the influx of Ca$^{2+}$ and Na$^{+}$. One such interaction is the associateion between transducin (G$_{t}$) and cGMP phosphodiesterase in the rhodopsin visual signaling cascasde [citation needed].\\
\indent Additionally, at least four $\beta$-subunits and six $\gamma$-subunits have been identified \cite{Hepler1992} in Fungi, and it has been demonstrated previously that the G$\beta$/$\gamma$ subunit is responsible for signal transduction in yeast \cite{Bolker1998}.\\
\indent The work in this chapter deals with presence and absence of components of the rhodopsin signaling pathway in basal fungi, including the distribution of heterotrimeric G protein subunits and potential effector proteins.\\
%%-- Chapter methods --%%
\section{Methods}
\subsection*{General Photosensory overview}
\subsection*{G protein Analysis}
\subsection*{Effector Analysis}
\subsection*{Maintenance of chytrid cultures}
\textit{B. dendrobatidis} JEL423 cultures were grown on 1\% Tryptone plates [tryptone (10 g/L), glucose (3.2 g/L), and 1\% agar] and maintained at 23$^{\circ}$C. \textit{S. punctatus} SW-1 cultures were grown on PmTG agar plates [peptonized milk (0.5 g/L), tryptone (1 g/L), glucose (5 g/L), and 1\% agar]. All cultures were maintained at room temperature (23$^{\circ}$C) under an unregulated lighting scheme. Motile zoospores, were collected from actively growing (2-4 day old) plates by flooding with 2-4 mls of sterile di H$\_{2}$O, waiting 30-45 minutes, and collecting the liquid. \\
\subsection*{Phototaxis}
To examine the extent of phototaxis in \textit{B. dendrobatidis} and \textit{S. punctatus}, I followed the protocol outlined by Muehlstein et al.\nocite{Muehlstein1987} to observe phototaxis in the marine Chytridiomycete \textit{Rhizophydium littoreum}. Briefly, light was projected upwards through the bottom of a 60mm plastic petri dish containing a concentrated (approx. 10$^{6}$ cells/ml) suspension of freshly harvested, motile (approx. 75\%) zoospores. An "X" pattern is cut in a piece of dark cardboard which is placed between the light source and the petri dish. Phototaxis is determined by zoospore aggregation, avoidance, or ignorance of the "X"-shaped light source, and is scored as attraction, aversion, or absence, respectively. The light source was a Kodak slide projector with 300W white light bulb and was redirected using a stainless steel cosmetic mirror. The light source was placed at a distance such that a light intensity of X was achieved. Phototaxis experiments were carried out in a darkroom at 25$^{\circ}$C. Experimental setup is diagrammed in Figure~\ref{fig:ChRhodA_phototaxDiagram}.\\ 
\subsection*{\textit{Pichia pastoris} heterologous expression}
Genomic DNA was extracted using a modified bead-beating procedure. Briefly, approximately 100 mg of material containing both zoospores and sporangia was scraped from actively-growing, zoospore-rich B. dendrobatidis and S. punctatus plates. The material was added to approximately 100 mg of silicon beads \emph{[size]} and mixed with 600 $\mu$l of Cell lysis solution \emph{[Qiagen, cite]} and 3 $\mu$l of proteinase K \emph{[cite]}. The solution was homegenized with a bead beater using a 30s pulse at 4$^{\circ}$C and subsequently incubated for 2h at 55$^{\circ}$C. 200 $\mu$l of protein precipitation solution \emph{[Qiagen, cite]} was added to the mixture and iced for 15 min. After centrifugation at 14000xg for 3 min at room temperature, the supernatant was collected, mixed with 600 $\mu$l isopropanol, and spun at 1400xg for 1 min at room temperature. Pellet was washed with 600 $\mu$l ice cold 70\% EtOH and spun at 1400xg for 1 min at room temperature. Finally the pellet was air dried for 15 min at room temperature, resuspended in 50 $\mu$l H$_{2}$O, incubated at 65$^{\circ}$C for 1 hr, and stored at -20$^{\circ}$C.\\
\indent For plasmid construction, the \textit{B. dendrobatidis} rhodopsin gene was cloned into the pHIL-S1 \textit{P. pastoris} expression plasmid using a strategy described previously \cite{Bieszke1999}.  Modifications to the \textit{B. dendrobatidis} gene include the addition of 5' and 3' EcoRI restriction sites, as well as a C-terminal hexahistidine epitope. These modifications were accomplished using the following PCR primers: forward (5'-CGAGAATTCCATCCTGAATTTCTCATCACTCTG-3') and reverse (5'-CGAGAATTCTCAGTGGTGGTGGTGGTGGTGCAAGTGGTAGTTATGAAGAGGTTT-3'). PCR conditions were as follows: 10 $\mu$l 5X buffer, 27.5 $\mu$l H2O, 5 $\mu$l of each primer (10 $\mu$M), 1 $\mu$l of 10 mM dNTPs, 1 ng DNA, and 0.5 $\mu$l Phusion Polymerase in each 50 $\mu$l reaction. Cycle parameters used were 98$^{\circ}$C for 30s, 30 rounds of: 95$^{\circ}$C for 5s, 58$^{\circ}$C for 20s, 72$^{\circ}$C for 30s, and a final 72$^{\circ}$C for 10 min. The resulting fragment was purified using the Qiagen PCR purification kit, digested with EcoRI, purified again, and inserted into the EcoRI site of the pHIL-S1 plasmid. The resulting BdpHIL-S1 plasmids were transformed into chemically competent \textit{E. coli} JM109 cells. Transformants were checked for proper insertion / orientation by colony PCR.\\ 
\indent The S. punctatus gene was synthesized and inserted into the pHIL-S1 vector by GenScript (GenScript USA Inc. Piscataway, NJ 08854). Two versions were constructed: a wild type, which codes for the conserved lysine residue, and a mutant, which replaces the lysine residue with alanine.  \\
\indent The NoppHIL-S1 vector containing the Nop-1 protein previously described in \textit{Neurospora crassa} \cite{Bieszke1999} was generously loaned from Dr. Katherine Borkovich at the University of California, Riverside. The availability of sequences used for these analyses is provided in the appendix.\\
\indent For expression and membrane preparation, the BdpHIL-S1, SppHIL-S1, NoppHIL-S1, and pHIL-S1 vectors were cloned into \textit{P. pastoris} strain GS115 using the strategy described previously by Bieszke et al. \nocite{Bieszke1999} for expresion of Nop-1. The transformation vectors were linearized overnight with StuI. A 500 ml culture of \textit{P. pastoris} was grown in Medium A [yeast extract (10 g/L), proteose peptone (20 g/L), and dextrose (20 g/L)] shaking at 250rpm and at 30$^{\circ}$C until A$_{600}$ = 2.1. The culture was split into two 250 ml samples and spun at 1500g for 5 minutes at 4$^{\circ}$C. The pellets were resuspended in 250 ml ice-cold H$_{2}$O, spun again, resuspended in 100 ml ice-cold H$_{2}$O, spun again, and resuspended in 1 ml ice-cold 1M sorbitol. The cells were transferred to 1.5 ml microfuge tube on ice and used in transformation by electroporation. 80 $\mu$l of cells were added to 10 $\mu$l of linearized vector, iced for 5 minutes, and electroporated using 2 mm gap cuvettes and the following parameters: voltage gradient: 7.5 kV/cm, resistance: 600 $\Omega$, capacitance: 25 $\mu$F. Transformants were screened for integration of the plasmids by PCR. One of each transformant was grown on a large scale as described previously \cite{Bieszke1999}.\\
\indent Harvested cell pellets were washed in 1 pellet volume of ice-cold, sterile water and centrifuged at 1500xg for 5 min at 4$^{\circ}$C. The pellets were subsequently washed and resuspended in 1 pellet volume of Buffer A (7 mM NaH2PO4, pH 6.5, 7 mM EDTA, 7 mM dithiothreitol, 1 mM PMSF). Acid-washed 0.5mm glass beads were used to disrupt 1 ml aliquots using three 1-min pulses and two 90-s pulses with a minibead beater at 4$^{\circ}$C. The supernatants were collected and pooled to yield the cell lysate. The lysate was layered on a 70\% sucrose cushion (w/v in Buffer A) and centrifuged with no braking at 92000xg for 1 hour at 4$^{\circ}$C in a SW27 swinging bucket rotor.  The membrane layer, located at the interface between the lysate and sucrose cushion, was collected, stored at 4 $^{\circ}$C, and used for the membrane preparation.\\
\indent Immunoblot analysis on the membrane preparation was conducted using 50 $\mu$g of protein by PAGE. Mouse anti-6X His monoclonal antibody \emph{(citation)} was used at both 1:1000 and 1:3000 dilution for the primary antibody. Goat anti-mouse \emph{(citation)} was used as the secondary antibody at a 1:5000 dilution.\\
%%-- Chapter results --%%
\section{Results}
\subsection*{Photosensory}
In order to expand on previous reviews (eg \cite{Idnurm2010}) of the extent of photosensing in fungi, I searched for known photosensory proteins within proteomes (Table~\ref{tab:AppData_taxa}) of sequenced fungi from across the kingdom, with a focus on the recently sequenced basal lineages. This search included White-Collar complex proteins, Phytochromes, Cryptochromes, and opsins (Type 1 and 2). The results (Table~\ref{tab:ChRhodA_photosense}) are consistent with previous reviews but provided a higher level of resolution in the basal lineages, particularly the Chytridiomycota and Blastocladiomycota.\\
\indent It is worth noting here that the three chytridiomycete genomes surveyed have three different complements of opsin. In \textit{Spizellomyces punctatus} is found a member of the Type 2 subgroup, the structure of which is elaborated upon in Chapter~\ref{chap:RhodStruct}. This chytrid protein is the only known so far which posesses the critical lysine residue which is important for proper rhodopsin function \cite{Smith2010}. In \textit{Batrachochytrium dendrobatidis} however, we see no examples of opsins, or, for that matter, any other type of known photoreceptor protein. Yet the proteome of its closest relative, the non-pathogenic \textit{Homolaphlyctis polyrhiza}, contains an apparent opsin-GC fusion protein, the likes of which have only recently been described in the Blastocladiomycete \textit{Blastocladiella emersonii} \cite{Avelar2014}. This architecture is also observed in the three other Blastocladiomycete organisms, \textit{Allomyces macrogynus}, \textit{Catenaria anguillalae}, and the transcriptome of \textit{Coelomomyces lativittatus} (described in more detail in Chapter~\ref{chap:ClatTranscriptome}). Structural features of these opsin-GC fusion proteins are explored further in Chapter~\ref{chap:RhodStruct}.\\
\subsection*{G protein Analysis}
\subsubsection*{G$\alpha$}
Multiple G$\alpha$ proteins were identified at an e-val < 1e$^{-20}$ in all surveyed chytrid genomes. Counts of predicted proteins are presented in Table~\ref{tab:ChRhodAux_Gprot}.\\
\indent One of the \textit{S. punctatus} G$\alpha$ proteins, SPPG05404, contains a C-terminal pertussis toxin sensitivity motif (C[GAVLIP]{2}X) and is 76.8\% identical to \textit{N.crassa} GNA-1 (NCU06493). Similarly, \textit{A. macrogynus} contains two predicted G$\alpha$ proteins, AMAG03583 and AMAG04903, both of which have the same motif. These proteins are approximately 70\% identical to \textit{N. crassa} GNA-1 (71.2\% and 69.6\%, respectively). Additionally, all of these proteins contain an N-terminal myristoylation motif (MGXXXS), consistent with members of the G$_{i}$ subfamily. \textit{R. allomycis}, possesses one protein with a pertussis toxin motif and has 69.77\% identity to \textit{N. crassa} GNA-1. \textit{Piromyces sp.} possesses two proteins (18092 and 48456) with pertussis motifs. Only the former, however, also posesses an N-terminal myristoylation motif. It is 73.7\% identical to \textit{N. crassa} GNA-1. The latter appears have a large portion of its N-terminus truncated relative to the former, and is more similar to \textit{N. crassa} GNA-3 than GNA-1 (65.7\% vs 55.2\% identity).\\
\indent Two proteins from \textit{B. dendrobatidis}, BDEG06989 and BDEG06990, have high similarity to GNA-1 at 66.3\% and 73.7\% identity, respectively. However, only BDEG06989 contains an N-terminal myristoylation motif. Neither of them contain the C-terminal pertussis motif.\\
\indent SPPG05884 from \textit{S. punctatus} has 75.6\% identity to \textit{N. crassa} GNA-1, but lacks the C-terminal pertussis motif (however it contains the N-terminal myristoylation motif).\\
\indent \textit{G. prolifera} posesses two predicted proteins with high (approx. 74\%) similarity to \textit{N. crassa} GNA-1. Both contain N-terminal myristoylation motifs, however only one also contains the C-terminal pertussis motif. A third predicted protein contains the myristoylation motif and is 62.6\% identical to \textit{N. crassa} GNA-3.\\
\indent The similarities of all identified chytrid G$\alpha$ proteins to identified \textit{N. crassa} G$\alpha$ proteins are described in Table~\ref{tab:ChRhodAux_Gcomp}.\\
\indent A multiple sequence alignment, highlighting the pertussis and myristoylation motifs shared among all fungal G$\alpha$ proteins containing such motifs is presented in Figures~\ref{fig:ga_nt_msa} and~\ref{fig:ga_ct_msa}.\\
\indent A phylogenetic analysis was performed on all identified fungal G$\alpha$ proteins using RaxML. The majority of identified but uncharacterized chytrid G$\alpha$ proteins were most closely related to the Group IV family. This group remains largely uncharacterized, though there is evidence to suggest that the \textit{Ustilago maydis} homolog is induced during pathogenic development \cite{Bolker1998}. All other characterized G$\alpha$ groups (I, II, and III) contain chytrid members, unsurprisingly suggesting an ancient origin for these families. Of note, certain duplication events seem to have occurred after species divergence, and the fungal and mammalian G$\alpha$ proteins represent distinct groups.\\
\subsubsection*{G$\beta$}
All surveyed chytrids were predicted to possess one or more G$\beta$ proteins. Counts and percent identity with \textit{N. crassa} GNB-1 are presented in Tables~\ref{tab:ChRhodAux_Gprot} and~\ref{tab:ChRhodAux_Gcomp}.
\subsubsection*{G$\gamma$}
\textit{A. macrogynus}, \textit{B. dendrobatidis}, \textit{H. polyrhiza}, and \textit{Piromyces} were each predicted to possess a single G$\gamma$ protein (e-val < 1e$^{-20}$). Of these identified G$\gamma$ proteins, only that from \textit{B. dendrobatidis} contained a pertussis toxin sensitivity motif, with 51.4\% identity to NCU00041 (\textit{N. crassa} GNG-1). Counts and percent identity with \textit{N. crassa} GNG-1 are presented in Tables~\ref{tab:ChRhodAux_Gprot} and~\ref{tab:ChRhodAux_Gcomp}.
\subsection*{Phototaxis and Heterologous Protein expression}
In order to determine the phototactic abilities of \textit{B. dendrobatidis} and \textit{S. punctatus}, I followed the procedure outlined in \cite{Muehlstein1987} for \textit{Rhizophydium littoreum}. No phototaxis was observed in either \textit{B. dendrobatidis} or \textit{S. punctatus}. Furthermore, zoospores of \textit{H. polyrhiza} were not collected at sufficiently high quantities for meaningful phototaxis observation.\\
\indent Heterologous protein expression was also not observed using either BdpHIL-S1 or SppHIL-S1, despite observation of NoppHIL-S1 expression.\\ 
%%-- Chapter conclusions --%%
\section{Discussion}
The goals for the research in this chapter were to assess, in the basal fungi, the complement of proteins which are secondarily involved in the rhodopsin-mediated photosignaling cascade (ie proteins other than the rhodopsin GCPR protein, including the intermediate heterotrimeric G proteins, and the effectors involved in the cAMP signaling and phosphotidylinositol pathways. From a functional perspective, this comparative study will demonstrate which of the basal fungi have a complete pathway, and would be helpful in understanding at which point in evolution these components were lost.\\
\indent \\
