%%%%%%%%%%%%%%%%%%%%%%%%%%%%%%%%%%%%%%%%%%%%%%%%%%%%%%%%%%%%%%%%%%%%%%%%%%%%%%%%%
%% Document: Thesis for PhD at UC Riverside                                    %%
%% Title: Investigating the evolution of environmental and biotic interactions %%
%%          in basal fungal lineages through comparative genomics              %%
%% Author: Steven Ahrendt                                                      %%
%%%%%%%%%%%%%%%%%%%%%%%%%%%%%%%%%%%%%%%%%%%%%%%%%%%%%%%%%%%%%%%%%%%%%%%%%%%%%%%%%
%% RHODOPSIN AUXILLARY CHAPTER %%
%%%%%%%%%%%%%%%%%%%%%%%%%%%%%%%%%
\chapter{Rhodopsin related signaling pathways in basal fungi}
\label{chap:RhodAux}
%%%%%%%%%%%%%%%%%%
%% Introduction %%
%%%%%%%%%%%%%%%%%%
\section{Introduction}
\indent G-protein coupled receptors (GPCRs) are a broad class of seven-transmembrane 
proteins which receive extracellular signals and initiate an intracellular response 
\cite{Lagerstrom2008}. This process is accomplished via signal transduction pathways 
which link certain effector proteins with the receptors using heterotrimeric GTP-binding 
and hydrolysing proteins (G proteins) \cite{Hepler1992}. There are two major pathways 
which GPCRs are principally associated with: the cAMP signaling pathway and the phosphatidylinositol pathway \cite{Gilman1987}.\\
\indent G protein complexes associate with a transmembrane GPCR and are loosely coupled to 
the intracellular side of the plasma membrane \cite{Clapham1997}.
These GTPases are composed of an $\alpha$, a $\beta$, and a $\gamma$ subunit. While the G$\alpha$ subunit 
was historically thought to be solely responsible for signal transduction \cite{Gilman1987}, 
it has been demonstrated that the G$\beta$/$\gamma$ subunit is capable of producing a 
response in yeast \cite{Clark1993}. Currently the number of effectors regulated by one 
or both subunits ($\alpha$ or $\beta$/$\gamma$) is equivalent \cite{Clapham1997}.\\ 
\indent The G$\alpha$ subunit is bound to a molecule of GDP in its inactive state. 
Upon receptor stimulation, for example photoisomerization of 11-\textit{cis}-retinal 
in rhodopsin, the bound GDP is exchanged for GTP and the now active G$\alpha$ subunit 
dissociates from both the receptor and the G$\beta$/$\gamma$ subunit \cite{Neves2002}.\\ 
\indent G$\alpha$ proteins of different types interact with difference downstream effectors. Four 
major G$\alpha$ protein subfamilies have been identified: G$_{s}$, G$_{i}$, G$_{q}$, and G$_{12}$ 
\cite{Hepler1992}. The type of G-protein to which the receptor is coupled can help classify the Type 2 
rhodopsins. The G$_{s}$ group contains the G$_{s}$ and G$_{olf}$  subunits, the latter being found 
in the olfactory neuroepithelial cells. These enzymes enhance the rate of cAMP synthesis by 
stimulating adenylate cyclase. Additionally, the G$_{s}\alpha$ subunit regulates Na$^{+}$ and 
Ca$^{2+}$ channels \cite{Hepler1992}. The G$_{i}$ group contains four subclasses: G$_{i}$, 
G$_{o}$, G$_{t}$, G$_{z}$. All G$_{i}$ subclasses possess a consensus sequence for 
ADP-ribosylation by pertussis toxin and function to inhibit adenylylcyclase \cite{KABMycotaIII}. 
G$_{o}$ proteins are abundant in the brain and implicated in membrane trafficking. 
G$_{t}$, or transducin, is activated by rhodopsin, activates cGMP-phosphodiesterase (cGMP-PDE) 
and closes cGMP-gated sodium channels. G$_{z}$ activity is relatively unknown, 
but evidence suggests that it inhibits adenylylcyclase activity as well \cite{KABMycotaIII}. 
The G$_{q}$ group contains members G$_{q}$, G$_{11}$, G$_{14}$, G$_{15}$, and G$_{16}$. 
These are widely expressed and involved in signal transduction through activation of phospholipase C-$\beta$1 \cite{KABMycotaIII}. 
Finally, the G$_{12}$ group contains the G$_{12}$ and G$_{13}$ subunits.\\
\indent Four distinct subgroups have been previously identified from fungal G$\alpha$ proteins \cite{Bolker1998}. Only Group I and III proteins have discernible similarity to mammalian families: the inhibitory (G$\alpha$i) and stimulatory (G$\alpha$s) families, respectively. Group I proteins contain amino-terminal myristoylation consensus sequences (MGxxxS) as well as carboxy-terminal pertussis toxin ADP-ribosylation sites (C[GAVLIP]{2}x), both of which are conserved in the G$\alpha$i superfamily \cite{Li2007}. Group III proteins are highly conserved and also possess a myristoylation motif at their amino-termini. While there is evidence that many Group III proteins influence cAMP levels, suggesting placement in the G$\alpha$s family, GNA-3 in \textit{N. crassa} is much more similar to G$\alpha$i2 \cite{Li2007}. Group II G$\alpha$ proteins have no homologous mammalian counterpart, and a biological function has only been observed for a few members. In \textit{S. cerevisiae}, the Gpa1p protein functions as a mating factor, while the Gpa2p protein functions in intracellular cAMP regulation \cite{KABMycotaIII}. The \textit{magC} gene from \textit{Magnaporthe grisea} has been demonstrated to be involved in ascospore development \cite{Liu1997}. The fourth and arguably minor G$\alpha$ subgroup has little to no identified function, and was initially characterized as one unusual protein identified in \textit{Ustilago maydis} which clustered separately from other fungal G$\alpha$ proteins and for deletion of which there was no obvious phenotype \cite{Bolker1998}.\\
\indent GTP-bound G$\alpha$ proteins are themselves acted on by a regulator of G protein signaling (RGS) protein \cite{DeVries2000}. RGS proteins are responsible for the rapid return of active GTP-bound G$\alpha$-subunits to their inactive state, and function to tightly regulate the signal generated by a GPCR. \\
\indent Activated G$\alpha$ proteins also interact with the $\gamma$-subunit of cGMP phosphodiesterase. Phosphodiesterases (PDEs) are a superfamily of proteins which enzymatically cleave phosphodiester bonds. In mammals, this superfamily is comprised of 12 families based on sequence similarity, tissue distribution, and substrate specificity. PDEs act on cAMP (PDE families 4, 7, and 8), cGMP (PDE families 5, 6, and 9), or both (PDE families 1, 2, 3, 10, and 11) \cite{Conti2000}. Due to this activity, they are importance regulators of signal transduction. In mammalian systems, for example, they associate with activeated G$\alpha$ subunits in order to close cGMP-gated cation channels located in the plasma membrane and regulate the influx of Ca$^{2+}$ and Na$^{+}$. One such interaction is the association between transducin (G$_{t}$) and cGMP phosphodiesterase in the rhodopsin visual signaling cascasde \cite{Deterre1988}. The transducin $\alpha$ subunit (T$\alpha$), activated by rhodopsin interaction with light, interacts with the PDE complex to disrupt the inactivation caused by the PDE$\gamma$ subunit, resulting in hyperpolarization of the cell membrane \cite{Deterre1988}.\\
\indent The G$\beta$/$\gamma$ subunit is responsible for a number of regulatory functions during signal transduction \cite{Clapham1997}. Previous predictions suggest that Dikarya fungi have one G$\beta$ subunit, while the zygomycete \textit{Rhizopus oryzae} has four \cite{Li2007}. Structurally, the G$\beta$ subunit contains a 20aa $\alpha$-helix region and a large domain composed of repeated WD40 motifs. While the WD-repeat region is unknown, it is presumed to be related to assembly of the complex. \cite{Clapham1997}.\\
\indent All G$\gamma$ subunits posess specific CaaX motifs at their C-termini. These motifs are subject to 
postranslational modification, the nature of which both targets the G$\beta$/$\gamma$ complex to 
the plasma membrane, and governs interactions between G$\gamma$ and G$\alpha$, receptors, and/or 
effector proteins \cite{Krystofova2005}.\\
\indent As discussed in Chapter~\ref{chap:RhodStruct}, rhodopsins are a broad class of photosensitive 
seven-transmembrane proteins which respond to light through photoisomerization of a retinaldehyde chromophore, 
typically 11-\textit{cis}-retinal. Type 2 rhodopsins are GPCRs which function in, among other things, metazoan visual pathways.\\
\indent The work in this chapter deals with presence and absence of components of the 
rhodopsin signaling pathway in basal fungi, including the distribution of heterotrimeric 
G protein subunits and potential effector proteins.\\

%%%%%%%%%%%%%
%% Methods %%
%%%%%%%%%%%%%
\section{Methods}

%---- Photosensory search ----%
\subsection*{Identification of homologous photosensory proteins}
To get a sense for the distribution of photosensory proteins in fungi, the fungal proteomes listed in Table~\ref{tab:AppData_taxa} were searched using the classes of photosensory proteins from previous analyses described in by Idnurm et al. The white collar protein complex, comprising WC-1 and 2, were searched for using the \textit{N. crassa} proteins NCU02356 and NCU00902, respectively, as queries using $ssearch36$ from the FASTA package \cite{Pearson1988} with an e-val cutoff of 1e-10. Type 1 and Type 2 opsins were searched for using HMMER (v3.0) $hmmsearch$ \cite{Eddy2011} with HMM models PF01036 and PF00001 (obtained from the Pfam database \cite{Finn2014}), respectively, using an e-val cutoff of 1e-20. Cryptochrome homologs were searched using HMMER (v3.0) $hmmsearch$ with an HMM generated from the 20 seed sequences in the Pfam domain PF12546 (Cryptochrome\_C). Hits were retained above a threshold of 1e-20. Similarly, phytochrome homologs were searched using $hmmsearch$ with an HMM profile generated from the 80 seed sequences in PF00360 (PHY). Hits were retained above a threshold of 1e-20.\\

%---- G protein subunits and effectors ----%
\subsection*{G protein analysis}
Homologs of G$\alpha$ proteins were identified using HMMER $hmmsearch$ with an HMM profile generated from the seed set of Pfam domain family PF00503. Fungal hits above a threshold of e-20 were kept for subsequent analysis. A maximum likelihood tree was constructed using RAxML (v7.5.4) \cite{Stamatakis2014} with 100 bootstrap replicates using representative Fungal hits from basal fungi, Zygomycete, Ascomycete, and Basidiomycete lineages, along with representative outgroup eukaryotic G$\alpha$ sequences from the G$_{s}$ (IPR000367), G$_{q}$ (IPR000654), G$_{i}$ (IPR001408), and G$_{12}$ (IPR000469) families.\\
\indent Homologs of G$\beta$ proteins were identified using HMMER $hmmsearch$ with an HMM profile generated from the seed set of PRINTS domain family PR00319. Fungal hits above a threshold of e-100 were kept for subsequent analysis. A maximum likelihood tree was constructed using RAxML (v7.5.4) with 100 boostrap replicates using fungal hits from basal lineages, Zyogomycete, Ascomycete, and Basidiomycete lineages, along with representative sequences for Gnb-1, Gnb-2, Gnb-3, Gnb-4, and Gnb-5 from mouse and human. Also included were other outgroup sequences from \textit{Phytophthora sojae}, \textit{Chlamydomonas reinhardtii}, and \textit{Drosphila melanogaster}.\\
\indent Homologs of G$\gamma$ proteins were identified using HMMER $hmmsearch$ with an HMM profile built using $hmmbuild$ from 43 sequences matching the InterPro IPR001770 domain: 1 Choanoflagellate, 2 Porifera, 1 Placozoa, 1 Ctenophora, 6 Cnidaria, 1 Nematoda, 6 Arthropoda, and 25 Chordata. Fungal hits above a threshold of e-5 were kept for phylogenetic analysis, along with a representative set from the original IPR001770 dataset: 1 Choanoflagelatte, 12 Chordata, 1 Placozoa, and 2 Porifera. \\
\indent Homologs of RGS proteins were identified using HMMER $hmmsearch$ with an HMM profile genearated from the seed set of Pfam domain family PF00615.\\
\indent Multiple sequence alignments were drawn and annotated using the \TeXshade package \cite{Beitz2000texshade}, and protein domain figures were drawn using the pgfmolbio package (\url{http://www.ctan.org/pkg/pgfmolbio}).\\
\subsection*{Phosphodiesterase analysis}
\indent Homologs of phospohodiesterase (PDE) subunits were identified using sequences from KEGG families K08718 ($\alpha$), K13756 ($\beta$), and K13759 ($\gamma$). Each KEGG family contained a set of phylogenetically diverse sequences from which HMM models were built using T-coffee (v8.97\_101117) \cite{Notredame2000} and HMMER. K08718 (PDE$\alpha$) contained 18 chordata sequences, K13756 (PDE$\beta$) contained 24 chordata sequences, and K13759 (PDE$\gamma$) contained 25 chordata sequences. These sequences correspond to PDE-6, a member of the cGMP-specific family of PDEs.\\

%---- Phototaxis and heterologous expression ----%
\subsection*{Maintenance of chytrid cultures}
\textit{B. dendrobatidis} JEL423 cultures were grown on 1\% Tryptone plates [tryptone (10 g/L), glucose (3.2 g/L), and 1\% agar] and maintained at 23$^{\circ}$C. \textit{S. punctatus} SW-1 cultures were grown on PmTG agar plates [peptonized milk (0.5 g/L), tryptone (1 g/L), glucose (5 g/L), and 1\% agar]. All cultures were maintained at room temperature (23$^{\circ}$C) under an unregulated lighting scheme. Motile zoospores, were collected from actively growing (2-4 day old) plates by flooding with 2-4 mls of sterile di H$\_{2}$O, waiting 30-45 minutes, and collecting the liquid. \\
\subsection*{Phototaxis}
To examine the extent of phototaxis in \textit{B. dendrobatidis} and \textit{S. punctatus}, I followed a protocol established previously to observe phototaxis in the marine Chytridiomycete \textit{Rhizophydium littoreum} \cite{Muehlstein1987}. Briefly, light was projected upwards through the bottom of a 60mm plastic petri dish containing a concentrated (approx. 10$^{6}$ cells/ml) suspension of freshly harvested, motile (approx. 75\%) zoospores. An "X" pattern is cut in a piece of dark cardboard which is placed between the light source and the petri dish. Phototaxis is determined by zoospore aggregation, avoidance, or ignorance of the "X"-shaped light source, and is scored as attraction, aversion, or absence, respectively. The light source was a Kodak slide projector with 300W white light bulb and was redirected using a stainless steel cosmetic mirror. The light source was placed at a distance such that a light intensity of 950 lux was achieved. Phototaxis experiments were carried out in a darkroom at 25$^{\circ}$C.\\ 
\subsection*{\textit{Pichia pastoris} heterologous expression}
Genomic DNA was extracted using a modified bead-beating procedure. Briefly, approximately 100 mg of material containing both zoospores and sporangia was scraped from actively-growing, zoospore-rich \textit{B. dendrobatidis} and \textit{S. punctatus} plates. The material was added to approximately 100 mg of silicon beads (0.5mm dia.) and mixed with 600 $\mu$l of Cell lysis solution (Qiagen, Germantown, MD) and 3 $\mu$l of proteinase K. The solution was homegenized with a bead beater using a 30s pulse at 4$^{\circ}$C and subsequently incubated for 2h at 55$^{\circ}$C. 200 $\mu$l of protein precipitation solution (Qiagen, Germantown, MD) was added to the mixture and iced for 15 min. After centrifugation at 14000xg for 3 min at room temperature, the supernatant was collected, mixed with 600 $\mu$l isopropanol, and spun at 1400xg for 1 min at room temperature. Pellet was washed with 600 $\mu$l ice cold 70\% EtOH and spun at 1400xg for 1 min at room temperature. Finally the pellet was air dried for 15 min at room temperature, resuspended in 50 $\mu$l H$_{2}$O, incubated at 65$^{\circ}$C for 1 hr, and stored at -20$^{\circ}$C.\\
\indent For plasmid construction, a sequence comprising the \textit{B. dendrobatidis} rhodopsin gene (BDEG\_04847), along with an additional 78 upstream bases to account for the first transmembrane helix, was amplified from \textit{B. dendrobatidis} JEL423 and cloned into the pHIL-S1 \textit{P. pastoris} expression plasmid using a strategy described previously \cite{Bieszke1999}.  Modifications to the \textit{B. dendrobatidis} gene include the addition of 5' and 3' EcoRI restriction sites, as well as a C-terminal hexahistidine epitope. These modifications were accomplished using the following PCR primers (provided in Appendix B): "$Bden\_EcoRI\_F$" (forward) and "$Bden\_EcoRI\_His\_R$" (reverse). PCR conditions were as follows: 10 $\mu$l 5X buffer, 27.5 $\mu$l H2O, 5 $\mu$l of each primer (10 $\mu$M), 1 $\mu$l of 10 mM dNTPs, 1 ng DNA, and 0.5 $\mu$l Phusion Polymerase in each 50 $\mu$l reaction. Cycle parameters used were 98$^{\circ}$C for 30s, 30 rounds of: 95$^{\circ}$C for 5s, 58$^{\circ}$C for 20s, 72$^{\circ}$C for 30s, and a final 72$^{\circ}$C for 10 min. The resulting fragment was purified using the Qiagen PCR purification kit, digested with EcoRI, purified again, and inserted into the EcoRI site of the pHIL-S1 plasmid. The resulting BdpHIL-S1 plasmids were transformed into chemically competent \textit{E. coli} JM109 cells. Transformants were checked for proper insertion / orientation by colony PCR.\\ 
\indent The \textit{S. punctatus} gene (SPPG\_00350) was synthesized and inserted into the pHIL-S1 vector by GenScript (GenScript USA Inc. Piscataway, NJ 08854). Two versions were constructed: a wild type, which codes for the conserved lysine residue, and a mutant, which replaces the lysine residue with alanine.  \\
\indent The NoppHIL-S1 vector containing the Nop-1 protein previously described in \textit{Neurospora crassa} \cite{Bieszke1999} was generously loaned from Dr. Katherine Borkovich at the University of California, Riverside. The availability of sequences used for these analyses is provided in the appendix.\\
\indent For expression and membrane preparation, the BdpHIL-S1, SppHIL-S1, NoppHIL-S1, and pHIL-S1 vectors were cloned into \textit{P. pastoris} strain GS115 using the strategy described previously by Bieszke et al. \nocite{Bieszke1999} for expression of Nop-1. The transformation vectors were linearized overnight with StuI. A 500 ml culture of \textit{P. pastoris} was grown in Medium A [yeast extract (10 g/L), proteose peptone (20 g/L), and dextrose (20 g/L)] shaking at 250rpm and at 30$^{\circ}$C until A$_{600}$ = 2.1. The culture was split into two 250 ml samples and spun at 1500g for 5 minutes at 4$^{\circ}$C. The pellets were resuspended in 250 ml ice-cold H$_{2}$O, spun again, resuspended in 100 ml ice-cold H$_{2}$O, spun again, and resuspended in 1 ml ice-cold 1M sorbitol. The cells were transferred to 1.5 ml microfuge tube on ice and used in transformation by electroporation. 80 $\mu$l of cells were added to 10 $\mu$l of linearized vector, iced for 5 minutes, and electroporated using 2 mm gap cuvettes and the following parameters: voltage gradient: 7.5 kV/cm, resistance: 600 $\Omega$, capacitance: 25 $\mu$F. Transformants were screened for integration of the plasmids by PCR. One of each transformant was grown on a large scale as described previously \cite{Bieszke1999}.\\
\indent Harvested cell pellets were washed in 1 pellet volume of ice-cold, sterile water and centrifuged at 1500xg for 5 min at 4$^{\circ}$C. The pellets were subsequently washed and resuspended in 1 pellet volume of Buffer A (7 mM NaH2PO4, pH 6.5, 7 mM EDTA, 7 mM dithiothreitol, 1 mM PMSF). Acid-washed 0.5mm glass beads were used to disrupt 1 ml aliquots using three 1-min pulses and two 90-s pulses with a minibead beater at 4$^{\circ}$C. The supernatants were collected and pooled to yield the cell lysate. The lysate was layered on a 70\% sucrose cushion (w/v in Buffer A) and centrifuged with no braking at 92000xg for 1 hour at 4$^{\circ}$C in a SW27 swinging bucket rotor. The membrane layer, located at the interface between the lysate and sucrose cushion, was collected, stored at 4 $^{\circ}$C, and used for the membrane preparation.\\
\indent Immunoblot analysis on the membrane preparation was conducted using 50 $\mu$g of protein by PAGE. Mouse anti-6X His monoclonal antibody (Fisher Scientific, Pittsburgh, PA) was used at both 1:1000 and 1:3000 dilution for the primary antibody. Goat anti-mouse (Bio-Rad, Hercules, CA) was used as the secondary antibody at a 1:5000 dilution.\\

%%%%%%%%%%%%%
%% Results %%
%%%%%%%%%%%%%
\section{Results}

%---- Photosensory ----%
\subsection*{Photosensory}
In order to expand on previous summaries (eg \cite{Idnurm2010}) of the extent of photosensing in fungi, I searched for known photosensory proteins within proteomes (Table~\ref{tab:AppData_taxa}) of sequenced fungi from across the kingdom, with a focus on the recently sequenced basal lineages. This search included White-Collar complex proteins, Phytochromes, Cryptochromes, and opsins (Type 1 and 2). The results (Figure~\ref{fig:ChRhodA_photosenseSurvey}) are consistent with previous reviews but provided a higher level of resolution in the basal lineages, particularly the Chytridiomycota and Blastocladiomycota.\\
\indent It is worth noting here that the basal fungal genomes surveyed have different capacities for photosensing. In \textit{Spizellomyces punctatus}, both a Type 2 rhodopsin and White collar complex members are identified, the structure of the former of which is elaborated upon in Chapter~\ref{chap:RhodStruct}. This rhodopsin protein is the only known so far in chytrids which posesses the critical lysine residue important for proper rhodopsin function \cite{Smith2010}. In \textit{Batrachochytrium dendrobatidis} however, the only photosensory protein identified is the Type 2 rhodopsin, which does not appear to posess the critical lysine residue. Yet the proteome of its closest relative, the non-pathogenic \textit{Homolaphlyctis polyrhiza}, contains an apparent opsin-GC fusion protein, the likes of which have only recently been described in the Blastocladiomycete \textit{Blastocladiella emersonii} \cite{Avelar2014}. This architecture is also observed in the three other Blastocladiomycete organisms, \textit{Allomyces macrogynus}, \textit{Catenaria anguillalae}, and the transcriptome of \textit{Coelomomyces lativittatus} (described in more detail in Chapter~\ref{chap:ClatTranscriptome}). Structural features of these opsin-GC fusion proteins are explored further in Chapter~\ref{chap:RhodStruct}.\\

%---- G alpha ----%
\subsection*{G protein analysis}
Multiple G$\alpha$ proteins were identified at an e-val < 1e$^{-20}$ in all surveyed chytrid genomes. Counts of predicted proteins are presented in Figure~\ref{fig:ChRhodA_photosenseSurvey}.\\
\indent One of the \textit{S. punctatus} G$\alpha$ proteins, SPPG\_05404, contains a C-terminal pertussis toxin sensitivity motif (C[GAVLIP]{2}X) and is 76.8\% identical to \textit{N.crassa} GNA-1 (NCU06493). Similarly, \textit{A. macrogynus} contains two predicted G$\alpha$ proteins, AMAG03583 and AMAG04903, both of which have the same motif. These proteins are approximately 70\% identical to \textit{N. crassa} GNA-1 (71.2\% and 69.6\%, respectively). Additionally, all of these proteins contain an N-terminal myristoylation motif (MGXXXS), consistent with members of the G$_{i}$ subfamily. \textit{R. allomycis}, possesses one protein with a pertussis toxin motif and has 69.77\% identity to \textit{N. crassa} GNA-1. \textit{Piromyces sp.} possesses two proteins (18092 and 48456) with pertussis motifs. Only the former, however, also posesses an N-terminal myristoylation motif. It is 73.7\% identical to \textit{N. crassa} GNA-1. The latter appears have a large portion of its N-terminus truncated relative to the former, and is more similar to \textit{N. crassa} GNA-3 than GNA-1 (65.7\% vs 55.2\% identity).\\
\indent Two proteins from \textit{B. dendrobatidis}, BDET\_07008 and BDEG\_07009, have high similarity to GNA-1 at 66.29\% and 70.68\% identity, respectively. However, only BDET\_07008 contains an N-terminal myristoylation motif. Neither of them contain the C-terminal pertussis motif.\\
\indent SPPG\_05884 from \textit{S. punctatus} has 75.6\% identity to \textit{N. crassa} GNA-1, but lacks the C-terminal pertussis motif (however it contains the N-terminal myristoylation motif). Similarly, SPPG\_01130 contains the N-terminal myristoylation motif and is 66.85\% identity to GNA-3.\\
\indent \textit{G. prolifera} posesses two predicted proteins with high (approx. 74\%) similarity to \textit{N. crassa} GNA-1. Both contain N-terminal myristoylation motifs, however only one also contains the C-terminal pertussis motif. A third predicted protein contains the myristoylation motif and is 62.6\% identical to \textit{N. crassa} GNA-3.\\
\indent The similarities of all identified chytrid G$\alpha$ proteins to identified \textit{N. crassa} G$\alpha$ proteins are presented in Table~\ref{tab:ChRhodA_GAcomp}, and a multiple sequence alignment, highlighting the pertussis and myristoylation motifs shared among all fungal G$\alpha$ proteins containing such motifs is presented in Figure~\ref{fig:ChRhodA_gaMSA}.\\
\indent A phylogenetic analysis was performed on all identified fungal G$\alpha$ proteins using RaxML (Figures~\ref{fig:ChRhodA_GalphaTree1}, \ref{fig:ChRhodA_GalphaTree2}, \ref{fig:ChRhodA_GalphaTree3}, and \ref{fig:ChRhodA_GalphaTree4}). All characterized G$\alpha$ groups contain chytrid members, suggesting an ancient origin for these families. Additionally, the fungal G$\alpha$ proteins cluster separately from the metazoan proteins. The Group II G$\alpha$ family contains only a single chytrid sequence from \textit{R. allomycis}. This protein is 360aa long and has the highest similarity to the G$\alpha$-9 subunit from \textit{Dictyostelium discoideum} (Uniprot: Q54R41.1; e-val 4e-52; 32\% identity). The majority of identified chytrid G$\alpha$ proteins were most closely associated with the Group IV family. This group remains largely uncharacterized, though there is evidence to suggest that the \textit{Ustilago maydis} homolog is induced during pathogenic development \cite{Bolker1998}.\\ 
%---- G beta ----%
\indent All surveyed chytrids were predicted to possess one or more G$\beta$ proteins. Counts are presented in Tables~\ref{fig:ChRhodA_photosenseSurvey}, and similarities to \textit{N. crassa} GNB-1 are presented in Table~\ref{tab:ChRhodA_GBcomp}. A phylogenetic tree is given in Figure~\ref{fig:ChRhodA_GbetaTree} illustrating sequences from basal lineages clustering among each other. All identified fungal sequences possess multiple WD40 repeat domains typical of G$\beta$ proteins identified with InterproScan (Figure~\ref{fig:ChRhodA_GbetaStruct}). \\
%---- G gamma ----%
\indent \textit{A. macrogynus}, \textit{B. dendrobatidis}, \textit{C. anguillulae}, \textit{H. polyrhiza}, \textit{Orpinomyces}, \textit{Piromyces}, and \textit{R. allomycis} were each predicted to possess a single G$\gamma$ protein (e-val < 1e-5). No G$\gamma$ subunits were predicted in \textit{S. punctatus} or from the \textit{C. lativittatus} transcriptome. Of these identified G$\gamma$ proteins, only those from \textit{B. dendrobatidis} and \textit{Orpinomyces} contained a classic pertussis toxin sensitivity motif of the form (C[GAVLIP]\{2\}X), with 51.4\% and 35.0\% identity to NCU00041 (\textit{N. crassa} GNG-1). Counts are presented in Figure~\ref{fig:ChRhodA_photosenseSurvey}, and similarities to \textit{N. crassa} GNG-1 are presented in Table~\ref{tab:ChRhodA_GGcomp}. Multiple sequence alignment of fungal sequences recovered which possess the pertussis motif is provided in Figure~\ref{fig:ChRhodA_ggMSA}. A maximum likelihood phylogenetic tree of recovered fungal sequences and representative metazoan G$\gamma$ sequences places the basal proteins expectedly at the base of the fungal group, which forms a distinct cluster from the metazoan outgroup sequences.\\
%---- RGS ----%
\indent In chytrids, RGS homologs were predicted in \textit{R. allomycis}, \textit{S. punctatus}, and \textit{B. dendrobatidis}. The two \textit{Sp} proteins shared 22.1\% and 18.8\% identity with the \textit{N. crassa} RGS protein NCU08319 \cite{Borkovich2004}, while the \textit{Bd} and \textit{R. allomycis} proteins shared 17\% and 13.4\% identity, respectively, with NCU08319. One of the \textit{Sp} proteins (SPPG07577) contained a G$\gamma$ binding region motif (GGL domain: PF00631), while SPPG04061 did not. The \textit{Bd} protein (BDEG00728) contained the GGL domain as well.\\

%---- Effector proteins ----%
\subsection*{Phosphodiesterase proteins}
A search of phosphodiesterase protein complement, associated with G protein signaling pathways, is presented in 
Figure~\ref{fig:ChRhodA_photosenseSurvey}). PDE $\alpha$ and $\beta$ subunit homologs were recovered 
in the basal lineages. In most cases, the recovered protein had significant similarity to both $\alpha$ and $\beta$ queries, which belong
to the cGMP-specific PDE-6 subfamily. A maximum likelihood phylogenetic analysis (Figure~\ref{fig:ChRhodA_PDEtree}) places recovered fungal proteins as distinct from metazoan lineages, with the exception of one sequence in \textit{Hp}.\\ 

%---- Phototaxis ----%
\subsection*{Phototaxis and Heterologous Protein expression}
In order to determine the phototactic abilities of \textit{B. dendrobatidis} and \textit{S. punctatus}, I followed the procedure outlined in \cite{Muehlstein1987} for \textit{Rhizophydium littoreum}. However, no phototaxis was observed in either \textit{B. dendrobatidis} or \textit{S. punctatus}. Furthermore, zoospores of \textit{H. polyrhiza} were not collected at sufficiently high quantities for meaningful phototaxis observation.\\
\indent Heterologous protein expression of the chytrid opsin proteins was not observed using either BdpHIL-S1 or SppHIL-S1 vectors, despite successful expression using NoppHIL-S1 (Figure~\ref{fig:ChRhodA_PichiaBlot}).\\ 

%%%%%%%%%%%%%%%%
%% Discussion %%
%%%%%%%%%%%%%%%%
\section{Discussion}
The goals for the research in this chapter were to assess, in the basal fungi, the complement of proteins which are secondarily involved in the rhodopsin-mediated photosignaling cascade (ie proteins other than the rhodopsin GCPR protein, including the intermediate heterotrimeric G proteins, and the effectors involved in the cAMP signaling and phosphotidylinositol pathways. From a functional perspective, this comparative study will demonstrate which of the basal fungi have a complete pathway, and would be helpful in understanding at which point in evolution these components were lost.\\
\indent G$\alpha$ subunits posessing N-terminal myristoylation and C-terminal pertussis motifs are members of the G$_{i}$ subfamily. Transducin is one member of this family, and is known to associate with rhodopsin to function to activate cGMP-PDE. \textit{S. punctatus} and \textit{A. macrogynus} posess G$\alpha$ proteins which contain these motifs and which have high similarity to \textit{N. crassa} Gna-1. In \textit{N. crassa}, \emph{$\Delta$gna-1} strains are deficient in multiple pathways during both vegetative and sexual development, one of which (macroconidiation and mass accumulation) seems to be related to photosensing \cite{Ivey1996}.\\
\indent The G$\alpha$ group in fungi can be classified into four families \cite{Bolker1998}. Chytrid G$\alpha$ proteins are present in all four of these families. Multiple G$\alpha$ proteins in \textit{Bd} and \textit{Hp} appear to fall within the fungal Group IV (for which there is no \textit{N. crassa} homolog, however there is an \textit{Ustillago maydis} homolog implicated in pathogenicity).\\
\indent Phosphodiesterase is an enzyme which interacts with G$\alpha$. PDE subunits were found accross the fungal lineages, but in the basal lineages they appear in higher numbers relative to the non-flagellated dikarya. PDE$\gamma$ subunits were found exclusively in the Blastocladiomycete lineages. In the Zygomycete lineages, some species have similar presence patterns to chytrids, while others have presence patterns similar to the dikarya lineages. Transducin is one class of G$\alpha$ protein which is used in the mammalian visual cascade and which interacts with both rhodopsin and phosphodiesterase. This interaction activates PDE to lower the concentration of cGMP causing hyperpolarization of the cell and a decrease in calcium levels. The PDE subunits $\alpha$ / $\beta$ subunits found in the fungi are distinct from those found in the metazoan lineages.\\
\indent Despite the presence of presumably active photoreceptor proteins in \textit{S. punctatus} and \textit{B. dendrobatidis}, phototactic behavior was not observed in experiments similar to those described for \textit{Rhizophydium littoreum} \cite{Muehlstein1987}. There is no prior evidence of phototaxis in these species, only observed in \textit{Allomyces reticulatus} \cite{Saranak1997} and \textit{R. littoreum} \cite{Muehlstein1987}. One hypothesis as to the lack of phototaxis is that perhaps the GPCR type 2 rhodopsins are not used in phototaxis, but govern another light-regulated response mechanism. While the receptors are present, the downstream associated effectors and response-related protein components are not. This interpretation is supported by the fact that \textit{Bd} and \textit{Sp} do not posess PDE-$\gamma$ subunit homologs, which are known to have functions related to hyperpolarization and which could thus be related to flagellar beating.\\
\indent The lack of heterologous protein expression using the \textit{Pichia pastoris} cloning system was unexpected given the successes enjoyed with expression of the Bovine rhodopsin \cite{Abdulaev1997} and an opsin protein identified in \textit{N. crassa} \cite{Bieszke1999}. This lack of expression for chytropsins from \textit{B. dendrobatidis} and \textit{S. punctatus} using the BdpHIL-S1 and SppHIL-S1 vectors, respectively, can potentially be attribituted to the long intracellular loop regions found in these proteins introducing sufficient disorder so as to impede proper folding and membrane integration. For comparision, the cytoplasmic loop 3 (CL3) regions in Bovine rhodopsin (1U19) and Nop1 are only 6 and 8 amino acids long, respectively, whereas the CL3 regions in \textit{Bd} and \textit{Sp} are 99 and 36 amino acids long, respectively.\\
