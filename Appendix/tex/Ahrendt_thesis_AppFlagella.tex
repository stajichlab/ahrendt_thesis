%%%%%%%%%%%%%%%%%%%%%%%%%%%%%%%%%%%%%%%%%%%%%%%%%%%%%%%%%%%%%%%%%%%%%%%%%%%%%%%%%
%% Document: Thesis for PhD at UC Riverside                                    %%
%% Title: Investigating the evolution of environmental and biotic interactions %%
%%          in basal fungal lineages through comparative genomics              %%
%% Author: Steven Ahrendt                                                      %%
%%%%%%%%%%%%%%%%%%%%%%%%%%%%%%%%%%%%%%%%%%%%%%%%%%%%%%%%%%%%%%%%%%%%%%%%%%%%%%%%%
%% FLAGELLAR ANALYSIS %
%%%%%%%%%%%%%%%%%%%%%%%
\chapter{Comparative genomics analysis of Flagellar motility}
\label{app:Flagella}
\section{Introduction}
\section{Results and Discussion}
A search for Rozella homologs of flagellar-associated proteins from the \textit{Naeglaria} genome \cite{FritzLaylin2011} reveal a pattern of presence/absence of proteins in Rozella which correlates with that found in the Chytridiomycota and Blastocladiomycota. This pattern, in general, differs from the Microsporidia, supporting the placement of Rozella as prior to the Chytridiomycota/Blastocladiomycota and a flagellar loss event after the divergence of the Microsporidia. \\
\indent (Figure~\ref{fig:AppFlag_heatmap}).\\
\section{Methods}
The flagellar analysis was carried out using a dataset of 173 flagellar motility proteins obtained from the \textit{Naeglaria grubeii} genome sequence \cite{FritzLaylin2011}. These proteins were used in a FASTA search (SSEARCH v36.07) using an e-val cutoff of e-20. Supplemental table X only considers proteins which are present in at least one Dikarya, Microsporidia, or Chytridiomycota proteom. PSI-BLAST \cite{Altschul1997} was used to identify homologs of the three polar-tube proteins (PTP1, PTP2, and PTP3) characterized previously [PMIDs: 11719806, 12076771].\\
