%%%%%%%%%%%%%%%%%%%%%%%%%%%%%%%%%%%%%%%%%%%%%%%%%%%%%%%%%%%%%%%%%%%%%%%%%%%%%%%%%
%% Document: Thesis for PhD at UC Riverside                                    %%
%% Title: Investigating the evolution of environmental and biotic interactions %%
%%          in basal fungal lineages through comparative genomics              %%
%% Author: Steven Ahrendt                                                      %%
%%%%%%%%%%%%%%%%%%%%%%%%%%%%%%%%%%%%%%%%%%%%%%%%%%%%%%%%%%%%%%%%%%%%%%%%%%%%%%%%%
%% COELOMOMYCES TRANSCRIPTOME CHAPTER %%
%%%%%%%%%%%%%%%%%%%%%%%%%%%%%%%%%%%%%%%%
\chapter{Transcriptome analysis of the anopholean pathogenic fungus \textit{Coelomomyces lativittatus}}
\label{chap:ClatTranscriptome}

%%%%%%%%%%%%%%%%%%
%% Introduction %%
%%%%%%%%%%%%%%%%%%
\section{Introduction}
Species of Coelomomyces belong to the phylum Blastocladiomycota, one of the basal fungal lineages. These species in general are obligate parasites which cycle between insect and crustacean hosts \cite{Whisler1975}. The lifecycle initiates begins when biflagellate zygotes encounter mosquito larvae. The spore settles on and attaches to the host cuticle, a process facilitated by the secretion of adhesion vesicles which contain a glue-like substance \cite{Travland1979}. After secretion of a thin cell wall, the encysted spore develops an appressorium and penetration tube which breaks through the host cuticle \cite{Zebold1979}. Once inside the larval hemocoel, the spore develops into a sporangia. Host death liberates these sporangia. Meiosis within the sporangia produces haploid uniflagellate meiospores of opposing mating types, which are subsequently released to individually infect crustacean hosts (typically copepods, though ostracods can serve as hosts as well \cite{Whisler2009}). The penetration of copepods is thought to occur in a manner similar to that of the mosquito larvae \cite{Zebold1979}. Gametangia develop from these meiospores within the copepod hemocoel, which are ultimately cleaved into gametes and released upon crustacean host death. In the environment once again, opposing gametes fuse to create biflagellate zygotes, which propagate the cycle by infecting new mosquito larvae \cite{Whisler1975}.\\
\indent Coelomomyces species have been studied previously as a potential system for biocontrol of mosquito populations \cite{Scholte2004}. While the potential for use as a biological control agent has been explored, the exact biochemical nature of mosquito infection, including descriptions of all enzymes and pathways involved, has not. However, the advent and development of genomic tools facilitates the study of these questions.\\
\indent There are many examples of entomopathogenic organisms specializing in mosquito hosts, covering 13 genera across 2 kingdoms (Fungi and Chromista) \cite{Scholte2004}. The Ascomycete fungus \textit{Metarhizium anisopliae} is one of the well-studied fungal models for investigations into this specialized group. Early research looked at the range of enzymes produced by pathogenic isolates of this fungus, and identified a variety including proteases, amino-/carboxy- peptidases, lipases, esterases, chitinases, NAGases, catalases, polyphenol oxidases, and deoxy- and ribonucleases \cite{StLeger1986}. Later studies added to this repertoire the production of toxic cyclic peptides known as destruxins \cite{Wang2012}. \\ 
\indent The dual-host, multistage life cycle of Coelomomyces, which passes through a number of chemically distinct environments, suggests the presence of an elaborate sensory repertoire. For instance, experimental evidence demonstrates that gametes of some Coelomomyces species are specifically attracted to mosquito ovaries, and that this attraction is, at least in part, mediated by the hormone 20-hydroxyecdysone (20HE) \cite{Lucarotti1992}. Other evidence demonstrates a species-specific, photoperiod-dependent periodicity of gamete release from the copepod host \cite{Federici1983}, strongly implying that Coelomomyces has the molecular capacity for some manner of circadian rhythm regulation. \\
\indent Coelomomyces are known producers of $\beta$-carotene \cite{Federici1979}, the production of which is indicative of mating type, resulting in gametangia and gametes that are either strong orange (arbitrarily "male") or colorless/amber (arbitrarily "female"). $\beta$-carotene is ubiquitous in nature and exists primarily as a precursor for the biosynthesis of Vitamin A.\\
\indent The total number of species of Coelomomyces worldwide is estimated to be several hundred, yet little is known about the more detailed aspects of biochemistry and genomics. Therefore, the work described in this chapter is motivated by an ongoing effort toward the assembly and annotation of a Coelomomyces transcriptome, which will not only add to the growing collection of knowledge about chytrid fungi broadly, but will also provide new insights into the underlying mechanisms that govern the alternating life cycle of Coelomomyces and can help further its development as a biological agent of mosquito control. \\
\indent This research represents the first exploratory investigation of Coelomomyces genomics using the transcriptome of \textit{C. lattivitatus}. In this chapter, I compare expressed protein functions relative to other zoosporic fungi, biochemically reconstruct known pathways of carotenoid and retinal biosynthesis, and identify potential members of what is presumed to be a vast and complicated sensory network. \\

%%%%%%%%%%%%%
%% Methods %%
%%%%%%%%%%%%%
\section{Methods}

%---- Culturing and library prep ----%
\subsection*{Mosquito and copepod cultures}
Mosquito larvae and copepods used for maintenance of \textit{C. lativittatus} cultures were \textit{Anopheles quadrimaculatus} and \textit{Acanthocyclops vernalis}, respectively. Cultures were maintained by the members of Dr. Brian Federici's entomology lab at UCR according to methods described previously \cite{Federici1983}.\\ 
\subsection*{RNA extraction and library preparation}
The following RNA extraction and library preparation protocol was performed in its entirety by Rob Hice of the Federici lab. Infected copepods were cleaned with water, and placed in tubes with distilled water. After the fungal gametes emerged from the copepod larvae, the copepod carcasses were allowed to settle to the tube bottom and the fungal supernatant was transferred to a new tube. Gametes/zygotes were spun down at 6000xg for 3 minutes, and the supernatant was removed. Pellets were snap-frozen in liquid nitrogen. RNA was extracted with Trizol  (Life Technologies, Grand Island, NY) as per the manufacturer's protocol. 1.2 $\mu$g of RNA was used as the starting material for the NEBNext Ultra Directional RNA Library Prep Kit for Illumina (New England BioLabs, Ipswich, MA). Poly-A RNA was purified as per instructions and converted to adapter-ligated, size-selected cDNA. An aliquot of the library was cloned into pJet1.2 (Thermo Fisher Scientific, Waltham, MA) and clones sequenced with standard methods to check library quality. An aliquot was also run on a Bioanalyzer 2100 (Agilent Technologies, Santa Clara, CA) to check average size, which was 371 bp (including adapters). The resulting library was sequenced by the Institute for Integrative Genome Biology Core facility at the University of California at Riverside using the MiSeq instrument (Illumina, San Diego, CA).\\

%---- Assembly and annotation ----%
\subsection*{Transcriptome assembly and annotation}
Transcriptome assembly was carried out using Trinity (v. r2014-02-14) \cite{Grabherr2011} with quality trimming performed using Trimmomatic \cite{Bolger2014}, normalization using Trinity's default built-in normalization process, and reconstruction using PasaFly, an implementation of the PASA assembly algorithm \cite{Haas2013}. ORF prediction was carried out using Transdecoder \cite{Haas2013} using the PFAM database. Annotation was performed using Trinotate, also part of the Trinity package.\\

%---- Insect virulence ----%
\subsection*{PFAM distribution}
PFAM domain distribution for the \textit{C. lativittatus} transcriptome was predicted using an HMM search against the PFAM database. Hits above a threshold of 1e-05 were retained. The top 20 most abundant domains were then queried in the \textit{Allomyces macrogynus, Catenaria anguilullae, Batrachochytrium dendrobatidis, Spizellomyces punctatus}, and \textit{Homolaphlyctis polyrhiza} proteomes to determine relative abundance and putative expansion of domains. \\
\subsection*{Insect virulence survey}
Enzymes predicted to be related to insect virulence were used as search queries against the \textit{C. lativittatus} transcriptome. To search for peptidase proteins, an HMM profile constructed from the 157 seed sequences for the PFAM domain family PF00112 (Peptidase\_C1) was used as a query in HMMER (v3.0) $hmmsearch$ against basal fungal proteomes (Table~\ref{tab:AppData_taxa}), and the resulting hits were filtered according to presence of the functional catalytic diad residues Cys25 and His159. Identified hits were trimmed such that only the mature peptide (defined as the C-terminal 200 residues, lacking the signal and precursor regions) were used in the subsequent analyses. A maximum likelihood tree using the seed sequences and the filtered fungal sequences was generated using RAxML (v7.5.4) using 100 bootstrap replicates and the GTR+$\Gamma$+WAG substitution model. Similarly for the Trypsin proteins, an HMM profile constructred from the 71 seed sequences for the PFAM domain family PF00089 (Trypsin) was used in an HMMsearch. The resulting hits were filtered according to presence of the functional catalytic triad residues His57, Asp102, and Ser195. A maximum likelihood tree was constructed using RAxML (v7.5.4) using 100 bootstrap replicates and the GTR+$\Gamma$+WAG substitution model. The Ecdysone receptor from \textit{Drosophila melanogaster} (GI:157318) was used as a query using $ssearch36$ at a threshold of 1e-5. A maximum likelihood tree using human nuclear receptors from all major families and EcR receptors identified in the majority of insect orders was generated using RAxML (v7.5.4) \cite{Stamatakis2014} using 100 boostrap replicates and the GTR+$\Gamma$+Dayhoff protein model.\\
\indent To predict potential adhesion related proteins, I obtained a dataset from the Fungal Adhesin and Adhesin-like Database (FaaDB; \url{http://bioinfo.icgeb.res.in/faap/faap.html}) containing experimentally verified fungal adhesins from predominantly Dikarya. This positive dataset was then used as a query in a FASTA search (using $ssearch36$) against the \textit{C. lativittatus} transcriptome. Hits matching an e-val threshold of 1e-10 were submitted to the FAApred SVM-based prediction method, which was trained on both positive and negative adhesin datasets. Positive matches had an SVM score greater than -0.8.\\
\indent TMHMM (v2.0) \cite{Krogh2001} was used to predict transmembrane proteins in the basal fungal proteomes and \textit{C. lativittatus} transcriptome. Those with 6-9 domains were retained and compared using OrthoMCL \cite{Li2003}. \\

%---- Beta carotene ----%
\subsection*{$\beta$-carotene survey}
To assess the completeness of the $\beta$-carotene pathway in \textit{C. lativittatus}, I queried the transcriptome using the three $\beta$-carotene biosynthesis enzymes, using the sequences obtained from \textit{Blastocladiella emersonii}: phytoene desaturase (Uniprot: KJ468786), lycopene cyclase / phytoene synthase (Uniprot: KJ468785), and $\beta$-carotene 15,15'-monooxygenase (BCMO1) (Uniprot: KJ468787). While functional biochemical characterization of these specific B. emersonii enzymes has not been performed, a BLASTP search against the SwissProt database reveals expected top hits with experimental verification of biochemical activity (Table~\ref{tab:ChClat_BemeVerify}). Additionally, HMM profiles were generated from sequences available from the Kyoto Encyclopedia of Genes and Genomes (KEGG) database \cite{Kanehisa2000,Kanehisa2014}: phytoene desaturase, K10027; phytoene synthase, K02291; $\beta$-carotene 15,15'-monooxygenase, K00515. When available, the Metazoan, Eukaryote, and/or Plant genes were used. Otherwise, the bacterial and archaeal protein sequences were used (Table~\ref{tab:ChClat_KEGGHMM}).\\
\indent Due to the lack of hits in \textit{C. lativittatus} for the \textit{B. emersonii} phytoene desaturase sequence, I made additional queries using the phytoene desaturase from \textit{Giberella fujikuroi} (CarB; UniProt accession: Q8X0Z0) and \textit{Neurospora crassa} (NCU00552).\\
\indent A maximum likelihood tree was constructed using RAxML (v7.5.4) with the identified BCMO1 fungal sequences and those from KEGG families using the GTR+$\Gamma$+LGF model and 100 bootstrap replicates. \\

%---- Photosensory ----%
\subsection*{Photosensory survey}
Putative photosensory proteins were identified using known fungal photobiology proteins WC-1 ,WC-2, FRQ, VIVID, and FWD-1 from \textit{N. crassa} \cite{Borkovich2004}. Cryptochrome homologs were searched using HMMER (v3.0) $hmmsearch$ with an HMM generated from the 20 seed sequences in the PFAM domain PF12546 (Cryptochrome\_C). Hits were retained above a threshold of 1e-20. Similarly, phytochrome homologs were searched using $hmmsearch$ with an HMM generated from the 80 seed sequences in PF00360 (PHY). Hits were retained above a threshold of 1e-20. \\
\indent Opsin proteins were identified as containing the PF00001 (7tm\_1) or PF01036 (Bac\_rhodopsin) domains after prediction from Trinotate as described above. Additional support was provided by TMHMM (v2.0) as containing either 6 or 7 transmembrane domains. BacOpsin-GC fusion proteins were identified as posessing both the PF01036 and PF00211 (Guanylate\_cyc) domains as predicted by Trinotate.\\

%%%%%%%%%%%%%
%% Results %%
%%%%%%%%%%%%%
\section{Results}

%---- Transcriptome overview ----%
\subsection*{Transcriptome Characterization}
After quality trimming, obtained a total of 28,698,279 reads with an average length of 196 nt. De novo assembly of reads using Trinity \cite{Grabherr2011} yielded 77,597 transcripts with an average length of 386 bp. Within these transcripts, 21,486 open reading frames (ORFs) were predicted using Transdecoder \cite{Haas2013}. Annotation with Trinotate predicted 12,156 transcripts with a BLASTp hit, 11,040 with predicted Pfam domain(s), and 29,076 with associated GO terms. \\
\indent The top 20 PFAM domains identified in the \textit{C. lativittatus} transcriptome and their respective counts in other chytrids are provided in Table~\ref{tab:ChClat_PFAM}. The most striking examples of domain families which are underrepresented among other chytrids are trypsin (PF00089), glycoside hydrolase family 47 (PF01532), and papain family cysteine protease (PF00112), all three of which have some manner of protease or carbohydrate degrading functionality. An additional family which appears to be overrepresented in \textit{C. lativittatus} is the Myosin tail family (PF01576), although the related Blastocladiomycete \textit{C. anguillalae} also has a higher number of these proteins relative to other Blastocladiomycete and Chytridiomycete speceis. Corresponding Gene Ontology (GO) Slim classifications for recovered transcripts are shown in Figure~\ref{fig:ChClat_GOPlot}. \\

%---- Insect virulence ----% 
\subsection*{Insect Virulence} 
\indent To test for the presence of and possible expansions in gene families that may be related to insect virulence, I scanned the \textit{C. lativittatus} transcriptome for specific protein domains which have been previously implicated in fungal associated insect virulence, or which may be otherwise related to fungal-insect association. \\
\subsubsection*{Proteases} 
The C1 cysteine proteases are commonly found in fruit (eg. papaya) and often used as meat tenderizers. The enzymes from fig, pineapple, and papaya plants have been studied as antihelmintics and found to have high proteolytic activity against nematode cuticles \cite{Stepek2004}. The family is characterized by the Peptidase C1 (PF00112) and C1-like (PF03051) Pfam domains. The \textit{C. lativittatus} transcriptome contains at least 56 transcripts containing peptidase C1 domains (< 98\% identity). Searches of Blastocladiomycete and Chytridiomycetes genomes found no proteins containing these domains, although proteins with this domain are present in the Dikarya lineages. Phylogenetic analysis of the Pfam seed sequences and a reduced set of the fungal copies revealed a number of observations (Figure~\ref{fig:ChClat_PF00112}). First, the \textit{C. lativittatus} transcripts are broadly distributed, with very few tight clusters. None of the transcripts cluster within the other Dikarya sequences; instead they fall sporadically among the other metazoan sequences, predominantly with other lower eukaryotes. One group of transcripts cluster as more recent divergences closer to the arthropod sequences. \\
\indent Trypsins are serine proteases found in the digestive systems of many vertebrates \cite{Rawlings1994}. These enzymes are characterized by the PF00089 Pfam domain, and 43 transcripts in \textit{C. lativittatus} were identified as having this domain. Searches of other non-insect associated Blastocladiomycte and Chytridiomycete proteomes revealed an order of magnitude fewer proteins containing these domains. Based on the presence of the catalytic triad, a collection of three residues (His-57, Asp-102, Ser-195) which are critical to active site function, 20 \textit{C. lativittatus} sequences were filtered and used in a phylogenetic analysis along with Pfam seed sequences. Most of the Chytrid sequences cluster with other fungi and away from the metazoan sequences.\\
\subsubsection*{Destruxins}
The destruxins are a class of insecticidal cyclic hexadepsipeptides produced by some entomopathogenic fungi, most notably by species of Metarhizium \cite{Donzelli2012,Wang2012}. Based on chemical differences in the hydroxy acid, R group, and N-methylation characteristics, these compounds can be divided into a total of 12 chemically distinct classes \cite{Pedras2002,Wang2012}. The biosynthesis of these compounds is presumed to be mediated by an NRPS gene cluster in \textit{Metarhizium robertsii} \cite{Wang2012}. A FASTA search with the destruxin synthase (dtxS1) protein in the \textit{M. robertsii} gene cluster did not identify any putative homologs in our \textit{C. lativittatus} transcriptome. No putative NRPS or PKS-related proteins searching transcriptome using antiSmash \cite{Blin2013}, though m.15019 (described in $\beta$-carotene results as a phytoene synthase) was recovered as a putative terpene synthase. Additionally, no hits for THIOL or CON using HMM searches were recovered \cite{Bushley2010}. Some hits from AMP HMM, but counts are on the order of other chytrids (~15-20).\\
\subsubsection*{Chitin related domains}
Chitin binding domains are a broad class of domains found in carbohydrate-active proteins. Overall, there are 71 different subfamilies within this broad class defined by sequence similarity in the Carbohydrate Active Enzymes database \cite{Lombard2014}. Five predicted ORFs were identified by InterPro as having a CBM18 domain and six ORFs identified with a CBM33 domain. One transcript (m.4968) was identified as posessing a chitin synthase domain (PF03142), and one transcript (m.4725) with a NADH-Ubiquinone domain (PF00361).\\
\subsubsection*{Adhesion-related proteins}
In the infection process, when biflagellate zygotes encounter mosquito larvae, the spore is observed to settle on and attach to the host cuticle. This process is hypothesized to be facilitated by the secretion of so-called "adhesion vesicles" which contain a glue-like substance \cite{Travland1979}. These vesicles have been observed developing prior to the attachment of the spore, localizing to points of contact between the spore and cuticle, and disappearing after host penetration \cite{Travland1979}. While the chemical nature of these "adhesion vesicles" remains unclear, a number of candidates exist. Fungal adhesins, for example, are membrane proteins which allow certain fungi to attach to surfaces and are usually involved in microbial community biofilm formation. One well studied example is the "hyphal wall protein (Hwp1)" implicated in \textit{Candida albicans} pathogenesis \cite{Staab1999}. However a FASTA search using $ssearch36$ did not recover any homologs of this protein in \textit{C. lativittatus} or other Blastocladiomycete or Chytridiomycetes surveyed.\\ 
\indent In an additional attempt to ascertain the nature of spore-cuticle attachment, I probed the FaaDB for putative adhesins in \textit{C. lativittatus}. This method identified 16 sequences as putative adhesins. In the other Blastocladiomycte and Chytridiomycetes surveyed, 10, 5, 10, 4, and 4 proteins were predicted as such in \textit{A. macrogynus, C. anguillulae, B. dendrobatidis, H. polyrhiza}, and \textit{S. punctatus}, respectively.\\ 
\subsubsection*{Ecdysone receptors}
The naturally occurring ecdysteroid hormone 20-hydroxyecdysone (20HE) controls moulting in arthropods \cite{Thummel2002}. There is evidence to suggest that 20HE plays a role in attracting \textit{Coelomomyces stegomyiae} to the ovaries of adult female \textit{Aedes aegypti} \cite{Lucarotti1992}. A FASTA search using $ssearch36$ with the known ecdysone receptor protein from \textit{D. melanogaster} EcR \cite{Koelle1991} identified a single \textit{C. lativittatus} transcript. This finding is surprising not only as it provides a straightforward answer to how \textit{C. lativittatus} could sense its host, but also given the presumption that nuclear receptors are only limited to the metazoan lineages and not found in fungi \cite{Escriva1998}. An HMM profile constructed from arthropod EcR receptor sequences and human nuclear receptors, when searched against the \textit{C. lativittatus} transcriptome, identified an additional three transcripts, though the originally identified transcript (m.9546) remained the highest scoring. An alignment is provided in Figure~\ref{fig:ChClat_20HEalign}.\\
\indent This 298 aa transcript is likely not full length, and only aligns to the DNA binding region of the \textit{D. melanogaster} receptor (approximate residues 239 to 401). The top blast hit for this transcript is the \textit{Caenorhabditis elegans} nuclear hormone receptor family member nhr-35 (SwissProt accession: Q17771, e-val 2e-20). InterProScan \cite{Jones2014} predicts the PF00105 domain covering positions 24-92. This domain is a Zinc Finger C4-type and is associated with nuclear receptors. No orthologs of the \textit{C. lativittatus} transcript were detectable in any other chytrids searching with an e-value threshold of at least 1e-05. \\
\indent Structurally, this transcript is most similar to the DNA-binding region of the \textit{D. melanogaster} ecdysone receptor (PDB ID: 2HAN, chain B) \cite{Jakob2007}. These two regions have 42\% sequence identity. A homology-based structure model of the \textit{C. lativittatus} transcript using SwissModel \cite{Arnold2006} has an RMSD of 0.2 (Dali Server prediction \cite{Holm2010}) when compared to 2HAN, chain B. \\
\indent The PF00104 ligand binding domain, associated with this and other nuclear receptors in the arthropod receptors, was not predicted to be associated with this transcript. However one other \textit{C. lativittatus} ORF is predicted to contain the PF00104 domain but has insignificant similarity to the \textit{D. melanogaster} EcR protein (m.10080, 21.5\% identity, e-val: 0.077). Nonetheless, Table~\ref{tab:ChClat_LBD} lists BLASTP results after searching the SwissProt database with m.10080. The top 5 hits are all to mammalian liver X receptors (LXRs). A total of six hits using a cutoff threshold of 1e-06 are to 20HE receptors from insects. All hits have approximately 40\% coverage and approximately 25\% identity. \\
\indent A maximum likelihood tree (Figure~\ref{fig:ChClat_20HE_NRtree}) constructed from arthropod 20HE sequences, as well as human nuclear receptor sequences from all nuclear receptor families, shows the \textit{C. lativittatus} putative DNA-binding homolog sequence falling outside of the metazoan nuclear receptor sequences. \\
\indent Finally, to determine if any other, non-hormone based receptors are uniquely found in \textit{C. lativittatus} relative to the other, non-insect associated chytrids, I searched for possible receptor candidate genes based on transmembrane domain architecture. In \textit{C. lativittatus}, 131 transcripts were predicted to have between 6 and 9 transmembrane domains. Of these, 29 are specific and not found in the other Chytridiomycota or Blastocladiomycota genomes surveyed based on ortholog clusters generated with OrthoMCL \cite{Li2003}. These 29 transcripts form 12 unique paralog clusters. HMMER (v3.0) searches of the Pfam database identified domains in 8 of these clusters, while the other 4 remained unclassified (Table~\ref{tab:ChClat_orthomcl}).\\

%---- Beta carotene ----%
\subsection*{$\beta$-carotene} 
\textit{C. lativittatus} likely has a typical $\beta$-carotene biosynthesis pathway, despite missing an enzyme in the transcriptome (Figure~\ref{fig:ChClat_Bcaro}). To determine the molecular characteristics of the $\beta$-carotene biosynthesis and metabolism pathways in \textit{C. lativittatus}, I queried the predicted ORFs from the transcriptome with three key enzymes from the biosynthesis pathway described in \textit{Blastocladiella emersonii} \cite{Avelar2014}. While functional biochemical characterization of these specific \textit{B. emersonii} enzymes has not been performed here or otherwise, a BLASTP search against the SwissProt database reveals expected top hits with experimental verification of biochemical activity (Table~\ref{tab:ChClat_BemeVerify}). \\
\indent The biosynthesis of $\beta$-carotene utilizes three enzymes and acts upon geranylgeranyl pyrophosphate. Phytoene synthase converts two molecules of geranylgeranyl pyrophosphate to one molecule of phytoene. Phytoene desaturase then works in a five-step pathway to convert phytoene into lycopene \cite{Hausmann2000}. Lycopene cyclase finally acts to convert the lycopene to $\beta$-carotene \cite{Cunningham1994}. The lycopene cyclase and phytoene synthase enzymes (fulfilling the first and last steps) are encoded as a single polypeptide.\\
\indent No candidate \textit{C. lativittatus} homolog was found with the putative \textit{B. emersonii} phytoene dehydrogenase sequence (KJ468786) in either the set of predicted ORFs (using the protein sequence in a direct search) nor in the set of assembled transcripts (using the protein sequence in a translated search). Additional queries using phytoene desaturase from \textit{Giberella fujikuroi} (CarB; UniProt accession: Q8X0Z0) and \textit{Neurospora crassa} (NCU00552) were similarly unsuccessful. \\
\indent One transcript, m.15019, contained a 599-aa long predicted ORF, and was identified as a putative homolog to \textit{B. emersonii} bifunctional lycopene cyclase / phytoene synthase (KJ468785) at 38.4\% identity. The best BLASTX hit of the \textit{C. lativittatus} transcript for this ORF against the SwissProt database was a "bifunctional lycopene cyclase/phytoene synthase" from\textit{ Phycomyces blakesleeanus} (UniProt accession Q9P854; e-val 3e-95). The m.15019 transcript has an FPKM value of 3.42. \\
\indent The conversion of $\beta$-carotene to retinal is facilitated by $\beta$-carotene 15,15'-monooxygenase (BCMO1). A FASTA search using the \textit{B. emersonii} putative carotenoid dioxygenase sequence (KJ468787) identified two transcripts contained ORFs which were identified as putative homologs, m.16827 (670-aa, 44.2\% identity) and m.4639 (156-aa, 26.6\% identity). The top BLASTP hit against SwissProt for m.16827 was BCMO1 from \textit{Homo sapiens} (UniProt accession Q9HAY6; e-val: 1e-44), and that for m.4639 was BCMO1 from \textit{Mus musculus} (UniProt accession Q9JJS6; e-val: 3e-09). These transcripts had FPKM values of 2.57 and 0.997, respectively. \\
\indent To provide additional support for the candidate transcripts identified above, HMM profiles were generated from sequences available from the Kyoto Encyclopedia of Genes and Genomes (KEGG) database \cite{Kanehisa2000,Kanehisa2014}. The candidate \textit{C. lativittatus} transcripts described above were also recovered using $hmmsearch$ with these HMM profiles. \\
\indent Sequence searches identified all three of these $\beta$-carotene metabolism genes in the genomes of two Blastocladiomycota fungi, \textit{Allomyces macrogynus} and \textit{Catenaria anguillulae}. Similar searches of the genomes of the Chytridiomycota fungi \textit{Batrachochytrium dendrobatidis}, \textit{Homolaphlyctis polyrhiza}, and \textit{Spizellomyces punctatus}, found an incomplete complement of these genes of this pathway. The \textit{B. dendrobatidis} genome contains no homologs for any of these genes, while the \textit{H. polyrhiza} genome contains a candidate phytoene desaturase homolog (top BLASTP hit against SwissProt: phytoene desaturase from \textit{P. blakesleeanus} [P54982.1], e-val: 2e-68, 49\% identity), and \textit{S. punctatus} possesses a candidate $\beta$-carotene oxygenase homolog (top BLASTP hit against SwissProt:$\beta$,$\beta$-carotene 9',10'-oxygenase from \textit{Macaca fascicularis} [Q8HXG8.2], e-val: 1e-37, 26\% identity).\\
\indent To assess the phylogenetic history of BCMO1, a maximum likelihood tree was constructed from homologs found in Ascomycete, Chytridiomycete, Blastocladiomycete, and Zygomycete species (Figure~\ref{fig:ChClat_BCMO1DO2}). Examination of the resulting gene tree topology provides strong support for the early-diverging fungal genes to cluster distinctly outside the metazoan gene lineages, and suggests at least 3 major duplication events. At least one duplication occurred exclusively in the metazoan lineages to give rise to BCMO1 and BCDO2. One duplication likely occurred prior to the fungal/metazoan divergence, resulting in the copies seen in the Chlorophyta, Dikarya, Zygomycota. A second duplication likely occurred after the divergence of the fungi and prior to the divergence of the Cryptomycota, resulting in two subtypes of fungal $\beta$-carotene oxygenase. Interestingly, while copies of each subtype can be found in members of the Zygomycota, only one or the other is found in the Chytridiomycota and Blastocladiomycota.\\

%---- Photosensing ----%
\subsection*{Photosensing capacity}
To determine the possible nature of observed photosensory capacity in \textit{C. lativittatus}, I searched the transcriptome for ORFs predicted to be associated with photobiology (Table~\ref{tab:ChClat_photosensing}) using known fungal photobiology proteins, including opsins and opsin-like proteins, circadian rhythm proteins (WC-1 and 2, FRQ, and FWD-1), cryptochromes, phytochromes, and the photoreceptor protein VIVID. \\
\indent A search for homologs to \textit{Neurospora crassa} White Collar-1 (WC-1, NCU02356) and White Collar-2 (WC-2, NCU00902) proteins identified a 368 aa ORF (m.12730) as a potential WC-1 homolog. They are reciprocal best-blast hits as NCU02356 is the top BLASTP hit against SwissProt using this m.12730 as a query (UniProt id: Q01371, e-val: 4e-43, 35.2\% identity). Additionally, m.12730 is predicted by InterPro to contain two PAS domains (IPR000014), similar to NCU02256. However, it is much shorter: only 368 aa compared to 1167 aa for NCU02256. Other WC-1 homologs were recovered from \textit{H. polyrhiza, S. punctatus, C. anguillalae}, and \textit{A. macrogynus} (Figure~\ref{fig:ChClat_PASaln}). No ORFs were predicted as WC-2 homologs, nor were there any other potential PAS-domain containing transcripts. \\
\indent In addition to the white collar complex proteins WC-1 and WC-2, the blue-light sensitive photoreceptor protein VIVID (VVD) identified in \textit{N. crassa} and other filamentous fungi is a small (186 aa), cytoplasmic flavoprotein that responds to increasing light intensity \cite{Schwerdtfeger2003}. No putative homologs of the \textit{N. crassa} VVD protein (NCU03967) were recovered in a BLASTP search against the \textit{C. lativittatus} transcriptome. Additionally, no homologs were observed in the other Blastocladiomycete and Chytridiomycete species surveyed. This absence is consistent with an observed absence of VVD homologs outside of the Sordariomycete lineages.\\
\indent Phytochromes are, among other things, circadian rhythm regulators in plants \cite{Rockwell2006}, and Velvet A homologs are demonstrated to be regulators for secondary metabolism and sporulation in several fungi \cite{Calvo2008}. There were no putative phytochrome homologs identified in \textit{C. lativittatus} using an HMM generated from seed sequences in PFAM family PF00360, nor any homologs of the phytochrome-associated Velvet protein using either the \textit{N. crassa} Velvet A-like protein NCU01731 or an HMM generated from \textit{N. crassa} and additional \textit{Aspergillus} and \textit{Fusarium} sequences. While phytochromes are known to be present in the Chytridiomycete \textit{Spizellomyces punctatus} (see \cite{Idnurm2010}), there were no homologs for Velvet or phytochrome observed in other members of the Chytridiomycota and Blastocladiomycota. \\
\indent A total of 6 ORFs were predicted to be opsin-related proteins based on predicted Pfam domains and predicted seven transmembrane helical domain architecture (Table~\ref{tab:ChClat_photosensing}). Of particular note is a predicted 537 aa ORF (m.7819), which has two identifiable Pfam domains: a 213 aa region with similarity to bacterial rhodopsin (PF01036; e-val: 4.6e-22), and a 178 aa region with similarity to guanylate cyclase (PF00211; e-val: 3.6e-51). This architecture is similar to that found in \textit{Allomyces macrogynus} and \textit{Catenaria anguillalae}, and described more fully in \textit{Blastocladiella emersonii} \cite{Avelar2014}. A homolog was also identified in \textit{Homolaphlyctis polyrhiza}. When compared with the other examples of this protein architecture found in the Blastocladiomycota and Chytridiomycota, this \textit{C. lativittatus} transcript shares 61.91\%, 72.98\%, 71.19\%, 64.20\%, 63.36\%, and 54.55\% identity with the \textit{B. emersonii}, each of the four \textit{A. macrogynus}, and the \textit{H. polyrhiza} proteins, respectively.\\
\indent To ascertain the putative placement of this \textit{C. lativittatus} Opsin-GC fusion protein among other opsin proteins, a maximum likelihood tree was generated using opsin sequences from Avalar et al., with additional inclusion of the opsin-GC fusion sequences recovered from \textit{H. polyrhiza, C. anguillalae, and C. lativittatus} (Figure~\ref{fig:ChClat_OpsinGCFusion}). The \textit{C. lativittatus} and \textit{C. anguillalae} sequences cluster expectedly with the other Blastocladiomycete sequences (\textit{A. macrogynus} and \textit{B. emersonii}) in a well supported early-diverging fungal group. Perhaps unexpectedly, the sequence from the Chytridiomycete \textit{H. polyrhiza} falls within, rather than outside of, the Blastocladiomycete sequences, albeit with a relatively long branch.\\

%%%%%%%%%%%%%%%%
%% Discussion %%
%%%%%%%%%%%%%%%%
\section{Discussion}
The research presented in this chapter dealt with one member of only known group of insect pathogens in the basal fungal lineages, \textit{Coelomomyces lativittatus}. The initial transcriptome generation and analysis represents the first generation of bioinformatic resources for this fungus and sought to support the phylogenetic placement within the Blastocladiomycota through three traits of \textit{C. lativittatus} biology: I) insect pathogenicity, II) $\beta$-carotene biosynthesis, and III) photosensing capacity.\\
\textit{Coelomomyces lativittatus} is a member of the only known genus of insect pathogens among the basal fungal lineages and has been well-studied as a potential mosquito control agent. This transcriptome study represents the first attempt at developing available genomic and proteomic resources for this and other \textit{Coelomomyces} species. Future work will most assuredly expand on the results demonstrated here, including whole genome sequencing, developmental and life stage-specific RNA sequencing, and proteomic extraction and characterization. Nonetheless, some observations from the analyses performed here are useful in the comparative genomics of non-insect associated early-diverging fungi, and can also provide a focus for future work in \textit{C. lativittatus}. \\
\indent While there are several examples of entomopathogenic fungi, \textit{C. lativittatus} and other members of \textit{Coelomomyces} are the only known members of the Blastocladiomycota which demonstrate this association, and the biological mechanism by which \textit{Coelomomyces psorophorae} infects mosquito larvae has been documented previously \cite{Travland1979,Zebold1979}. In general, infection is initiated by the settling of the spore onto the host cuticle, followed by encystment and secretion of thin cell wall. The appearance of an appressorium and subsequent development of a penetration tube which pierces the integument of the host then allows the fungus to enter the host hemocoel \cite{Zebold1979}. \\
\indent As noted previously in \textit{C. psorophorae} \cite{Travland1979}, there is a correlation between disruption of the outermost layer of the cuticle and accumulation of an amorphous, electron-dense material at the cuticle-contacting tips of penetration tubes. As the appressorium tip is the site of actual penetration through the cuticle and into the mosquito larvae, a speculative explanation of the observable electron-dense material would be the proteases and other degradation-related proteins unique to \textit{C. lativittatus} recovered in this study. Indeed, a hypothesis postulated at the time suggested that this material may be enzymatic in nature \cite{Travland1979}. \\
\indent A comparison of counts of the top 20 PFAM domains (Table~\ref{tab:ChClat_PFAM}) suggests four protein families which appear to be uniqely expanded in \textit{C. lativittatus} relative to the other non-insect associated Blastocladiomycete and Chytridiomycete species. These include "myosin tail" (PF01576), "glyco\_hydro\_47" (PF01532), "trypsin" (PF00089), and "C1 peptidase" (PF00112). Of these four, the latter two have clear protease and degradation functions. While the PF00112 phylogenetic history is a little unclear, at least two \textit{C. lativittatus} specific groups can be identified, one of which appears to cluster with known arthropod sequences. An additional \textit{C. lativittatus} sequence clusters with the plant sequences from papaya and pineapple, known to have culticle degrading activity in nematodes. Further work, especially life-stage dependent RNAseq experiments are completely necessary to confirm this hypothesis and would be critical in order to map expression levels of these and other protease genes before, during, and after infection. Aspects of infection which cannot be described further in this study are the adhesion vesicles, which are hypothesized to secrete the "glue" that attaches the spore to the host, and the observed pseudopodia structures, which appear after the settling of the spore. \\
\indent Searches for a 20-hydroxyecdysone receptor based on similarity to known arthropod receptors identified at least one candidate transcript with similarity to the DNA binding domain of the ecdysone receptor in \textit{D. melanogaster}. This DNA binding domain profile, PFAM id: PF00105, is specifically associated with nuclear receptors. Sequence, structure, and phylogenetic analysis suggests that while it is significantly diverged in sequence, it may be a hormone receptor and is unlikely to be the result of a horizontal transfer event or sequence contamination. Furthermore, homologous sequences are not found in other Blastocladiomycete and Chytridiomycete species surveyed, providing a tantalyzing explanation of Coelomomyces observed affinity for 20-hydroxyecdysone from mosquitoes. However, I am hesitant to hold this as undeniable evidence of presence of this receptor in \textit{C. lativittatus}; rather it is submitted as a starting point for future analyses. Further work to evaluate gene expression changes in this and other transcripts when C. lativittatus is exposed to the anophelid larvae or the 20-hydroxyecdysone hormone are absolutely crucial and will provide better insight into these candidate genes. \\
\subsection*{Sensory results are consistent with previous hypotheses, but undetermined if this represents a specific insect association aspect.}
The presence of a moderately complex sensory network governing the full \textit{C. lativittatus} life cycle can be inferred from experimental research on other \textit{Coelomomyces} species. For example, \textit{C. psorophorae} zygotes need to seek out \textit{Culiseta inornata} larvae \cite{Whisler1975}. Once infected, the zygotes must develop into sporangia, regulate the timing of meiospore release, and these meiospores need to find the crustacean host: the copepod \textit{Cyclops vernalis} in the case of \textit{C. psorophorae} \cite{Whisler1974}. Once inside the crustacean host, similar regulation of sporangial development and spore release must also take place, but in a much different environment. Reared under identical conditions, dehiscence of \textit{Coelomomyces dodgei} and \textit{Coelomomyces punctatus} occurs at significantly different times \cite{Federici1983} suggesting the presence of a photoperiod dependent regulatory mechanism. These spores must then seek out members of the opposite mating type \cite{Federici1983}, fuse to form zygotes, and exhibit phototactic capabilities to swim upwards to the water surface \cite{Federici1983}. \\
\indent Given this evidence in other Coelomomyces species, \textit{C. lativittatus} likely also possesses a complex sensory network relative to chytrids which display no insect association. The demonstrated ability for photoperiod regulation prompted our search for transcripts predicted to be involved in photosensing. From this search, one putative homolog of the \textit{N. crassa} White collar-1 protein was identified. The remaining components of the white collar / circadian rhythm process (White collar-2, FRQ, FWD-1), however, were not recovered. The White collar-2 protein is present, however, in the chytridiomycete \textit{S. punctatus} and the Blastocladiomycete \textit{C. anguillulae}, but is absent in the other Chytridiomycete and Blastocladiomycete species surveyed. Therefore it is not necessarily unusual to find its incomplete presence in \textit{C. lativittatus}, especially given the limitations of the current transcriptome study. \\
\indent Several proteins predicted to be opsins or opsin-related were identified based on transmembrane domain architecture and PFAM domain identification. Notably, one transcript is predicted to have a type 1 microbial rhodopsin domain fused with a guanylate cyclase domain. This structure is similar to a novel fusion protein recently described in the Blastocladiomycete \textit{B. emersonii}, and additional homologs can be identified in the genome assemblies of other Blastocladiomycetes \textit{A. macrogynus} and \textit{C.  anguillulae} \cite{Avelar2014}. The mechanism of activity of the fusion protein described for \textit{B. emersonii} is that light activates the type 1 rhodopsin domain, which in turn activates the coupled GC domain. This facilitates synthesis of cGMP, which activates K$^{+}$-selective cyclic nucleotide gated channels. Voltage-activated Ca$^{2+}$ channels, activated by the resulting hyperpolarization of the plasma membrane, would elevate Ca$^{2+}$ levels, prompting interaction with the flagellum and ultimately mediating phototaxis \cite{Avelar2014}. \\
\indent The presence of this fusion protein in \textit{C. lativittatus}, in addition to its previously described presence in \textit{B. emersonii, A. macrogynus}, and \textit{C. anguillulae}, supports the hypothesis that the novel fusion gene appeared prior to the divergence of the Blastocladiomycota lineage as it can now be said to be present in all three of the Blastocladiaceae, Catenariaceae, and Coelomomycetaceae families. Furthermore, its presence in the Chytridiomycete \textit{H. polyrhiza} suggests that the fusion appeared earlier. However the fusion architecture does not appear in any of the other Chytridiomycete genomes surveyed, suggesting that its presence in \textit{H. polyrhiza} is the result of either a recent fusion event, duplication and losses in the other Chytridiomycete lineages, or HGT event.\\
\subsection*{$\beta$-carotene biosynthesis and metabolism pathways are present and nearly complete in the \textit{C. lativittatus} transcriptome.}
Endogenous $\beta$-carotene production is biologically important for many reasons, one of which is that it functions as precursor to retinal, the critical component of rhodopsin-mediated photoreception. Carotenoid production in the Blastocladiomycota is well known, and while $\gamma$-carotene is the predominant molecule in \textit{B. emersonii} and several Allomyces species, $\beta$-carotene specifically is known to be produced by \textit{C. dodgei}. In this and other \textit{Coelomomyces} species, the relative levels of $\beta$-carotene are indicative of mating type, implying that either the production and/or regulation of $\beta$-carotene is at some level influenced by or related to the same mechanics which govern sexual reproduction. The extent of this relationship remains to be explored. \\
\indent The presence of nearly all critical enzymes in the retinal biosynthesis pathway in \textit{C. lativittatus} is consistent with these previous observations about, and suggests a fairly straightforward biological mechanism for, $\beta$-carotene production in Coelomomyces. Additionally, the presence of this pathway, coupled with the identification of multiple opsin-related transcripts, suggests that \textit{Coelomomyces} has the biochemical capacity for rhodopsin-mediated photoreception. \\
\indent However, the lack of a phytoene dehydrogenase homolog, the first step in $\beta$-carotene biosynthesis, is unusual given its presence in related Blastocladiomycetes A. macrogynus, \textit{B. emersonii}, and \textit{C. anguillulae}. This absence suggests that either this gene is not transcriptionally active during the life stage sampled, mRNA transcripts from this gene were not recovered at detectable levels during RNA extraction, or \textit{C. lativittatus} uses a novel mechanism for conversion of phytoene to lycopene to produce $\beta$-carotene. \\
\indent The enzyme responsible for conversion of $\beta$-carotene to retinal is $\beta$-carotene monoxygenase. This is one of two enzymes capable of cleaving $\beta$-carotene, the other being $\beta$-carotene dioxygenase. Two transcripts were recovered bearing similarities to the predicted BCMO1 protein from \textit{B. emersonii}, and strong similarity to BCMO1 profiles generated from known metazoan sequences. A phylogenetic reconstruction positions these transcripts expectedly within a Blastocladiomycete specific group of mono-oxygenase homologs, itself within a fungal-specific group. This suggests at least one duplication event occurred after the divergence of Fungi from the metazoan lineages. \\
\indent These findings presented in this chapter are descriptive and represent the first insights into the deeper molecular biology of this insect pathogen. The extent of the proteins, pathways, and networks studied in this work can be elucidated more completely once an annotated genome and life-stage specific transcriptomes are generated. For example, the diversity and copy number of sensory proteins actually present in the genome is likely to be higher than those captured by this transcriptome study, and a clearer picture of $\beta$-carotene biosynthesis will almost certainly be observed. \\
\indent Nonetheless, this initial work still provides a perspective on the underlying biology in the single chytrid pathogen of insects. Near-term future work build upon these findings and deal with obtaining a draft reference genome from \textit{C. lativittatus}, enhanced transcriptome sequencing accounting for important developmental timepoints, and proteomics studies dealing with surface receptors necessary for environment sensing. Potential long term applications of this and other work include exploitation as a means of mosquito population control. \\
