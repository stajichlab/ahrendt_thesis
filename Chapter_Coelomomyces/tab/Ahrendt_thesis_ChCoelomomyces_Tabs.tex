%%%%%%%%%%%%%%%%%%%%%%%%%%%%%%%%%%%%%%%%%%%%%%%%%%%%%%%%%%%%%%%%%%%%%%%%%%%%%%%%%
%% Document: Thesis for PhD at UC Riverside                                    %%
%% Title: Investigating the evolution of environmental and biotic interactions %%
%%          in basal fungal lineages through comparative genomics              %%
%% Author: Steven Ahrendt                                                      %%
%%%%%%%%%%%%%%%%%%%%%%%%%%%%%%%%%%%%%%%%%%%%%%%%%%%%%%%%%%%%%%%%%%%%%%%%%%%%%%%%%
% COELOMOMYCES TABLES %
%%%%%%%%%%%%%%%%%%%%%%%

% latex table generated in R 3.0.1 by xtable 1.7-4 package
% Fri Apr 24 09:45:44 2015
\begin{table}[tbp]
\centering
\begin{tabular}{rlrrrrrr}
  \hline
\hline
 & Description & \emph{Clat} & \emph{Amac} & \emph{Cang} & \emph{Spun} & \emph{Bden} & \emph{Hpol} \\ 
  \hline
PF00069 & Pkinase & 297 & 409 & 185 & 354 & 404 & 149 \\ 
  PF00400 & WD40 & 211 & 1287 & 185 & 1710 & 1075 & 170 \\ 
  PF00076 & RRM 1 & 143 & 201 &  94 & 278 & 240 &  53 \\ 
  PF00153 & Mito carr & 119 & 222 &  42 & 294 & 231 &  49 \\ 
  PF00012 & HSP70 & 112 &  42 &  11 &  18 &  16 &   9 \\ 
  PF00118 & Cpn60 TCP1 &  93 &  22 &  11 &  20 &  31 &  10 \\ 
  PF00071 & Ras &  86 &  80 &  57 & 108 &  96 &  47 \\ 
  PF00270 & DEAD &  70 & 115 &  94 & 136 & 126 &  81 \\ 
  PF00036 & efhand &  67 &  27 &  49 &  60 &  68 &  30 \\ 
  PF00112 & Peptidase C1 &  65 &   0 &   1 &   0 &   0 &   0 \\ 
  PF01576 & Myosin tail 1 &  59 &   1 &  42 &   4 &   3 &  10 \\ 
  PF00004 & AAA &  58 & 128 & 121 & 134 & 135 &  93 \\ 
  PF00271 & Helicase C &  52 & 140 &  77 & 184 & 199 &  71 \\ 
  PF00227 & Proteasome &  52 &  26 &  16 &  28 &  43 &  19 \\ 
  PF01532 & Glyco hydro 47 &  52 &   2 &   3 &   8 &  12 &   8 \\ 
  PF07690 & MFS 1 &  50 & 144 &  57 & 120 &  69 &  36 \\ 
  PF02985 & HEAT &  50 &  34 &  64 &  56 & 136 &  58 \\ 
  PF00005 & ABC tran &  50 & 217 & 180 & 168 & 231 &  96 \\ 
  PF00009 & GTP EFTU &  48 &  79 &  80 &  92 &  48 &  54 \\ 
  PF00089 & Trypsin &  47 &   2 &   9 &   8 &   4 &   2 \\ 
   \hline
\hline
\end{tabular}
\caption[Top 20 PFAM domain counts]{Comparisions of counts of top 20 PFAM domains from \textit{C. lativittatus} as identified in other chytrids. Details for organism abbreviations can be found in appendix.} 
\label{tab:ChClat_PFAM}
\end{table}
% latex table generated in R 3.0.1 by xtable 1.7-4 package
% Fri Apr 24 09:45:44 2015
\begin{table}[tbp]
\centering
\begin{tabular}{rllllllrrrr}
  \hline
\hline
 & EMBLID & UniprotID & B..emersonii.Description & TopHitID & Top.hit.Description & Source.organism & X..Cov & E.val & X..ID & PMID \\ 
  \hline
1 & KJ468786 & A0A060GS52 & putative phytoene dehydrogenase & P54982 & phytoene desaturase & Phycomyces blakesleeanus NRRL 1555 &  85 & 0.00 &  53 & 9079885 \\ 
  2 & KJ468785 & A0A060GVE0 & putative lycopene cyclase / phytoene synthase & Q9UUQ6 & bifunctional lycopene cyclase/phytoene synthase & Mucor circinelloides f. lusitanicus &  93 & 0.00 &  33 & 10951210 \\ 
  3 & KJ468787 & A0A060GW07 & putative carotenoid dioxygenase & Q9I993 & Beta,beta-carotene 15,15'-monooxygenase & Gallus gallus &  51 & 0.00 &  27 & 10799297 \\ 
   \hline
\hline
\end{tabular}
\caption[BLASTP results for \textit{B. emersonii} $\beta$-carotene genes]{Top BLASTP hits against SwissProt for three \textit{B. emersonii} $\beta$-carotene metabolism genes. PMID references publications for SwissProt hit describing experimental verification of biological function} 
\label{tab:ChClat_BemeVerify}
\end{table}
% latex table generated in R 3.0.1 by xtable 1.7-4 package
% Fri Apr 24 09:45:44 2015
\begin{table}[tbp]
\centering
\begin{tabular}{rlll}
  \hline
\hline
 & KEGG.ID & Description & Sequences.used \\ 
  \hline
1 & K10027 & phytoene desaturase & 442 Bacterial proteins, 49 Archaeal proteins \\ 
  2 & K02291 & phytoene synthase & 6 Eukaryote proteins, 45 Plant proteins [excluded 49 Archaeal and 843 Bacterial proteins] \\ 
  3 & K00515 & beta-carotene 15,15'-monooxygenase & 64 Metazoan proteins [excluded 1 Archaeal protein] \\ 
   \hline
\hline
\end{tabular}
\caption[$\beta$-carotene HMM]{Sequences from KEGG used in generating HMMs for $\beta$-carotene metabolism searches} 
\label{tab:ChClat_KEGGHMM}
\end{table}
%<<label=ChClat_OrthoMCL,echo=FALSE,results=tex>>=
%library(xtable)
%data <- read.table("../dat/Clat_uniqueClusters.tsv",sep="\t",header=T)
%xdata <- xtable(data,label="tab:ChClat_orthomcl",caption=c("orthoMCL Long caption","orthoMCL short caption"))
%hlines <- c(-1,-1,0,nrow(xdata),nrow(xdata))
%print(xdata,      table.placement="tbp",      caption.placement="bottom",      hline.after=hlines,)
%@


% latex table generated in R 3.0.1 by xtable 1.7-4 package
% Fri Apr 24 09:45:44 2015
\begin{table}[tbp]
\centering
\begin{tabular}{rlllllr}
  \hline
\hline
 & OrfID & Type & UniProtKB.BLAST.match & InterPro & TMPred & FPKM \\ 
  \hline
1 & m.12730 & WC-1 & Neurospora crassa\verb|^|Q01371\verb|^|4e-43 & -- & -- & 2.37 \\ 
  2 & m.7819 & OpGC & Halobacterium sp. NRC-1\verb|^|P71411\verb|^|7.4e-07;Drosophila melanogaster\verb|^|Q8INF0\verb|^|8e-44 & PF01036\verb|^|32:241\verb|^|5.0e-37;PF00211\verb|^|355:533\verb|^|4.6e-51 & -- & 18.00 \\ 
  3 & m.9338 & Op & Danio rerio\verb|^|Q2KNE5\verb|^|3e-08 & PF10317\verb|^|21:205\verb|^|1.9e-7;PF00001\verb|^|50:355\verb|^|1.5e-17 & -- & 12.40 \\ 
  4 & m.15402 & Op & Apis mellifera\verb|^|Q17053\verb|^|8e-10 & PF00001\verb|^|36:341\verb|^|1.0e-21 & -- & 5.17 \\ 
  5 & m.16182 & Op & Cambarus hubrichti\verb|^|O18312\verb|^|1e-04 & PF00001\verb|^|14:255\verb|^|9.7e-13 & -- & 4.70 \\ 
  6 & m.18794 & Op & Limulus polyphemus\verb|^|P35360\verb|^|3e-10 & PF00001\verb|^|20:323\verb|^|5.9e-22 & -- & 1.62 \\ 
  7 & m.11198 & Op & Halobacterium sp. AUS-2\verb|^|P29563\verb|^|3e-30 & PF01036\verb|^|32:241\verb|^|5.0e-37 & -- & 1.54 \\ 
   \hline
\hline
\end{tabular}
\caption[\textit{C. lativittatus} photosensing proteins]{Predicted \textit{C. lativittatus} proteins associated with photosensing. PFAM Family definitions: PF13426, "PAS domain" [PAS\_9]; PF08447, "PAS fold" [PAS\_3]; PF00001, "7 transmembrane receptor (rhodopsin family)" [7tm\_1]; PF01036, "bacteriorhodopsin-like protein" [Bac\_rhodopsin]; PF10317, "Serpentine type 7TM GPCR chemoreceptor Srd" [7TM\_GPCR\_Srd]; PF00211, "Adenylate and Guanylate cyclase catalytic domain" [Guanylate\_cyc]} 
\label{tab:ChClat_photosensing}
\end{table}
%<<label=ChClat_FAADB,echo=FALSE,results=tex>>=
%library(xtable)
%specify_decimal <- function(x,k=2) format(round(x,k),nsmall=k)
%data <- read.table("../dat/FAADB_SVM_Blast.tsv",sep="\t",header=T,row.names=1)
%data$SVM.Score <- as.numeric(specify_decimal(data$SVM.Score))
%xdata <- xtable(data,label="tab:ChClat_FAADB",caption=c("FAADB Long caption","FAADB short caption"))
%hlines <- c(-1,-1,0,nrow(xdata),nrow(xdata))
%print(xdata,      table.placement="tbp",      caption.placement="bottom",      hline.after=hlines,)
%@

